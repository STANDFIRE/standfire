% Generated by Sphinx.
\def\sphinxdocclass{report}
\newif\ifsphinxKeepOldNames \sphinxKeepOldNamestrue
\documentclass[letterpaper,10pt,english]{sphinxmanual}
\usepackage{iftex}

\ifPDFTeX
  \usepackage[utf8]{inputenc}
\fi
\ifdefined\DeclareUnicodeCharacter
  \DeclareUnicodeCharacter{00A0}{\nobreakspace}
\fi
\usepackage{cmap}
\usepackage[T1]{fontenc}
\usepackage{amsmath,amssymb,amstext}
\usepackage{babel}
\usepackage{times}
\usepackage[Sonny]{fncychap}
\usepackage{longtable}
\usepackage{sphinx}
\usepackage{multirow}
\usepackage{eqparbox}


\addto\captionsenglish{\renewcommand{\figurename}{Fig.\@ }}
\addto\captionsenglish{\renewcommand{\tablename}{Table }}
\SetupFloatingEnvironment{literal-block}{name=Listing }

\addto\extrasenglish{\def\pageautorefname{page}}

\setcounter{tocdepth}{2}


\title{STANDFIRE}
\date{Sep 30, 2016}
\release{}
\author{}
\newcommand{\sphinxlogo}{}
\renewcommand{\releasename}{Release}
\makeindex

\makeatletter
\def\PYG@reset{\let\PYG@it=\relax \let\PYG@bf=\relax%
    \let\PYG@ul=\relax \let\PYG@tc=\relax%
    \let\PYG@bc=\relax \let\PYG@ff=\relax}
\def\PYG@tok#1{\csname PYG@tok@#1\endcsname}
\def\PYG@toks#1+{\ifx\relax#1\empty\else%
    \PYG@tok{#1}\expandafter\PYG@toks\fi}
\def\PYG@do#1{\PYG@bc{\PYG@tc{\PYG@ul{%
    \PYG@it{\PYG@bf{\PYG@ff{#1}}}}}}}
\def\PYG#1#2{\PYG@reset\PYG@toks#1+\relax+\PYG@do{#2}}

\expandafter\def\csname PYG@tok@gd\endcsname{\def\PYG@tc##1{\textcolor[rgb]{0.63,0.00,0.00}{##1}}}
\expandafter\def\csname PYG@tok@gu\endcsname{\let\PYG@bf=\textbf\def\PYG@tc##1{\textcolor[rgb]{0.50,0.00,0.50}{##1}}}
\expandafter\def\csname PYG@tok@gt\endcsname{\def\PYG@tc##1{\textcolor[rgb]{0.00,0.27,0.87}{##1}}}
\expandafter\def\csname PYG@tok@gs\endcsname{\let\PYG@bf=\textbf}
\expandafter\def\csname PYG@tok@gr\endcsname{\def\PYG@tc##1{\textcolor[rgb]{1.00,0.00,0.00}{##1}}}
\expandafter\def\csname PYG@tok@cm\endcsname{\let\PYG@it=\textit\def\PYG@tc##1{\textcolor[rgb]{0.25,0.50,0.56}{##1}}}
\expandafter\def\csname PYG@tok@vg\endcsname{\def\PYG@tc##1{\textcolor[rgb]{0.73,0.38,0.84}{##1}}}
\expandafter\def\csname PYG@tok@vi\endcsname{\def\PYG@tc##1{\textcolor[rgb]{0.73,0.38,0.84}{##1}}}
\expandafter\def\csname PYG@tok@mh\endcsname{\def\PYG@tc##1{\textcolor[rgb]{0.13,0.50,0.31}{##1}}}
\expandafter\def\csname PYG@tok@cs\endcsname{\def\PYG@tc##1{\textcolor[rgb]{0.25,0.50,0.56}{##1}}\def\PYG@bc##1{\setlength{\fboxsep}{0pt}\colorbox[rgb]{1.00,0.94,0.94}{\strut ##1}}}
\expandafter\def\csname PYG@tok@ge\endcsname{\let\PYG@it=\textit}
\expandafter\def\csname PYG@tok@vc\endcsname{\def\PYG@tc##1{\textcolor[rgb]{0.73,0.38,0.84}{##1}}}
\expandafter\def\csname PYG@tok@il\endcsname{\def\PYG@tc##1{\textcolor[rgb]{0.13,0.50,0.31}{##1}}}
\expandafter\def\csname PYG@tok@go\endcsname{\def\PYG@tc##1{\textcolor[rgb]{0.20,0.20,0.20}{##1}}}
\expandafter\def\csname PYG@tok@cp\endcsname{\def\PYG@tc##1{\textcolor[rgb]{0.00,0.44,0.13}{##1}}}
\expandafter\def\csname PYG@tok@gi\endcsname{\def\PYG@tc##1{\textcolor[rgb]{0.00,0.63,0.00}{##1}}}
\expandafter\def\csname PYG@tok@gh\endcsname{\let\PYG@bf=\textbf\def\PYG@tc##1{\textcolor[rgb]{0.00,0.00,0.50}{##1}}}
\expandafter\def\csname PYG@tok@ni\endcsname{\let\PYG@bf=\textbf\def\PYG@tc##1{\textcolor[rgb]{0.84,0.33,0.22}{##1}}}
\expandafter\def\csname PYG@tok@nl\endcsname{\let\PYG@bf=\textbf\def\PYG@tc##1{\textcolor[rgb]{0.00,0.13,0.44}{##1}}}
\expandafter\def\csname PYG@tok@nn\endcsname{\let\PYG@bf=\textbf\def\PYG@tc##1{\textcolor[rgb]{0.05,0.52,0.71}{##1}}}
\expandafter\def\csname PYG@tok@no\endcsname{\def\PYG@tc##1{\textcolor[rgb]{0.38,0.68,0.84}{##1}}}
\expandafter\def\csname PYG@tok@na\endcsname{\def\PYG@tc##1{\textcolor[rgb]{0.25,0.44,0.63}{##1}}}
\expandafter\def\csname PYG@tok@nb\endcsname{\def\PYG@tc##1{\textcolor[rgb]{0.00,0.44,0.13}{##1}}}
\expandafter\def\csname PYG@tok@nc\endcsname{\let\PYG@bf=\textbf\def\PYG@tc##1{\textcolor[rgb]{0.05,0.52,0.71}{##1}}}
\expandafter\def\csname PYG@tok@nd\endcsname{\let\PYG@bf=\textbf\def\PYG@tc##1{\textcolor[rgb]{0.33,0.33,0.33}{##1}}}
\expandafter\def\csname PYG@tok@ne\endcsname{\def\PYG@tc##1{\textcolor[rgb]{0.00,0.44,0.13}{##1}}}
\expandafter\def\csname PYG@tok@nf\endcsname{\def\PYG@tc##1{\textcolor[rgb]{0.02,0.16,0.49}{##1}}}
\expandafter\def\csname PYG@tok@si\endcsname{\let\PYG@it=\textit\def\PYG@tc##1{\textcolor[rgb]{0.44,0.63,0.82}{##1}}}
\expandafter\def\csname PYG@tok@s2\endcsname{\def\PYG@tc##1{\textcolor[rgb]{0.25,0.44,0.63}{##1}}}
\expandafter\def\csname PYG@tok@nt\endcsname{\let\PYG@bf=\textbf\def\PYG@tc##1{\textcolor[rgb]{0.02,0.16,0.45}{##1}}}
\expandafter\def\csname PYG@tok@nv\endcsname{\def\PYG@tc##1{\textcolor[rgb]{0.73,0.38,0.84}{##1}}}
\expandafter\def\csname PYG@tok@s1\endcsname{\def\PYG@tc##1{\textcolor[rgb]{0.25,0.44,0.63}{##1}}}
\expandafter\def\csname PYG@tok@ch\endcsname{\let\PYG@it=\textit\def\PYG@tc##1{\textcolor[rgb]{0.25,0.50,0.56}{##1}}}
\expandafter\def\csname PYG@tok@m\endcsname{\def\PYG@tc##1{\textcolor[rgb]{0.13,0.50,0.31}{##1}}}
\expandafter\def\csname PYG@tok@gp\endcsname{\let\PYG@bf=\textbf\def\PYG@tc##1{\textcolor[rgb]{0.78,0.36,0.04}{##1}}}
\expandafter\def\csname PYG@tok@sh\endcsname{\def\PYG@tc##1{\textcolor[rgb]{0.25,0.44,0.63}{##1}}}
\expandafter\def\csname PYG@tok@ow\endcsname{\let\PYG@bf=\textbf\def\PYG@tc##1{\textcolor[rgb]{0.00,0.44,0.13}{##1}}}
\expandafter\def\csname PYG@tok@sx\endcsname{\def\PYG@tc##1{\textcolor[rgb]{0.78,0.36,0.04}{##1}}}
\expandafter\def\csname PYG@tok@bp\endcsname{\def\PYG@tc##1{\textcolor[rgb]{0.00,0.44,0.13}{##1}}}
\expandafter\def\csname PYG@tok@c1\endcsname{\let\PYG@it=\textit\def\PYG@tc##1{\textcolor[rgb]{0.25,0.50,0.56}{##1}}}
\expandafter\def\csname PYG@tok@o\endcsname{\def\PYG@tc##1{\textcolor[rgb]{0.40,0.40,0.40}{##1}}}
\expandafter\def\csname PYG@tok@kc\endcsname{\let\PYG@bf=\textbf\def\PYG@tc##1{\textcolor[rgb]{0.00,0.44,0.13}{##1}}}
\expandafter\def\csname PYG@tok@c\endcsname{\let\PYG@it=\textit\def\PYG@tc##1{\textcolor[rgb]{0.25,0.50,0.56}{##1}}}
\expandafter\def\csname PYG@tok@mf\endcsname{\def\PYG@tc##1{\textcolor[rgb]{0.13,0.50,0.31}{##1}}}
\expandafter\def\csname PYG@tok@err\endcsname{\def\PYG@bc##1{\setlength{\fboxsep}{0pt}\fcolorbox[rgb]{1.00,0.00,0.00}{1,1,1}{\strut ##1}}}
\expandafter\def\csname PYG@tok@mb\endcsname{\def\PYG@tc##1{\textcolor[rgb]{0.13,0.50,0.31}{##1}}}
\expandafter\def\csname PYG@tok@ss\endcsname{\def\PYG@tc##1{\textcolor[rgb]{0.32,0.47,0.09}{##1}}}
\expandafter\def\csname PYG@tok@sr\endcsname{\def\PYG@tc##1{\textcolor[rgb]{0.14,0.33,0.53}{##1}}}
\expandafter\def\csname PYG@tok@mo\endcsname{\def\PYG@tc##1{\textcolor[rgb]{0.13,0.50,0.31}{##1}}}
\expandafter\def\csname PYG@tok@kd\endcsname{\let\PYG@bf=\textbf\def\PYG@tc##1{\textcolor[rgb]{0.00,0.44,0.13}{##1}}}
\expandafter\def\csname PYG@tok@mi\endcsname{\def\PYG@tc##1{\textcolor[rgb]{0.13,0.50,0.31}{##1}}}
\expandafter\def\csname PYG@tok@kn\endcsname{\let\PYG@bf=\textbf\def\PYG@tc##1{\textcolor[rgb]{0.00,0.44,0.13}{##1}}}
\expandafter\def\csname PYG@tok@cpf\endcsname{\let\PYG@it=\textit\def\PYG@tc##1{\textcolor[rgb]{0.25,0.50,0.56}{##1}}}
\expandafter\def\csname PYG@tok@kr\endcsname{\let\PYG@bf=\textbf\def\PYG@tc##1{\textcolor[rgb]{0.00,0.44,0.13}{##1}}}
\expandafter\def\csname PYG@tok@s\endcsname{\def\PYG@tc##1{\textcolor[rgb]{0.25,0.44,0.63}{##1}}}
\expandafter\def\csname PYG@tok@kp\endcsname{\def\PYG@tc##1{\textcolor[rgb]{0.00,0.44,0.13}{##1}}}
\expandafter\def\csname PYG@tok@w\endcsname{\def\PYG@tc##1{\textcolor[rgb]{0.73,0.73,0.73}{##1}}}
\expandafter\def\csname PYG@tok@kt\endcsname{\def\PYG@tc##1{\textcolor[rgb]{0.56,0.13,0.00}{##1}}}
\expandafter\def\csname PYG@tok@sc\endcsname{\def\PYG@tc##1{\textcolor[rgb]{0.25,0.44,0.63}{##1}}}
\expandafter\def\csname PYG@tok@sb\endcsname{\def\PYG@tc##1{\textcolor[rgb]{0.25,0.44,0.63}{##1}}}
\expandafter\def\csname PYG@tok@k\endcsname{\let\PYG@bf=\textbf\def\PYG@tc##1{\textcolor[rgb]{0.00,0.44,0.13}{##1}}}
\expandafter\def\csname PYG@tok@se\endcsname{\let\PYG@bf=\textbf\def\PYG@tc##1{\textcolor[rgb]{0.25,0.44,0.63}{##1}}}
\expandafter\def\csname PYG@tok@sd\endcsname{\let\PYG@it=\textit\def\PYG@tc##1{\textcolor[rgb]{0.25,0.44,0.63}{##1}}}

\def\PYGZbs{\char`\\}
\def\PYGZus{\char`\_}
\def\PYGZob{\char`\{}
\def\PYGZcb{\char`\}}
\def\PYGZca{\char`\^}
\def\PYGZam{\char`\&}
\def\PYGZlt{\char`\<}
\def\PYGZgt{\char`\>}
\def\PYGZsh{\char`\#}
\def\PYGZpc{\char`\%}
\def\PYGZdl{\char`\$}
\def\PYGZhy{\char`\-}
\def\PYGZsq{\char`\'}
\def\PYGZdq{\char`\"}
\def\PYGZti{\char`\~}
% for compatibility with earlier versions
\def\PYGZat{@}
\def\PYGZlb{[}
\def\PYGZrb{]}
\makeatother

\renewcommand\PYGZsq{\textquotesingle}

\begin{document}

\maketitle
\tableofcontents
\phantomsection\label{index::doc}


Contents:


\chapter{STANDFIRE User Guide}
\label{user_guide:standfire-user-guide}\label{user_guide::doc}\label{user_guide:welcome-to-standfire}
Contents:


\section{Getting Started}
\label{getting_started:getting-started}\label{getting_started::doc}

\subsection{Prerequisites}
\label{getting_started:prerequisites}

\section{Tutorials}
\label{tutorials:tutorials}\label{tutorials::doc}
Contents:


\subsection{Interfacing with FVS}
\label{interfacing_with_fvs:interfacing-with-fvs}\label{interfacing_with_fvs::doc}
Use Suppose to generate a keyword file. Or use the following example .key

\begin{Verbatim}[commandchars=\\\{\}]
NOSCREEN
RANNSEED           0
!STATS
STDIDENT
STANDFIRE\PYGZus{}example
DESIGN           \PYGZhy{}10       500         5         9
STDINFO          103       140      60.0       0.0       0.0      36.0
INVYEAR         2010
NUMCYCLE          10
TREEDATA
FMIN
END
STATS
SVS                0                   0         0        15
FMIn
Potfire
FuelOut
BurnRept
MortRept
FuelRept
SnagSum
End
PROCESS
STOP
\end{Verbatim}

If don't have a FVS tree list file, then copy and paste the following text and save  it to the same directory where the keyword file lives, give it the same prefix as the \sphinxcode{.key} but with a \sphinxcode{.tre} extension.

\begin{Verbatim}[commandchars=\\\{\}]
1   95       9PP 105    35                          0 0
1   96       0PP 43     17        1                 0 0
1   97       0PP 148    43        2                 0 0
1   98       0PP 49     30        1                 0 0
1   99       9PP 54     30                          0 0
1   100      0PP 100    40        3                 0 0
1   101      0PP 42     30        2                 0 0
1   102      0PP 53     34        1                 0 0
1   103      0PP 97     42        3                 0 0
1   104      0PP 61     35        1                 0 0
1   105      0PP 81     40        1                 0 0
1   106      9PP 80     33                          0 0
1   107      0PP 41     32        2                 0 0
1   108      9PP 71     40                          0 0
1   109      9PP 73     41                          0 0
1   110      9PP 94     35                          0 0
1   111      9PP 103    32                          0 0
\end{Verbatim}

Once you have a keyword file and a tree list file in the same directory we can start to build a script to do some work.

First we import the Fvsfuels class from the fuels module.

\begin{Verbatim}[commandchars=\\\{\}]
\PYG{g+gp}{\PYGZgt{}\PYGZgt{}\PYGZgt{} }\PYG{k+kn}{from} \PYG{n+nn}{standfire.fuels} \PYG{k+kn}{import} \PYG{n}{Fvsfuels}
\end{Verbatim}

Next create an instance of the class passes the desired variant as an argument and register the keyword file.

\begin{Verbatim}[commandchars=\\\{\}]
\PYG{g+gp}{\PYGZgt{}\PYGZgt{}\PYGZgt{} }\PYG{n}{stand\PYGZus{}1} \PYG{o}{=} \PYG{n}{Fvsfuel}\PYG{p}{(}\PYG{l+s+s2}{\PYGZdq{}}\PYG{l+s+s2}{iec}\PYG{l+s+s2}{\PYGZdq{}}\PYG{p}{)}
\PYG{g+gp}{\PYGZgt{}\PYGZgt{}\PYGZgt{} }\PYG{n}{stand\PYGZus{}1}\PYG{o}{.}\PYG{n}{set\PYGZus{}keyword}\PYG{p}{(}\PYG{l+s+s2}{\PYGZdq{}}\PYG{l+s+s2}{/Users/standfire/fvs\PYGZus{}exp/example.key}\PYG{l+s+s2}{\PYGZdq{}}\PYG{p}{)}
\PYG{g+go}{TIMEINT not found in keyword file, default is 10 years}
\end{Verbatim}

We get a message telling us that the TIMEINT keyword was not found in the keyword file. No problem, STANDFIRE automatically sets this value to 10 years.

\begin{Verbatim}[commandchars=\\\{\}]
\PYG{g+gp}{\PYGZgt{}\PYGZgt{}\PYGZgt{} }\PYG{n}{stand\PYGZus{}1}\PYG{o}{.}\PYG{n}{keywords}
\PYG{g+go}{\PYGZob{}\PYGZsq{}TIMEINT\PYGZsq{}: 10, \PYGZsq{}NUMCYCLE\PYGZsq{}: 10, \PYGZsq{}INVYEAR\PYGZsq{}: 2010, \PYGZsq{}SVS\PYGZsq{}: 15, \PYGZsq{}FUELOUT\PYGZsq{}: 1\PYGZcb{}}
\end{Verbatim}

Notice the keys in the keywords dictionary.  \sphinxcode{TIMEINT} is the time interval of the FVS simulation in year, \sphinxcode{NUMCYCLE} is the number of cycles, \sphinxcode{INVYEAR} is the year of the inventory, and \sphinxcode{SVS} and \sphinxcode{FUELOUT} are there to check if these keywords are in the keyword file. If the \sphinxcode{SVS} and \sphinxcode{FUELOUT} keywords are not defined the keyword file then FVS will not calculate tree positions or fuel attributes. So be sure you add these to your keyword file before registering the .key with FVS. You can use \emph{post processors*} in Suppose to do so.  \sphinxcode{TIMEINT}, \sphinxcode{NUMCYCLE}, and \sphinxcode{INVYEAR} can be manually changed by calling setters for each. For instance, if you only want to calculate fuel attributes for trees during the year of the inventory then simply change the \sphinxcode{NUMCYCLE} value in the keyword dictionary.

\begin{Verbatim}[commandchars=\\\{\}]
\PYG{g+gp}{\PYGZgt{}\PYGZgt{}\PYGZgt{} }\PYG{n}{stand\PYGZus{}1}\PYG{o}{.}\PYG{n}{set\PYGZus{}num\PYGZus{}cycle}\PYG{p}{(}\PYG{l+m+mi}{0}\PYG{p}{)}
\PYG{g+gp}{\PYGZgt{}\PYGZgt{}\PYGZgt{} }\PYG{n}{stand\PYGZus{}1}\PYG{o}{.}\PYG{n}{keywords}
\PYG{g+go}{\PYGZob{}\PYGZsq{}TIMEINT\PYGZsq{}: 10, \PYGZsq{}NUMCYCLE\PYGZsq{}: 0, \PYGZsq{}INVYEAR\PYGZsq{}: 2010, \PYGZsq{}SVS\PYGZsq{}: 15, \PYGZsq{}FUELOUT\PYGZsq{}: 1\PYGZcb{}}
\end{Verbatim}

Now that we have our simulation parameters established, we startup FVS.

\begin{Verbatim}[commandchars=\\\{\}]
\PYG{g+gp}{\PYGZgt{}\PYGZgt{}\PYGZgt{} }\PYG{n}{stand\PYGZus{}1}\PYG{o}{.}\PYG{n}{run\PYGZus{}fvs}\PYG{p}{(}\PYG{p}{)}
\end{Verbatim}


\chapter{STANDFIRE API Reference}
\label{api_ref:standfire-api-reference}\label{api_ref::doc}
Contents:


\section{Fuels module}
\label{fuels:module-fuels}\label{fuels::doc}\label{fuels:fuels-module}\index{fuels (module)}
This module is the interface to FVS. Given a FVS variant name, a keyword file
and the corresponding tree file, a user can run a FVS simulation and request
various fuels information from individual trees. The Fvsfuels class will also
produce the 4 fuels files needed for the Capsis fuel matrix generator.

Currently FVS MC Access database querying not available on all platforms. 
If a user has a keyword file that points to a MS Access database, then the user
can generate a tree file by exporting the Access database to an comma-delimited
file and pass it through the Inventory class. The output will be the same
inventory present in the mdb file, but formatted to FVS .tre file standards.

Contents:


\subsection{Fvsfuels}
\label{fuels_Fvsfuels::doc}\label{fuels_Fvsfuels:fvsfuels}
\noindent\sphinxincludegraphics{{fvsfuels}.png}
\index{Fvsfuels (class in fuels)}

\begin{fulllineitems}
\phantomsection\label{fuels_Fvsfuels:fuels.Fvsfuels}\pysiglinewithargsret{\sphinxstrong{class }\sphinxcode{fuels.}\sphinxbfcode{Fvsfuels}}{\emph{variant}}{}
Bases: \sphinxcode{object}

A Fvsfuels object is used to calculate component fuels at the individual
tree level using the Forest Vegetation Simulator. To create an instance
of this class you need two items: a keyword file (.key) and tree list file
(.tre) with the same prefix as the keyword file. If you don't already have
a tree list file then you can use \sphinxcode{{}`fuels.Inventory{}`} class to generate
one.
\begin{quote}\begin{description}
\item[{Parameters}] \leavevmode
\textbf{\texttt{variant}} (\emph{\texttt{string}}) -- FVS variant to be imported

\end{description}\end{quote}

\textbf{Example:}

A basic example to extract live canopy biomass for individual trees during
year of inventory

\begin{Verbatim}[commandchars=\\\{\}]
\PYG{g+gp}{\PYGZgt{}\PYGZgt{}\PYGZgt{} }\PYG{k+kn}{from} \PYG{n+nn}{standfire}\PYG{n+nn}{.}\PYG{n+nn}{fuels} \PYG{k}{import} \PYG{n}{Fvsfuels}
\PYG{g+gp}{\PYGZgt{}\PYGZgt{}\PYGZgt{} }\PYG{n}{stand001} \PYG{o}{=} \PYG{n}{Fvsfuels}\PYG{p}{(}\PYG{l+s+s2}{\PYGZdq{}}\PYG{l+s+s2}{iec}\PYG{l+s+s2}{\PYGZdq{}}\PYG{p}{)}
\PYG{g+gp}{\PYGZgt{}\PYGZgt{}\PYGZgt{} }\PYG{n}{stand001}\PYG{o}{.}\PYG{n}{set\PYGZus{}keyword}\PYG{p}{(}\PYG{l+s+s2}{\PYGZdq{}}\PYG{l+s+s2}{/Users/standfire/test/example.key}\PYG{l+s+s2}{\PYGZdq{}}\PYG{p}{)}
\PYG{g+go}{TIMEINT not found in keyword file, default is 10 years}
\PYG{g+gp}{\PYGZgt{}\PYGZgt{}\PYGZgt{} }\PYG{n}{stand001}\PYG{o}{.}\PYG{n}{keywords}
\PYG{g+go}{\PYGZob{}\PYGZsq{}TIMEINT\PYGZsq{}: 10, \PYGZsq{}NUMCYCLE\PYGZsq{}: 10, \PYGZsq{}INVYEAR\PYGZsq{}: 2010, \PYGZsq{}SVS\PYGZsq{}: 15, \PYGZsq{}FUELOUT\PYGZsq{}: 1\PYGZcb{}}
\end{Verbatim}

The keyword file is setup to simulate 100 years at a time interval of 10
years. Lets change this to only simulate the inventory year.

\begin{Verbatim}[commandchars=\\\{\}]
\PYG{g+gp}{\PYGZgt{}\PYGZgt{}\PYGZgt{} }\PYG{n}{stand001}\PYG{o}{.}\PYG{n}{set\PYGZus{}num\PYGZus{}cycles}\PYG{p}{(}\PYG{l+m+mi}{0}\PYG{p}{)}
\PYG{g+gp}{\PYGZgt{}\PYGZgt{}\PYGZgt{} }\PYG{n}{stand001}\PYG{o}{.}\PYG{n}{keywords}
\PYG{g+go}{\PYGZob{}\PYGZsq{}TIMEINT\PYGZsq{}: 10, \PYGZsq{}NUMCYCLE\PYGZsq{}: 0, \PYGZsq{}INVYEAR\PYGZsq{}: 2010, \PYGZsq{}SVS\PYGZsq{}: 15, \PYGZsq{}FUELOUT\PYGZsq{}: 1\PYGZcb{}}
\PYG{g+gp}{\PYGZgt{}\PYGZgt{}\PYGZgt{} }\PYG{n}{stand001}\PYG{o}{.}\PYG{n}{run\PYGZus{}fvs}\PYG{p}{(}\PYG{p}{)}
\end{Verbatim}

Now we can write the trees data frame to disk

\begin{Verbatim}[commandchars=\\\{\}]
\PYG{g+gp}{\PYGZgt{}\PYGZgt{}\PYGZgt{} }\PYG{n}{stand001}\PYG{o}{.}\PYG{n}{save\PYGZus{}trees\PYGZus{}by\PYGZus{}year}\PYG{p}{(}\PYG{l+m+mi}{2010}\PYG{p}{)}
\end{Verbatim}

\begin{notice}{note}{Note:}
The argument must match one of the available variant in the
PyFVS module. Search through standfire/pyfvs/ to see all
variants
\end{notice}
\index{get\_simulation\_years() (fuels.Fvsfuels method)}

\begin{fulllineitems}
\phantomsection\label{fuels_Fvsfuels:fuels.Fvsfuels.get_simulation_years}\pysiglinewithargsret{\sphinxbfcode{get\_simulation\_years}}{}{}
Returns a list of the simulated years
\begin{quote}\begin{description}
\item[{Returns}] \leavevmode
simulated year

\item[{Return type}] \leavevmode
list of integers

\end{description}\end{quote}

\end{fulllineitems}

\index{get\_snags() (fuels.Fvsfuels method)}

\begin{fulllineitems}
\phantomsection\label{fuels_Fvsfuels:fuels.Fvsfuels.get_snags}\pysiglinewithargsret{\sphinxbfcode{get\_snags}}{\emph{year}}{}
Returns pandas data frame of the snags by indexed year
\begin{quote}\begin{description}
\item[{Parameters}] \leavevmode
\textbf{\texttt{year}} (\emph{\texttt{int}}) -- simulation year of the data frame to return

\item[{Returns}] \leavevmode
data frame of snags at indexed year

\item[{Return type}] \leavevmode
pandas dataframe

\end{description}\end{quote}

\begin{notice}{note}{Note:}
If a data frame for the specified year does not exist then
a message will be printed to the console.
\end{notice}

\end{fulllineitems}

\index{get\_standid() (fuels.Fvsfuels method)}

\begin{fulllineitems}
\phantomsection\label{fuels_Fvsfuels:fuels.Fvsfuels.get_standid}\pysiglinewithargsret{\sphinxbfcode{get\_standid}}{}{}
Returns stand ID as defined in the keyword file of the class instance
\begin{quote}\begin{description}
\item[{Returns}] \leavevmode
stand ID value

\item[{Return type}] \leavevmode
string

\end{description}\end{quote}

\end{fulllineitems}

\index{get\_trees() (fuels.Fvsfuels method)}

\begin{fulllineitems}
\phantomsection\label{fuels_Fvsfuels:fuels.Fvsfuels.get_trees}\pysiglinewithargsret{\sphinxbfcode{get\_trees}}{\emph{year}}{}
Returns pandas data frame of the trees by indexed year
\begin{quote}\begin{description}
\item[{Parameters}] \leavevmode
\textbf{\texttt{year}} (\emph{\texttt{int}}) -- simulation year of the data frame to return

\item[{Returns}] \leavevmode
data frame of trees at indexed year

\item[{Return type}] \leavevmode
pandas dataframe

\end{description}\end{quote}

\begin{notice}{note}{Note:}
If a data frame for the specified year does not exist then
a message will be printed to the console.
\end{notice}

\end{fulllineitems}

\index{run\_fvs() (fuels.Fvsfuels method)}

\begin{fulllineitems}
\phantomsection\label{fuels_Fvsfuels:fuels.Fvsfuels.run_fvs}\pysiglinewithargsret{\sphinxbfcode{run\_fvs}}{}{}
Runs the FVS simulation

This method runs a FVS simulation using the previously specified keyword
file. The simulation will be paused at each time interval and the trees
and snag data collected and appended to the fuels attribute of the
Fvsfuels object.

\textbf{Example:}

\begin{Verbatim}[commandchars=\\\{\}]
\PYG{g+gp}{\PYGZgt{}\PYGZgt{}\PYGZgt{} }\PYG{k+kn}{from} \PYG{n+nn}{standfire}\PYG{n+nn}{.}\PYG{n+nn}{fuels} \PYG{k}{import} \PYG{n}{Fvsfuels}
\PYG{g+gp}{\PYGZgt{}\PYGZgt{}\PYGZgt{} }\PYG{n}{stand010} \PYG{o}{=} \PYG{n}{Fvsfuels}\PYG{p}{(}\PYG{l+s+s2}{\PYGZdq{}}\PYG{l+s+s2}{iec}\PYG{l+s+s2}{\PYGZdq{}}\PYG{p}{)}
\PYG{g+gp}{\PYGZgt{}\PYGZgt{}\PYGZgt{} }\PYG{n}{stand010}\PYG{o}{.}\PYG{n}{set\PYGZus{}keyword}\PYG{p}{(}\PYG{l+s+s2}{\PYGZdq{}}\PYG{l+s+s2}{/Users/standfire/example/test.key}\PYG{l+s+s2}{\PYGZdq{}}\PYG{p}{)}
\PYG{g+gp}{\PYGZgt{}\PYGZgt{}\PYGZgt{} }\PYG{n}{stand010}\PYG{o}{.}\PYG{n}{run\PYGZus{}fvs}\PYG{p}{(}\PYG{p}{)}
\PYG{g+gp}{\PYGZgt{}\PYGZgt{}\PYGZgt{} }\PYG{n}{stand010}\PYG{o}{.}\PYG{n}{fuels}\PYG{p}{[}\PYG{l+s+s2}{\PYGZdq{}}\PYG{l+s+s2}{trees}\PYG{l+s+s2}{\PYGZdq{}}\PYG{p}{]}\PYG{p}{[}\PYG{l+m+mi}{2010}\PYG{p}{]}
\PYG{g+go}{xloc    yloc    species   dbh     ht    crd    cratio  crownwt0  crownwt1 ...}
\PYG{g+go}{33.49  108.58   PIPO     19.43   68.31  8.77     25    33.46      4.3}
\PYG{g+go}{24.3    90.4    PIPO     11.46   56.6   5.63     15     6.55     2.33}
\PYG{g+go}{88.84  162.98   PIPO     18.63   67.76  9.48     45    75.88     6.89}
\PYG{g+gp}{...}
\end{Verbatim}

\end{fulllineitems}

\index{save\_all() (fuels.Fvsfuels method)}

\begin{fulllineitems}
\phantomsection\label{fuels_Fvsfuels:fuels.Fvsfuels.save_all}\pysiglinewithargsret{\sphinxbfcode{save\_all}}{}{}
Writes all data frame in the \sphinxcode{fuels} attribute of the class to the
specified working directory. Output file are .csv.

\end{fulllineitems}

\index{save\_snags\_by\_year() (fuels.Fvsfuels method)}

\begin{fulllineitems}
\phantomsection\label{fuels_Fvsfuels:fuels.Fvsfuels.save_snags_by_year}\pysiglinewithargsret{\sphinxbfcode{save\_snags\_by\_year}}{\emph{year}}{}
Writes snag data frame at indexed year to .csv in working directory

\end{fulllineitems}

\index{save\_trees\_by\_year() (fuels.Fvsfuels method)}

\begin{fulllineitems}
\phantomsection\label{fuels_Fvsfuels:fuels.Fvsfuels.save_trees_by_year}\pysiglinewithargsret{\sphinxbfcode{save\_trees\_by\_year}}{\emph{year}}{}
Writes tree data frame at indexed year to .csv in working directory

\end{fulllineitems}

\index{set\_dir() (fuels.Fvsfuels method)}

\begin{fulllineitems}
\phantomsection\label{fuels_Fvsfuels:fuels.Fvsfuels.set_dir}\pysiglinewithargsret{\sphinxbfcode{set\_dir}}{\emph{wdir}}{}
Sets the working directory of a Fvsfuels object

This method is called by \sphinxcode{Fvsfuels.set\_keyword()}. Thus, the default
working directory is the folder containing the specified keyword file.
If you wish to store simulation outputs in a different directory then
use this methods to do so.
\begin{quote}\begin{description}
\item[{Parameters}] \leavevmode
\textbf{\texttt{wdir}} (\emph{\texttt{string}}) -- path/to/desired\_directory

\end{description}\end{quote}

\textbf{Example:}

\begin{Verbatim}[commandchars=\\\{\}]
\PYG{g+gp}{\PYGZgt{}\PYGZgt{}\PYGZgt{} }\PYG{k+kn}{from} \PYG{n+nn}{standfire}\PYG{n+nn}{.}\PYG{n+nn}{fuel} \PYG{k}{import} \PYG{n}{Fvsfuels}
\PYG{g+gp}{\PYGZgt{}\PYGZgt{}\PYGZgt{} }\PYG{n}{test} \PYG{o}{=} \PYG{n}{Fvsfuels}\PYG{p}{(}\PYG{l+s+s2}{\PYGZdq{}}\PYG{l+s+s2}{emc}\PYG{l+s+s2}{\PYGZdq{}}\PYG{p}{)}
\PYG{g+gp}{\PYGZgt{}\PYGZgt{}\PYGZgt{} }\PYG{n}{test}\PYG{o}{.}\PYG{n}{set\PYGZus{}keyword}\PYG{p}{(}\PYG{l+s+s2}{\PYGZdq{}}\PYG{l+s+s2}{/Users/standfire/test/example.key}\PYG{l+s+s2}{\PYGZdq{}}\PYG{p}{)}
\end{Verbatim}

Whoops, I would like to store simulation outputs elsewhere...

\begin{Verbatim}[commandchars=\\\{\}]
\PYG{g+gp}{\PYGZgt{}\PYGZgt{}\PYGZgt{} }\PYG{n}{test}\PYG{o}{.}\PYG{n}{set\PYGZus{}dir}\PYG{p}{(}\PYG{l+s+s2}{\PYGZdq{}}\PYG{l+s+s2}{/Users/standfire/outputs/}\PYG{l+s+s2}{\PYGZdq{}}\PYG{p}{)}
\end{Verbatim}

\end{fulllineitems}

\index{set\_inv\_year() (fuels.Fvsfuels method)}

\begin{fulllineitems}
\phantomsection\label{fuels_Fvsfuels:fuels.Fvsfuels.set_inv_year}\pysiglinewithargsret{\sphinxbfcode{set\_inv\_year}}{\emph{inv\_year}}{}
Sets inventory year for FVS simulation
\begin{quote}\begin{description}
\item[{Parameters}] \leavevmode
\textbf{\texttt{inv\_year}} (\emph{\texttt{int}}) -- year of the inventory

\end{description}\end{quote}

\end{fulllineitems}

\index{set\_keyword() (fuels.Fvsfuels method)}

\begin{fulllineitems}
\phantomsection\label{fuels_Fvsfuels:fuels.Fvsfuels.set_keyword}\pysiglinewithargsret{\sphinxbfcode{set\_keyword}}{\emph{keyfile}}{}
Sets the keyword file to be used in the FVS simulation
\begin{quote}\begin{description}
\item[{Date}] \leavevmode
2015-8-12

\item[{Authors}] \leavevmode
Lucas Wells

\end{description}\end{quote}

This method will initalize a FVS simulation by registering the
specified keyword file (.key) with FVS. The working directory of a
Fvsfuels object will be set to the folder containing the keyword file.
You can manually change the working directory with Fvsfuels.set\_dir().
This function will also call private methods in this class to extract
information from the keyword file and set class fields accordingly for
use in other methods.
\begin{quote}\begin{description}
\item[{Parameters}] \leavevmode
\textbf{\texttt{keyfile}} (\emph{\texttt{string}}) -- path/to/keyword\_file. This must have a .key extension

\end{description}\end{quote}

\textbf{Example:}

\begin{Verbatim}[commandchars=\\\{\}]
\PYG{g+gp}{\PYGZgt{}\PYGZgt{}\PYGZgt{} }\PYG{k+kn}{from} \PYG{n+nn}{standfire}\PYG{n+nn}{.}\PYG{n+nn}{fuels} \PYG{k}{import} \PYG{n}{Fvsfuels}
\PYG{g+gp}{\PYGZgt{}\PYGZgt{}\PYGZgt{} }\PYG{n}{test} \PYG{o}{=} \PYG{n}{Fvsfuels}\PYG{p}{(}\PYG{l+s+s2}{\PYGZdq{}}\PYG{l+s+s2}{iec}\PYG{l+s+s2}{\PYGZdq{}}\PYG{p}{)}
\PYG{g+gp}{\PYGZgt{}\PYGZgt{}\PYGZgt{} }\PYG{n}{test}\PYG{o}{.}\PYG{n}{set\PYGZus{}keyword}\PYG{p}{(}\PYG{l+s+s2}{\PYGZdq{}}\PYG{l+s+s2}{/Users/standfire/test/example.key}\PYG{l+s+s2}{\PYGZdq{}}\PYG{p}{)}
\end{Verbatim}

\end{fulllineitems}

\index{set\_num\_cycles() (fuels.Fvsfuels method)}

\begin{fulllineitems}
\phantomsection\label{fuels_Fvsfuels:fuels.Fvsfuels.set_num_cycles}\pysiglinewithargsret{\sphinxbfcode{set\_num\_cycles}}{\emph{num\_cyc}}{}
Sets number of cycles for FVS simulation
\begin{quote}\begin{description}
\item[{Parameters}] \leavevmode
\textbf{\texttt{num\_cyc}} (\emph{\texttt{int}}) -- number of simulation cycles

\end{description}\end{quote}

\end{fulllineitems}

\index{set\_stop\_point() (fuels.Fvsfuels method)}

\begin{fulllineitems}
\phantomsection\label{fuels_Fvsfuels:fuels.Fvsfuels.set_stop_point}\pysiglinewithargsret{\sphinxbfcode{set\_stop\_point}}{\emph{code=5}, \emph{year=-1}}{}
Set the FVS stop point code and year
\begin{quote}\begin{description}
\item[{Parameters}] \leavevmode\begin{itemize}
\item {} 
\textbf{\texttt{code}} (\emph{\texttt{integer}}) -- stop point code (default=5)

\item {} 
\textbf{\texttt{year}} (\emph{\texttt{integer}}) -- stop point year (default=-1)

\end{itemize}

\end{description}\end{quote}

\begin{notice}{note}{Note:}
year=0 means never stop and year=-1 means stop every cycle
\end{notice}

\noindent\begin{tabulary}{\linewidth}{|L|L|}
\hline
\textsf{\relax 
stop point code
\unskip}\relax &\textsf{\relax 
Definition
\unskip}\relax \\
\hline
0
&
Never stop
\\
\hline
-1
&
Stop at every location
\\
\hline
1
&
Stop just before first call to Event Monitor
\\
\hline
2
&
Stop just after first call to Event Monitor
\\
\hline
3
&
Stop just before second call to Event Monitor
\\
\hline
4
&
Stop just after second call to Event Monitor
\\
\hline
5
&
Stop after growth and mort has been computed, but before applied
\\
\hline
6
&
Stop just before the ESTAB routine is called
\\
\hline\end{tabulary}


\end{fulllineitems}

\index{set\_time\_int() (fuels.Fvsfuels method)}

\begin{fulllineitems}
\phantomsection\label{fuels_Fvsfuels:fuels.Fvsfuels.set_time_int}\pysiglinewithargsret{\sphinxbfcode{set\_time\_int}}{\emph{time\_int}}{}
Sets time interval for FVS simulation
\begin{quote}\begin{description}
\item[{Parameters}] \leavevmode
\textbf{\texttt{time\_int}} (\emph{\texttt{int}}) -- length of simulation time step

\end{description}\end{quote}

\end{fulllineitems}


\end{fulllineitems}



\subsection{Inventory}
\label{fuels_Inventory::doc}\label{fuels_Inventory:inventory}\index{Inventory (class in fuels)}

\begin{fulllineitems}
\phantomsection\label{fuels_Inventory:fuels.Inventory}\pysigline{\sphinxstrong{class }\sphinxcode{fuels.}\sphinxbfcode{Inventory}}
Bases: \sphinxcode{object}

This class contains methods for converting inventory data to FVS .tre
format

This class currently does not read inventory data from an FVS access
database.  The FVS\_TreeInit database first needs to be exported as comma
delimited values. Multiple stands can be exported in the same file, the
\sphinxcode{formatFvsTreeFile()} function will format a .tre string for each stand.
All column headings must be default headings and unaltered during export.
You can view the default format by importing this class and typing \sphinxcode{FMT}.
See the FVS guide \footnote[1]{\sphinxAtStartFootnote%
Gary E. Dixon, Essential FVS: A User's Guide to the Forest
Vegetation Simulator Tech. Rep., U.S. Department of Agriculture , Forest
Service, Forest Management Service Center, Fort Collins, Colo, USA, 2003.
} for more information regarding the format of .tre
files.

\textbf{Example:}

\begin{Verbatim}[commandchars=\\\{\}]
\PYG{g+gp}{\PYGZgt{}\PYGZgt{}\PYGZgt{} }\PYG{k+kn}{from} \PYG{n+nn}{standfire} \PYG{k}{import} \PYG{n}{fuels}
\PYG{g+gp}{\PYGZgt{}\PYGZgt{}\PYGZgt{} }\PYG{n}{toDotTree} \PYG{o}{=} \PYG{n}{fuels}\PYG{o}{.}\PYG{n}{Inventory}\PYG{p}{(}\PYG{p}{)}
\PYG{g+gp}{\PYGZgt{}\PYGZgt{}\PYGZgt{} }\PYG{n}{toDotTree}\PYG{o}{.}\PYG{n}{read\PYGZus{}inventory}\PYG{p}{(}\PYG{l+s+s2}{\PYGZdq{}}\PYG{l+s+s2}{path/to/FVS\PYGZus{}TreeInit.csv}\PYG{l+s+s2}{\PYGZdq{}}\PYG{p}{)}
\PYG{g+gp}{\PYGZgt{}\PYGZgt{}\PYGZgt{} }\PYG{n}{toDotTree}\PYG{o}{.}\PYG{n}{format\PYGZus{}fvs\PYGZus{}tree\PYGZus{}file}\PYG{p}{(}\PYG{p}{)}
\PYG{g+gp}{\PYGZgt{}\PYGZgt{}\PYGZgt{} }\PYG{n}{toDotTree}\PYG{o}{.}\PYG{n}{save}\PYG{p}{(}\PYG{p}{)}
\end{Verbatim}
\begin{quote}\begin{description}
\item[{References}] \leavevmode
\end{description}\end{quote}
\index{convert\_sp\_codes() (fuels.Inventory method)}

\begin{fulllineitems}
\phantomsection\label{fuels_Inventory:fuels.Inventory.convert_sp_codes}\pysiglinewithargsret{\sphinxbfcode{convert\_sp\_codes}}{\emph{method=`2to4'}}{}
Converts species codes from 4 letter codes to 2 letter codes
or vise versa
\begin{quote}\begin{description}
\item[{Parameters}] \leavevmode
\textbf{\texttt{method}} (\emph{\texttt{string}}) -- must be either ``2to4'' or ``4to2''

\end{description}\end{quote}

\end{fulllineitems}

\index{crwratio\_percent\_to\_code() (fuels.Inventory method)}

\begin{fulllineitems}
\phantomsection\label{fuels_Inventory:fuels.Inventory.crwratio_percent_to_code}\pysiglinewithargsret{\sphinxbfcode{crwratio\_percent\_to\_code}}{}{}
Converts crown ratio from percent to ICR code

ICR code is described in the Essential FVS Guide on pages 58 and 59.
This method should only be used if crown ratios values are percentages
in the FVS\_TreeInit.csv.  If you use this method before calling
\sphinxtitleref{formatFvsTreeFile()} then you must set the optional argument
\sphinxcode{cratioToCode} to \sphinxcode{False}.

\end{fulllineitems}

\index{filter\_by\_stand() (fuels.Inventory method)}

\begin{fulllineitems}
\phantomsection\label{fuels_Inventory:fuels.Inventory.filter_by_stand}\pysiglinewithargsret{\sphinxbfcode{filter\_by\_stand}}{\emph{stand\_list}}{}
Filters data by a list of stand IDs
\begin{quote}\begin{description}
\item[{Parameters}] \leavevmode
\textbf{\texttt{stand\_list}} (\emph{\texttt{python list}}) -- List of stand ID to retain in the data. All other
stands will be removed

\end{description}\end{quote}

\end{fulllineitems}

\index{format\_fvs\_tree\_file() (fuels.Inventory method)}

\begin{fulllineitems}
\phantomsection\label{fuels_Inventory:fuels.Inventory.format_fvs_tree_file}\pysiglinewithargsret{\sphinxbfcode{format\_fvs\_tree\_file}}{\emph{cratio\_to\_code=True}}{}
Converts data in FVS\_TreeInit.csv to FVS .tre format

This methods reads entries in the pandas data frame (\sphinxcode{self.data}) and
writes them to a formated text string following FVS .tre data formating
standards shown in \sphinxcode{FMT}.  If multiple stands exist in \sphinxcode{self.data}
then each stand will written as a (key,value) pair in
\sphinxcode{self.fvsTreeFile} where the key is the stand ID and the value is the
formated text string.
\begin{quote}\begin{description}
\item[{Parameters}] \leavevmode
\textbf{\texttt{cratio\_to\_code}} (\emph{\texttt{boolean}}) -- default = True

\end{description}\end{quote}

\begin{notice}{note}{Note:}
If the \sphinxcode{crwratio\_percent\_to\_code()} methods has
been called prior to call this methods, then the \sphinxcode{cratio\_to\_code}
optional argument must be set to \sphinxcode{False} to prevent errors in crown
ratio values.
\end{notice}

\textbf{Example:}

\begin{Verbatim}[commandchars=\\\{\}]
\PYG{g+gp}{\PYGZgt{}\PYGZgt{}\PYGZgt{} }\PYG{n}{toDotTree}\PYG{o}{.}\PYG{n}{format\PYGZus{}fvs\PYGZus{}tree\PYGZus{}file}\PYG{p}{(}\PYG{p}{)}
\PYG{g+gp}{\PYGZgt{}\PYGZgt{}\PYGZgt{} }\PYG{n}{toDotTree}\PYG{o}{.}\PYG{n}{fvsTreeFile}\PYG{p}{[}\PYG{l+s+s1}{\PYGZsq{}}\PYG{l+s+s1}{Stand\PYGZus{}ID\PYGZus{}1}\PYG{l+s+s1}{\PYGZsq{}}\PYG{p}{]}
\PYG{g+go}{5   1  5     0PP 189    65        3                 0 0}
\PYG{g+go}{5   2  15    0PP 110    52        2                 0 0}
\PYG{g+go}{5   3  5     0PP 180    64        5                 0 0}
\PYG{g+go}{5   4  14    0PP 112    56        3                 0 0}
\PYG{g+go}{5   5  6     0PP 167    60        4                 0 0}
\PYG{g+go}{5   6  5     0PP 190    60        5                 0 0}
\PYG{g+go}{5   7  7     0PP 161    62        3                 0 0}
\PYG{g+go}{5   8  86    0PP 46     37        1                 0 0}
\PYG{g+go}{5   9  10    0PP 130    50        2                 0 0}
\PYG{g+go}{5   10 5     0PP 182    60        3                 0 0}
\PYG{g+go}{5   11 8     9PP 144    50                          0 0}
\PYG{g+go}{6   1  16    0PP 107    42        4                 0 0}
\PYG{g+go}{6   2  109   0PP 41     27        2                 0 0}
\PYG{g+gp}{...}
\end{Verbatim}

\end{fulllineitems}

\index{get\_fvs\_cols() (fuels.Inventory method)}

\begin{fulllineitems}
\phantomsection\label{fuels_Inventory:fuels.Inventory.get_fvs_cols}\pysiglinewithargsret{\sphinxbfcode{get\_fvs\_cols}}{}{}
Get list of FVS standard columns
\begin{quote}\begin{description}
\item[{Returns}] \leavevmode
FVS standard columns

\item[{Return type}] \leavevmode
list of strings

\end{description}\end{quote}

\end{fulllineitems}

\index{get\_stands() (fuels.Inventory method)}

\begin{fulllineitems}
\phantomsection\label{fuels_Inventory:fuels.Inventory.get_stands}\pysiglinewithargsret{\sphinxbfcode{get\_stands}}{}{}
Returns unique stand IDs
\begin{quote}\begin{description}
\item[{Returns}] \leavevmode
stand IDs

\item[{Return type}] \leavevmode
list of strings

\end{description}\end{quote}

\textbf{Example:}

\begin{Verbatim}[commandchars=\\\{\}]
\PYG{g+gp}{\PYGZgt{}\PYGZgt{}\PYGZgt{} }\PYG{n}{toDotTree}\PYG{o}{.}\PYG{n}{get\PYGZus{}stands}\PYG{p}{(}\PYG{p}{)}
\PYG{g+go}{[\PYGZsq{}BR\PYGZsq{}, \PYGZsq{}TM\PYGZsq{}, \PYGZsq{}SW\PYGZsq{}, HB\PYGZsq{}]}
\end{Verbatim}

\end{fulllineitems}

\index{print\_format\_standards() (fuels.Inventory method)}

\begin{fulllineitems}
\phantomsection\label{fuels_Inventory:fuels.Inventory.print_format_standards}\pysiglinewithargsret{\sphinxbfcode{print\_format\_standards}}{}{}
Print FVS formating standards

The FVS formating standard for .tre files as described in the Essenital
FVS Guide is stored in \sphinxcode{FMT} as a class attribute.  This method
is for viewing this format.  The keys of the dictionary are the column
headings and values are as follows: 0 = variable name, 1 = variable
type, 2 = column location, 3 = units, and 4 = implied decimal place.

\textbf{Example:}

\begin{Verbatim}[commandchars=\\\{\}]
\PYG{g+gp}{\PYGZgt{}\PYGZgt{}\PYGZgt{} }\PYG{n}{toDotTree}\PYG{o}{.}\PYG{n}{print\PYGZus{}format\PYGZus{}standards}\PYG{p}{(}\PYG{p}{)}
\PYG{g+go}{\PYGZob{}\PYGZsq{}Plot\PYGZus{}ID\PYGZsq{}       : [\PYGZsq{}ITRE\PYGZsq{},      \PYGZsq{}integer\PYGZsq{},  [0,3],   None,      None],}
\PYG{g+go}{ \PYGZsq{}Tree\PYGZus{}ID\PYGZsq{}       : [\PYGZsq{}IDTREE2\PYGZsq{},   \PYGZsq{}integer\PYGZsq{},  [4,6],   None,      None],}
\PYG{g+go}{ \PYGZsq{}Tree\PYGZus{}Count\PYGZsq{}    : [\PYGZsq{}PROB\PYGZsq{},      \PYGZsq{}integer\PYGZsq{},  [7,12],  None,      None],}
\PYG{g+go}{ \PYGZsq{}History\PYGZsq{}       : [\PYGZsq{}ITH\PYGZsq{},       \PYGZsq{}integer\PYGZsq{},  [13,13], \PYGZsq{}trees\PYGZsq{},   0   ],}
\PYG{g+go}{ \PYGZsq{}Species\PYGZsq{}       : [\PYGZsq{}ISP\PYGZsq{},       \PYGZsq{}alphanum\PYGZsq{}, [14,16], None,      None],}
\PYG{g+go}{ \PYGZsq{}DBH\PYGZsq{}           : [\PYGZsq{}DBH\PYGZsq{},       \PYGZsq{}real\PYGZsq{},     [17,20], \PYGZsq{}inches\PYGZsq{},  1   ],}
\PYG{g+go}{ \PYGZsq{}DG\PYGZsq{}            : [\PYGZsq{}DG\PYGZsq{},        \PYGZsq{}real\PYGZsq{},     [21,23], \PYGZsq{}inches\PYGZsq{},  1   ],}
\PYG{g+go}{ \PYGZsq{}Ht\PYGZsq{}            : [\PYGZsq{}HT\PYGZsq{},        \PYGZsq{}real\PYGZsq{},     [24,26], \PYGZsq{}feet\PYGZsq{},    0   ],}
\PYG{g+go}{ \PYGZsq{}HtTopK\PYGZsq{}        : [\PYGZsq{}THT\PYGZsq{},       \PYGZsq{}real\PYGZsq{},     [27,29], \PYGZsq{}feet\PYGZsq{},    0   ],}
\PYG{g+go}{ \PYGZsq{}HTG\PYGZsq{}           : [\PYGZsq{}HTG\PYGZsq{},       \PYGZsq{}real\PYGZsq{},     [30,33], \PYGZsq{}feet\PYGZsq{},    1   ],}
\PYG{g+go}{ \PYGZsq{}CrRatio\PYGZsq{}       : [\PYGZsq{}ICR\PYGZsq{},       \PYGZsq{}integer\PYGZsq{},  [34,34], None,      None],}
\PYG{g+go}{ \PYGZsq{}Damage1\PYGZsq{}       : [\PYGZsq{}IDCD(1)\PYGZsq{},   \PYGZsq{}integer\PYGZsq{},  [35,36], None,      None],}
\PYG{g+go}{ \PYGZsq{}Severity1\PYGZsq{}     : [\PYGZsq{}IDCD(2)\PYGZsq{},   \PYGZsq{}integer\PYGZsq{},  [37,38], None,      None],}
\PYG{g+go}{ \PYGZsq{}Damage2\PYGZsq{}       : [\PYGZsq{}IDCD(3)\PYGZsq{},   \PYGZsq{}integer\PYGZsq{},  [39,40], None,      None],}
\PYG{g+go}{ \PYGZsq{}Severity2\PYGZsq{}     : [\PYGZsq{}IDCD(4)\PYGZsq{},   \PYGZsq{}integer\PYGZsq{},  [41,42], None,      None],}
\PYG{g+go}{ \PYGZsq{}Damage3\PYGZsq{}       : [\PYGZsq{}IDCD(5)\PYGZsq{},   \PYGZsq{}integer\PYGZsq{},  [43,44], None,      None],}
\PYG{g+go}{ \PYGZsq{}Severity3\PYGZsq{}     : [\PYGZsq{}IDCD(6)\PYGZsq{},   \PYGZsq{}integer\PYGZsq{},  [45,46], None,      None],}
\PYG{g+go}{ \PYGZsq{}TreeValue\PYGZsq{}     : [\PYGZsq{}IMC\PYGZsq{},       \PYGZsq{}integer\PYGZsq{},  [47,47], None,      None],}
\PYG{g+go}{ \PYGZsq{}Prescription\PYGZsq{}  : [\PYGZsq{}IPRSC\PYGZsq{},     \PYGZsq{}integer\PYGZsq{},  [48,48], None,      None],}
\PYG{g+go}{ \PYGZsq{}Slope\PYGZsq{}         : [\PYGZsq{}IPVARS(1)\PYGZsq{}, \PYGZsq{}integer\PYGZsq{},  [49,50], \PYGZsq{}percent\PYGZsq{}, None],}
\PYG{g+go}{ \PYGZsq{}Aspect\PYGZsq{}        : [\PYGZsq{}IPVARS(2)\PYGZsq{}, \PYGZsq{}integer\PYGZsq{},  [51,53], \PYGZsq{}code\PYGZsq{},    None],}
\PYG{g+go}{ \PYGZsq{}PV\PYGZus{}Code\PYGZsq{}       : [\PYGZsq{}IPVARS(3)\PYGZsq{}, \PYGZsq{}integer\PYGZsq{},  [54,56], \PYGZsq{}code\PYGZsq{},    None],}
\PYG{g+go}{ \PYGZsq{}TopoCode\PYGZsq{}      : [\PYGZsq{}IPVARS(4)\PYGZsq{}, \PYGZsq{}integer\PYGZsq{},  [57,59], \PYGZsq{}code\PYGZsq{},    None],}
\PYG{g+go}{ \PYGZsq{}SitePrep\PYGZsq{}      : [\PYGZsq{}IPVARS(5)\PYGZsq{}, \PYGZsq{}integer\PYGZsq{},  [58,58], \PYGZsq{}code\PYGZsq{},    None],}
\PYG{g+go}{ \PYGZsq{}Age\PYGZsq{}           : [\PYGZsq{}ABIRTH\PYGZsq{},    \PYGZsq{}real\PYGZsq{},     [59,61], \PYGZsq{}years\PYGZsq{},   0   ]\PYGZcb{}}
\end{Verbatim}

See page 61 and 62 in the Essential FVS Guide.

\end{fulllineitems}

\index{read\_inventory() (fuels.Inventory method)}

\begin{fulllineitems}
\phantomsection\label{fuels_Inventory:fuels.Inventory.read_inventory}\pysiglinewithargsret{\sphinxbfcode{read\_inventory}}{\emph{fname}}{}
Reads a .csv file containing tree records.

The csv must be in the correct format as described in \sphinxcode{FMT}.  This
method check the format of the file by calling a private method
\sphinxcode{\_is\_correct\_format()} that raises a value error.
\begin{quote}\begin{description}
\item[{Parameters}] \leavevmode
\textbf{\texttt{fname}} (\emph{\texttt{string}}) -- path to and file name of the Fvs\_TreeInit.csv file

\end{description}\end{quote}

\textbf{Example:}

\begin{Verbatim}[commandchars=\\\{\}]
\PYG{g+gp}{\PYGZgt{}\PYGZgt{}\PYGZgt{} }\PYG{k+kn}{from} \PYG{n+nn}{standfire} \PYG{k}{import} \PYG{n}{fuels}
\PYG{g+gp}{\PYGZgt{}\PYGZgt{}\PYGZgt{} }\PYG{n}{toDotTree} \PYG{o}{=} \PYG{n}{fuels}\PYG{o}{.}\PYG{n}{Inventory}\PYG{p}{(}\PYG{p}{)}
\PYG{g+gp}{\PYGZgt{}\PYGZgt{}\PYGZgt{} }\PYG{n}{toDotTree}\PYG{o}{.}\PYG{n}{readInventory}\PYG{p}{(}\PYG{l+s+s2}{\PYGZdq{}}\PYG{l+s+s2}{path/to/FVS\PYGZus{}TreeInit.csv}\PYG{l+s+s2}{\PYGZdq{}}\PYG{p}{)}
\PYG{g+gp}{\PYGZgt{}\PYGZgt{}\PYGZgt{} }\PYG{n}{np}\PYG{o}{.}\PYG{n}{mean}\PYG{p}{(}\PYG{n}{toDotTree}\PYG{o}{.}\PYG{n}{data}\PYG{p}{[}\PYG{l+s+s1}{\PYGZsq{}}\PYG{l+s+s1}{DBH}\PYG{l+s+s1}{\PYGZsq{}}\PYG{p}{]}\PYG{p}{)}
\PYG{g+go}{9.0028318584070828}
\end{Verbatim}

The \sphinxcode{read\_inventory()} method stores the data in a pandas data frame.
There are countless operations that can be performed on these objects.
For example, we can explore the relationship between diameter and
height by fitting a linear model

\begin{Verbatim}[commandchars=\\\{\}]
\PYG{g+gp}{\PYGZgt{}\PYGZgt{}\PYGZgt{} }\PYG{k+kn}{import} \PYG{n+nn}{statsmodels}\PYG{n+nn}{.}\PYG{n+nn}{formula}\PYG{n+nn}{.}\PYG{n+nn}{api} \PYG{k}{as} \PYG{n+nn}{sm}
\PYG{g+gp}{\PYGZgt{}\PYGZgt{}\PYGZgt{} }\PYG{n}{fit} \PYG{o}{=} \PYG{n}{sm}\PYG{o}{.}\PYG{n}{ols}\PYG{p}{(}\PYG{n}{formula}\PYG{o}{=}\PYG{l+s+s2}{\PYGZdq{}}\PYG{l+s+s2}{HT \PYGZti{} DBH}\PYG{l+s+s2}{\PYGZdq{}}\PYG{p}{,} \PYG{n}{data}\PYG{o}{=}\PYG{n}{test}\PYG{o}{.}\PYG{n}{data}\PYG{p}{)}\PYG{o}{.}\PYG{n}{fit}\PYG{p}{(}\PYG{p}{)}
\PYG{g+gp}{\PYGZgt{}\PYGZgt{}\PYGZgt{} }\PYG{n+nb}{print} \PYG{n}{fit}\PYG{o}{.}\PYG{n}{params}
\PYG{g+go}{Intercept    19.688167}
\PYG{g+go}{DBH           2.161420}
\PYG{g+go}{dtype: float64}
\PYG{g+gp}{\PYGZgt{}\PYGZgt{}\PYGZgt{} }\PYG{n+nb}{print} \PYG{n}{fit}\PYG{o}{.}\PYG{n}{summary}\PYG{p}{(}\PYG{p}{)}
\PYG{g+go}{OLS Regression Results}
\PYG{g+go}{==============================================================================}
\PYG{g+go}{Dep. Variable:                     Ht   R\PYGZhy{}squared:                       0.738}
\PYG{g+go}{Model:                            OLS   Adj. R\PYGZhy{}squared:                  0.736}
\PYG{g+go}{Method:                 Least Squares   F\PYGZhy{}statistic:                     351.8}
\PYG{g+go}{Date:                Tue, 07 Jul 2015   Prob (F\PYGZhy{}statistic):           3.77e\PYGZhy{}38}
\PYG{g+go}{Time:                        08:32:02   Log\PYGZhy{}Likelihood:                \PYGZhy{}407.10}
\PYG{g+go}{No. Observations:                 127   AIC:                             818.2}
\PYG{g+go}{Df Residuals:                     125   BIC:                             823.9}
\PYG{g+go}{Df Model:                           1}
\PYG{g+go}{Covariance Type:            nonrobust}
\PYG{g+go}{==============================================================================}
\PYG{g+go}{coef    std err          t      P\PYGZgt{}\textbar{}t\textbar{}      [95.0\PYGZpc{} Conf. Int.]}
\PYG{g+go}{\PYGZhy{}\PYGZhy{}\PYGZhy{}\PYGZhy{}\PYGZhy{}\PYGZhy{}\PYGZhy{}\PYGZhy{}\PYGZhy{}\PYGZhy{}\PYGZhy{}\PYGZhy{}\PYGZhy{}\PYGZhy{}\PYGZhy{}\PYGZhy{}\PYGZhy{}\PYGZhy{}\PYGZhy{}\PYGZhy{}\PYGZhy{}\PYGZhy{}\PYGZhy{}\PYGZhy{}\PYGZhy{}\PYGZhy{}\PYGZhy{}\PYGZhy{}\PYGZhy{}\PYGZhy{}\PYGZhy{}\PYGZhy{}\PYGZhy{}\PYGZhy{}\PYGZhy{}\PYGZhy{}\PYGZhy{}\PYGZhy{}\PYGZhy{}\PYGZhy{}\PYGZhy{}\PYGZhy{}\PYGZhy{}\PYGZhy{}\PYGZhy{}\PYGZhy{}\PYGZhy{}\PYGZhy{}\PYGZhy{}\PYGZhy{}\PYGZhy{}\PYGZhy{}\PYGZhy{}\PYGZhy{}\PYGZhy{}\PYGZhy{}\PYGZhy{}\PYGZhy{}\PYGZhy{}\PYGZhy{}\PYGZhy{}\PYGZhy{}\PYGZhy{}\PYGZhy{}\PYGZhy{}\PYGZhy{}\PYGZhy{}\PYGZhy{}\PYGZhy{}\PYGZhy{}\PYGZhy{}\PYGZhy{}\PYGZhy{}\PYGZhy{}\PYGZhy{}\PYGZhy{}\PYGZhy{}\PYGZhy{}}
\PYG{g+go}{Intercept     19.6882      1.205     16.338      0.000        17.303    22.073}
\PYG{g+go}{DBH            2.1614      0.115     18.757      0.000         1.933     2.389}
\PYG{g+go}{==============================================================================}
\PYG{g+go}{Omnibus:                        2.658   Durbin\PYGZhy{}Watson:                   0.995}
\PYG{g+go}{Prob(Omnibus):                  0.265   Jarque\PYGZhy{}Bera (JB):                2.115}
\PYG{g+go}{Skew:                          \PYGZhy{}0.251   Prob(JB):                        0.347}
\PYG{g+go}{Kurtosis:                       3.385   Cond. No.                         23.8}
\PYG{g+go}{==============================================================================}
\end{Verbatim}

Read more about pandas at \url{http://pandas.pydata.org/}

\end{fulllineitems}

\index{save() (fuels.Inventory method)}

\begin{fulllineitems}
\phantomsection\label{fuels_Inventory:fuels.Inventory.save}\pysiglinewithargsret{\sphinxbfcode{save}}{\emph{outputPath}}{}
Writes formated fvs tree files to specified location

If multiple stands exist in the FVS\_TreeInit then the same number of
files will be created in the specified directory.  The file names
will be the same as the Stand\_ID with a \sphinxcode{.tre} extension.
\begin{quote}\begin{description}
\item[{Parameters}] \leavevmode
\textbf{\texttt{outputPath}} (\emph{\texttt{string}}) -- directory to store output .tre files

\end{description}\end{quote}

\begin{notice}{note}{Note:}
This method will throw an error if it is called prior to the
\sphinxcode{format\_fvs\_tree\_file()} method.
\end{notice}

\end{fulllineitems}

\index{set\_FVS\_variant() (fuels.Inventory method)}

\begin{fulllineitems}
\phantomsection\label{fuels_Inventory:fuels.Inventory.set_FVS_variant}\pysiglinewithargsret{\sphinxbfcode{set\_FVS\_variant}}{\emph{var}}{}
Sets FVS variant. This is need if converting from USDA plant symbols
(PSME) to alpha codes (DF). If so, then all stand you wish to process
must be of the same FVS variant
\begin{quote}\begin{description}
\item[{Parameters}] \leavevmode
\textbf{\texttt{var}} (\emph{\texttt{string}}) -- FVS variant (`iec', `emc' ...)

\end{description}\end{quote}

\end{fulllineitems}


\end{fulllineitems}



\subsection{FuelCalc}
\label{fuels_FuelCalc:fuelcalc}\label{fuels_FuelCalc::doc}\index{FuelCalc (class in fuels)}

\begin{fulllineitems}
\phantomsection\label{fuels_FuelCalc:fuels.FuelCalc}\pysiglinewithargsret{\sphinxstrong{class }\sphinxcode{fuels.}\sphinxbfcode{FuelCalc}}{\emph{trees}}{}
Bases: \sphinxcode{object}

This class implements various fuel calculation based on the FVS output.
\begin{quote}\begin{description}
\item[{Parameters}] \leavevmode
\textbf{\texttt{trees}} (\emph{\texttt{comma-delimited file or pandas data frame of tree list}}) -- FVS output tree list

\end{description}\end{quote}
\index{calc\_bulk\_density() (fuels.FuelCalc method)}

\begin{fulllineitems}
\phantomsection\label{fuels_FuelCalc:fuels.FuelCalc.calc_bulk_density}\pysiglinewithargsret{\sphinxbfcode{calc\_bulk\_density}}{}{}
Calculates crown bulk density based on crown volume and biomass weight

\end{fulllineitems}

\index{calc\_crown\_volume() (fuels.FuelCalc method)}

\begin{fulllineitems}
\phantomsection\label{fuels_FuelCalc:fuels.FuelCalc.calc_crown_volume}\pysiglinewithargsret{\sphinxbfcode{calc\_crown\_volume}}{}{}
Calculates crown volume based on geometry and crown dimensions

\end{fulllineitems}

\index{cone\_volume() (fuels.FuelCalc method)}

\begin{fulllineitems}
\phantomsection\label{fuels_FuelCalc:fuels.FuelCalc.cone_volume}\pysiglinewithargsret{\sphinxbfcode{cone\_volume}}{\emph{r}, \emph{h}}{}~\begin{quote}

Returns the volume of a cone
\begin{quote}\begin{description}
\item[{param r}] \leavevmode
radius

\item[{type r}] \leavevmode
float

\item[{param h}] \leavevmode
height

\item[{type h}] \leavevmode
float

\item[{return}] \leavevmode
volume

\item[{rtype}] \leavevmode
float

\end{description}\end{quote}
\begin{equation*}
\begin{split}\pi r^2 \end{split}
\end{equation*}\end{quote}

rac\{h\}\{3\}

\end{fulllineitems}

\index{convert\_units() (fuels.FuelCalc method)}

\begin{fulllineitems}
\phantomsection\label{fuels_FuelCalc:fuels.FuelCalc.convert_units}\pysiglinewithargsret{\sphinxbfcode{convert\_units}}{\emph{from\_to=1}}{}
Convert all units in data frame
\begin{quote}\begin{description}
\item[{Parameters}] \leavevmode
\textbf{\texttt{from\_to}} (\emph{\texttt{integer}}) -- 1 = english to metric; 2 = metric to english (default=1)

\end{description}\end{quote}

\begin{notice}{note}{Note:}
if this method is called more than once on the same instance
of a data frame with the same conversion code a warning will
be printed to the console
\end{notice}

\end{fulllineitems}

\index{cylinder\_volume() (fuels.FuelCalc method)}

\begin{fulllineitems}
\phantomsection\label{fuels_FuelCalc:fuels.FuelCalc.cylinder_volume}\pysiglinewithargsret{\sphinxbfcode{cylinder\_volume}}{\emph{r}, \emph{h}}{}
Returns the volume of a cylinder
\begin{quote}\begin{description}
\item[{Parameters}] \leavevmode\begin{itemize}
\item {} 
\textbf{\texttt{r}} (\emph{\texttt{float}}) -- radius

\item {} 
\textbf{\texttt{h}} (\emph{\texttt{float}}) -- height

\end{itemize}

\item[{Returns}] \leavevmode
volume

\item[{Return type}] \leavevmode
float

\end{description}\end{quote}
\begin{equation*}
\begin{split}\pi r^2 h\end{split}
\end{equation*}
\end{fulllineitems}

\index{frustum\_volume() (fuels.FuelCalc method)}

\begin{fulllineitems}
\phantomsection\label{fuels_FuelCalc:fuels.FuelCalc.frustum_volume}\pysiglinewithargsret{\sphinxbfcode{frustum\_volume}}{\emph{R}, \emph{h}, \emph{r=0.5}}{}~\begin{quote}

Returns the volume of a frustum
\begin{quote}\begin{description}
\item[{param r}] \leavevmode
small (top) radius

\item[{type r}] \leavevmode
float

\item[{param h}] \leavevmode
height

\item[{type h}] \leavevmode
float

\item[{param R}] \leavevmode
big (bottom) radius

\item[{type R}] \leavevmode
float

\item[{return}] \leavevmode
volume

\item[{rtype}] \leavevmode
float

\end{description}\end{quote}
\end{quote}

rac\{pi h\}\{3\}(R\textasciicircum{}2+rR+r\textasciicircum{}2)

\end{fulllineitems}

\index{get\_crown\_base\_ht() (fuels.FuelCalc method)}

\begin{fulllineitems}
\phantomsection\label{fuels_FuelCalc:fuels.FuelCalc.get_crown_base_ht}\pysiglinewithargsret{\sphinxbfcode{get\_crown\_base\_ht}}{}{}
Calculates crown base height for each tree based on crown ratio and
tree height. This value is added to the data frame
\begin{equation*}
\begin{split}h - (c_{ratio} h)\end{split}
\end{equation*}
\end{fulllineitems}

\index{get\_crown\_ht() (fuels.FuelCalc method)}

\begin{fulllineitems}
\phantomsection\label{fuels_FuelCalc:fuels.FuelCalc.get_crown_ht}\pysiglinewithargsret{\sphinxbfcode{get\_crown\_ht}}{}{}
Calculates crown height for each trees based on crown ratio. This
value is added to the data frame
\begin{equation*}
\begin{split}c_{ratio}h\end{split}
\end{equation*}
\end{fulllineitems}

\index{get\_species\_list() (fuels.FuelCalc method)}

\begin{fulllineitems}
\phantomsection\label{fuels_FuelCalc:fuels.FuelCalc.get_species_list}\pysiglinewithargsret{\sphinxbfcode{get\_species\_list}}{}{}
Return set of species existing in trees file supplied to constructor
\begin{quote}\begin{description}
\item[{Returns}] \leavevmode
unique list of species

\item[{Return type}] \leavevmode
list

\end{description}\end{quote}

\begin{notice}{note}{Note:}
This methods is useful when assigning geometries by species.
A user can first retrieve the species list then use it to
assigne crown geometries
\end{notice}

\end{fulllineitems}

\index{rectangle\_volume() (fuels.FuelCalc method)}

\begin{fulllineitems}
\phantomsection\label{fuels_FuelCalc:fuels.FuelCalc.rectangle_volume}\pysiglinewithargsret{\sphinxbfcode{rectangle\_volume}}{\emph{w}, \emph{h}}{}
Returns the volume of a rectangle
\begin{quote}\begin{description}
\item[{Parameters}] \leavevmode\begin{itemize}
\item {} 
\textbf{\texttt{w}} (\emph{\texttt{float}}) -- width

\item {} 
\textbf{\texttt{h}} (\emph{\texttt{float}}) -- height

\end{itemize}

\item[{Returns}] \leavevmode
volume

\item[{Return type}] \leavevmode
float

\end{description}\end{quote}
\begin{equation*}
\begin{split}wwh\end{split}
\end{equation*}
\end{fulllineitems}

\index{save\_trees() (fuels.FuelCalc method)}

\begin{fulllineitems}
\phantomsection\label{fuels_FuelCalc:fuels.FuelCalc.save_trees}\pysiglinewithargsret{\sphinxbfcode{save\_trees}}{\emph{save\_to}}{}
Write trees data frame to specified directory
\begin{quote}\begin{description}
\item[{Parameters}] \leavevmode
\textbf{\texttt{save\_to}} (\emph{\texttt{string}}) -- directory and filename of file to save

\end{description}\end{quote}

\end{fulllineitems}

\index{set\_crown\_geometry() (fuels.FuelCalc method)}

\begin{fulllineitems}
\phantomsection\label{fuels_FuelCalc:fuels.FuelCalc.set_crown_geometry}\pysiglinewithargsret{\sphinxbfcode{set\_crown\_geometry}}{\emph{sp\_geom\_dict}}{}
Appends crown geometry to each tree in the data frame conditional on
species.
\begin{quote}\begin{description}
\item[{Parameters}] \leavevmode
\textbf{\texttt{sp\_geom\_dict}} (\emph{\texttt{python dictionary}}) -- dictionary of species specific crown geometries

\item[{Example}] \leavevmode
\begin{Verbatim}[commandchars=\\\{\}]
\PYG{g+gp}{\PYGZgt{}\PYGZgt{}\PYGZgt{} }\PYG{n}{sp\PYGZus{}dict} \PYG{o}{=} \PYG{p}{\PYGZob{}}\PYG{l+s+s1}{\PYGZsq{}}\PYG{l+s+s1}{PIPO}\PYG{l+s+s1}{\PYGZsq{}} \PYG{p}{:} \PYG{l+s+s1}{\PYGZsq{}}\PYG{l+s+s1}{cylinder}\PYG{l+s+s1}{\PYGZsq{}}\PYG{p}{,} \PYG{l+s+s1}{\PYGZsq{}}\PYG{l+s+s1}{PSME}\PYG{l+s+s1}{\PYGZsq{}} \PYG{p}{:} \PYG{l+s+s1}{\PYGZsq{}}\PYG{l+s+s1}{frustum}\PYG{l+s+s1}{\PYGZsq{}}\PYG{p}{\PYGZcb{}}
\PYG{g+gp}{\PYGZgt{}\PYGZgt{}\PYGZgt{} }\PYG{n}{fuels}\PYG{o}{.}\PYG{n}{set\PYGZus{}crown\PYGZus{}geometry}\PYG{p}{(}\PYG{n}{sp\PYGZus{}dict}\PYG{p}{)}
\end{Verbatim}

\end{description}\end{quote}

\end{fulllineitems}


\end{fulllineitems}



\section{Capsis module}
\label{capsis:capsis-module}\label{capsis::doc}\label{capsis:module-capsis}\index{capsis (module)}
This module is a Python wrapper for the Capsis Standfire suite. Currently
Capsis' role in Standfire is to distribute the fuels present in
the files generated by Fvsfuels. Capsis uses a pre-generated FDS grid file (.xyz)
to place canopy and surface in a user defined domain. Capsis provides many options
for placing these fuels. The pertenant arguments can be controlled through 
the Capsis RunConig class. The Execute class is used to run Capsis. An up-to-date
Java installation is required since Capsis run on a Java Virtual Machine.

Contents:


\subsection{RunConfig}
\label{capsis_RunConfig:runconfig}\label{capsis_RunConfig::doc}\index{RunConfig (class in capsis)}

\begin{fulllineitems}
\phantomsection\label{capsis_RunConfig:capsis.RunConfig}\pysiglinewithargsret{\sphinxstrong{class }\sphinxcode{capsis.}\sphinxbfcode{RunConfig}}{\emph{run\_directory}}{}
Bases: \sphinxcode{object}

The Capsis RunConfig class is used to configure a capsis run.
\begin{quote}\begin{description}
\item[{Parameters}] \leavevmode
\textbf{\texttt{run\_directory}} -- desired path for Capsis run

\item[{Run\_directory type}] \leavevmode
string

\end{description}\end{quote}

{\color{red}\bfseries{}*}Example:\#

\begin{Verbatim}[commandchars=\\\{\}]
\PYG{g+gp}{\PYGZgt{}\PYGZgt{}\PYGZgt{} }\PYG{k+kn}{import} \PYG{n+nn}{capsis}
\PYG{g+gp}{\PYGZgt{}\PYGZgt{}\PYGZgt{} }\PYG{n}{config} \PYG{o}{=} \PYG{n}{capsis}\PYG{o}{.}\PYG{n}{RunConfig}\PYG{p}{(}\PYG{l+s+s1}{\PYGZsq{}}\PYG{l+s+s1}{/path/to/capsis\PYGZus{}run/}\PYG{l+s+s1}{\PYGZsq{}}\PYG{p}{)}
\PYG{g+gp}{\PYGZgt{}\PYGZgt{}\PYGZgt{} }\PYG{n}{config}\PYG{o}{.}\PYG{n}{set\PYGZus{}x\PYGZus{}size}\PYG{p}{(}\PYG{l+m+mi}{200}\PYG{p}{)}
\PYG{g+gp}{\PYGZgt{}\PYGZgt{}\PYGZgt{} }\PYG{n}{config}\PYG{o}{.}\PYG{n}{set\PYGZus{}y\PYGZus{}size}\PYG{p}{(}\PYG{l+m+mi}{145}\PYG{p}{)}
\PYG{g+gp}{\PYGZgt{}\PYGZgt{}\PYGZgt{} }\PYG{n}{config}\PYG{o}{.}\PYG{n}{set\PYGZus{}svs\PYGZus{}base}\PYG{p}{(}\PYG{l+s+s1}{\PYGZsq{}}\PYG{l+s+s1}{stand\PYGZus{}0001}\PYG{l+s+s1}{\PYGZsq{}}\PYG{p}{)}
\PYG{g+gp}{\PYGZgt{}\PYGZgt{}\PYGZgt{} }\PYG{n}{config}\PYG{o}{.}\PYG{n}{set\PYGZus{}crown\PYGZus{}space}\PYG{p}{(}\PYG{l+m+mf}{1.5}\PYG{p}{)}
\PYG{g+gp}{\PYGZgt{}\PYGZgt{}\PYGZgt{} }\PYG{n}{config}\PYG{o}{.}\PYG{n}{set\PYGZus{}show3d}\PYG{p}{(}\PYG{l+s+s1}{\PYGZsq{}}\PYG{l+s+s1}{true}\PYG{l+s+s1}{\PYGZsq{}}\PYG{p}{)}
\PYG{g+gp}{\PYGZgt{}\PYGZgt{}\PYGZgt{} }\PYG{n}{config}\PYG{o}{.}\PYG{n}{save\PYGZus{}config}\PYG{p}{(}\PYG{p}{)}
\end{Verbatim}
\index{save\_config() (capsis.RunConfig method)}

\begin{fulllineitems}
\phantomsection\label{capsis_RunConfig:capsis.RunConfig.save_config}\pysiglinewithargsret{\sphinxbfcode{save\_config}}{}{}
This method uses the defined parameters to generate Capsis input files

\end{fulllineitems}

\index{set\_crown\_space() (capsis.RunConfig method)}

\begin{fulllineitems}
\phantomsection\label{capsis_RunConfig:capsis.RunConfig.set_crown_space}\pysiglinewithargsret{\sphinxbfcode{set\_crown\_space}}{\emph{space}}{}
Set the distance between crown for Capsis intervention
\begin{quote}\begin{description}
\item[{Parameters}] \leavevmode
\textbf{\texttt{space}} -- crown spacing in meters

\item[{Space type}] \leavevmode
float

\end{description}\end{quote}

\end{fulllineitems}

\index{set\_path() (capsis.RunConfig method)}

\begin{fulllineitems}
\phantomsection\label{capsis_RunConfig:capsis.RunConfig.set_path}\pysiglinewithargsret{\sphinxbfcode{set\_path}}{\emph{path}}{}
Sets path to Capsis run directory
\begin{quote}\begin{description}
\item[{Parameters}] \leavevmode
\textbf{\texttt{path}} (\emph{\texttt{string}}) -- path to Capsis run directory

\end{description}\end{quote}

\end{fulllineitems}

\index{set\_prune\_height() (capsis.RunConfig method)}

\begin{fulllineitems}
\phantomsection\label{capsis_RunConfig:capsis.RunConfig.set_prune_height}\pysiglinewithargsret{\sphinxbfcode{set\_prune\_height}}{\emph{prune}}{}
Set the prunning height (vertical spacing between ground and crown)
for a Capsis intervention
\begin{quote}\begin{description}
\item[{Parameters}] \leavevmode
\textbf{\texttt{prune}} -- prunning height

\item[{Prune type}] \leavevmode
float

\end{description}\end{quote}

\end{fulllineitems}

\index{set\_show3D() (capsis.RunConfig method)}

\begin{fulllineitems}
\phantomsection\label{capsis_RunConfig:capsis.RunConfig.set_show3D}\pysiglinewithargsret{\sphinxbfcode{set\_show3D}}{\emph{value}}{}
Set the boolean value of the show3D parameter in the Capsis run file. If
true, Capsis will open a 3D displaying the simulation domain.
\begin{quote}\begin{description}
\item[{Parameters}] \leavevmode
\textbf{\texttt{value}} -- Truth value of the show3D parameter

\item[{Value type}] \leavevmode
boolean

\end{description}\end{quote}

\end{fulllineitems}

\index{set\_srf\_cbh() (capsis.RunConfig method)}

\begin{fulllineitems}
\phantomsection\label{capsis_RunConfig:capsis.RunConfig.set_srf_cbh}\pysiglinewithargsret{\sphinxbfcode{set\_srf\_cbh}}{\emph{shrub\_cbh}, \emph{herb\_cbh}}{}
\end{fulllineitems}

\index{set\_srf\_cover() (capsis.RunConfig method)}

\begin{fulllineitems}
\phantomsection\label{capsis_RunConfig:capsis.RunConfig.set_srf_cover}\pysiglinewithargsret{\sphinxbfcode{set\_srf\_cover}}{\emph{shrub\_cover}, \emph{herb\_cover}}{}
\end{fulllineitems}

\index{set\_srf\_dead\_load() (capsis.RunConfig method)}

\begin{fulllineitems}
\phantomsection\label{capsis_RunConfig:capsis.RunConfig.set_srf_dead_load}\pysiglinewithargsret{\sphinxbfcode{set\_srf\_dead\_load}}{\emph{shrub\_load}, \emph{herb\_load}, \emph{litter\_load}}{}
\end{fulllineitems}

\index{set\_srf\_dead\_mc() (capsis.RunConfig method)}

\begin{fulllineitems}
\phantomsection\label{capsis_RunConfig:capsis.RunConfig.set_srf_dead_mc}\pysiglinewithargsret{\sphinxbfcode{set\_srf\_dead\_mc}}{\emph{shrub\_mc}, \emph{herb\_mc}, \emph{litter\_mc}}{}
\end{fulllineitems}

\index{set\_srf\_dead\_svr() (capsis.RunConfig method)}

\begin{fulllineitems}
\phantomsection\label{capsis_RunConfig:capsis.RunConfig.set_srf_dead_svr}\pysiglinewithargsret{\sphinxbfcode{set\_srf\_dead\_svr}}{\emph{shrub\_svr}, \emph{herb\_svr}, \emph{litter\_svr}}{}
\end{fulllineitems}

\index{set\_srf\_height() (capsis.RunConfig method)}

\begin{fulllineitems}
\phantomsection\label{capsis_RunConfig:capsis.RunConfig.set_srf_height}\pysiglinewithargsret{\sphinxbfcode{set\_srf\_height}}{\emph{shrub\_ht}, \emph{herb\_ht}, \emph{litter\_ht}}{}
\end{fulllineitems}

\index{set\_srf\_live\_load() (capsis.RunConfig method)}

\begin{fulllineitems}
\phantomsection\label{capsis_RunConfig:capsis.RunConfig.set_srf_live_load}\pysiglinewithargsret{\sphinxbfcode{set\_srf\_live\_load}}{\emph{shrub\_load}, \emph{herb\_load}}{}
\end{fulllineitems}

\index{set\_srf\_live\_mc() (capsis.RunConfig method)}

\begin{fulllineitems}
\phantomsection\label{capsis_RunConfig:capsis.RunConfig.set_srf_live_mc}\pysiglinewithargsret{\sphinxbfcode{set\_srf\_live\_mc}}{\emph{shrub\_mc}, \emph{herb\_mc}}{}
\end{fulllineitems}

\index{set\_srf\_live\_svr() (capsis.RunConfig method)}

\begin{fulllineitems}
\phantomsection\label{capsis_RunConfig:capsis.RunConfig.set_srf_live_svr}\pysiglinewithargsret{\sphinxbfcode{set\_srf\_live\_svr}}{\emph{shrub\_svr}, \emph{herb\_svr}}{}
\end{fulllineitems}

\index{set\_svs\_base() (capsis.RunConfig method)}

\begin{fulllineitems}
\phantomsection\label{capsis_RunConfig:capsis.RunConfig.set_svs_base}\pysiglinewithargsret{\sphinxbfcode{set\_svs\_base}}{\emph{base\_name}}{}
Sets the base file name for FVS/SVS fuel output files
\begin{quote}\begin{description}
\item[{Parameters}] \leavevmode
\textbf{\texttt{base\_name}} (\emph{\texttt{string}}) -- base file name for fuel output files

\end{description}\end{quote}

\begin{notice}{note}{Note:}
Only the tree.csv file is required. If snags, cwd and scalar
files exist in the same directory they will be used by
Capsis when writing WFDS fuel inputs.
\end{notice}

\end{fulllineitems}

\index{set\_x\_offset() (capsis.RunConfig method)}

\begin{fulllineitems}
\phantomsection\label{capsis_RunConfig:capsis.RunConfig.set_x_offset}\pysiglinewithargsret{\sphinxbfcode{set\_x\_offset}}{\emph{offset}}{}
Set the x offset of the area of analysis
\begin{quote}\begin{description}
\item[{Parameters}] \leavevmode
\textbf{\texttt{offset}} -- x offset of the AOA

\item[{Offset type}] \leavevmode
integer

\end{description}\end{quote}

\end{fulllineitems}

\index{set\_x\_size() (capsis.RunConfig method)}

\begin{fulllineitems}
\phantomsection\label{capsis_RunConfig:capsis.RunConfig.set_x_size}\pysiglinewithargsret{\sphinxbfcode{set\_x\_size}}{\emph{x\_size}}{}
Sets scene x dimension
\begin{quote}\begin{description}
\item[{Parameters}] \leavevmode
\textbf{\texttt{x\_size}} (\emph{\texttt{integer}}) -- size of scene in the x domain (meters)

\end{description}\end{quote}

\begin{notice}{note}{Note:}
\sphinxtitleref{x\_size} must be greater than or equal to 64 meters
\end{notice}

\end{fulllineitems}

\index{set\_y\_offset() (capsis.RunConfig method)}

\begin{fulllineitems}
\phantomsection\label{capsis_RunConfig:capsis.RunConfig.set_y_offset}\pysiglinewithargsret{\sphinxbfcode{set\_y\_offset}}{\emph{offset}}{}
Set the y offset of the area of analysis
\begin{quote}\begin{description}
\item[{Parameters}] \leavevmode
\textbf{\texttt{offset}} -- y offset of the AOA

\item[{Offset type}] \leavevmode
integer

\end{description}\end{quote}

\end{fulllineitems}

\index{set\_y\_size() (capsis.RunConfig method)}

\begin{fulllineitems}
\phantomsection\label{capsis_RunConfig:capsis.RunConfig.set_y_size}\pysiglinewithargsret{\sphinxbfcode{set\_y\_size}}{\emph{y\_size}}{}
Sets scene y dimension
\begin{quote}\begin{description}
\item[{Parameters}] \leavevmode
\textbf{\texttt{y\_size}} (\emph{\texttt{integer}}) -- size of scene in the y domain (meters)

\end{description}\end{quote}

\begin{notice}{note}{Note:}
\sphinxtitleref{y\_size} must be greater than or equal to 64 meters
\end{notice}

\end{fulllineitems}

\index{set\_z\_size() (capsis.RunConfig method)}

\begin{fulllineitems}
\phantomsection\label{capsis_RunConfig:capsis.RunConfig.set_z_size}\pysiglinewithargsret{\sphinxbfcode{set\_z\_size}}{\emph{z\_size}}{}
Sets scene z dimension
\begin{quote}\begin{description}
\item[{Parameters}] \leavevmode
\textbf{\texttt{z\_size}} (\emph{\texttt{integer}}) -- size of scene in the z domain (meters)

\end{description}\end{quote}

\begin{notice}{note}{Note:}
\sphinxtitleref{z\_size} must be greater than or equal to tallest tree in domain
\end{notice}

\end{fulllineitems}


\end{fulllineitems}



\subsection{Execute}
\label{capsis_Execute:execute}\label{capsis_Execute::doc}\index{Execute (class in capsis)}

\begin{fulllineitems}
\phantomsection\label{capsis_Execute:capsis.Execute}\pysiglinewithargsret{\sphinxstrong{class }\sphinxcode{capsis.}\sphinxbfcode{Execute}}{\emph{path\_to\_run\_file}}{}
Bases: \sphinxcode{object}

This class executes capsis according to the configuration from RunConfig.
Capsis execution is platform agnostic

\end{fulllineitems}



\section{WFDS module}
\label{wfds:module-wfds}\label{wfds::doc}\label{wfds:wfds-module}\index{wfds (module)}
The wfds module is for configuring and running WFDS simulations. Use the WFDS
class to setup the run. The WFDS class inherits the Mesh class, so meshes can
be dealt with there.

Contents:


\subsection{Mesh}
\label{wfds_Mesh:mesh}\label{wfds_Mesh::doc}\index{Mesh (class in wfds)}

\begin{fulllineitems}
\phantomsection\label{wfds_Mesh:wfds.Mesh}\pysiglinewithargsret{\sphinxstrong{class }\sphinxcode{wfds.}\sphinxbfcode{Mesh}}{\emph{x}, \emph{y}, \emph{z}, \emph{res}, \emph{n}}{}
Bases: \sphinxcode{object}

The Mesh class can be used to easily and quickly create FDS meshes.
\begin{quote}\begin{description}
\item[{Parameters}] \leavevmode\begin{itemize}
\item {} 
\textbf{\texttt{x}} -- X dimension of the simulation domain

\item {} 
\textbf{\texttt{y}} (\emph{\texttt{integer}}) -- Y dimension of the simulation domain

\item {} 
\textbf{\texttt{z}} (\emph{\texttt{integer}}) -- Z dimension of the simulation domain

\item {} 
\textbf{\texttt{res}} (\emph{\texttt{integer}}) -- 3-space resolution prior to mesh stretching

\item {} 
\textbf{\texttt{n}} -- Number of meshes

\end{itemize}

\item[{X type}] \leavevmode
integer

\item[{N type}] \leavevmode
integer

\end{description}\end{quote}

\emph{Example:}

\begin{Verbatim}[commandchars=\\\{\}]
\PYG{g+gp}{\PYGZgt{}\PYGZgt{}\PYGZgt{} }\PYG{k+kn}{import} \PYG{n+nn}{wfds}
\PYG{g+gp}{\PYGZgt{}\PYGZgt{}\PYGZgt{} }\PYG{n}{mesh} \PYG{o}{=} \PYG{n}{wfds}\PYG{o}{.}\PYG{n}{mesh}\PYG{p}{(}\PYG{l+m+mi}{160}\PYG{p}{,}\PYG{l+m+mi}{90}\PYG{p}{,}\PYG{l+m+mi}{50}\PYG{p}{,}\PYG{l+m+mi}{1}\PYG{p}{,}\PYG{l+m+mi}{9}\PYG{p}{)}
\PYG{g+gp}{\PYGZgt{}\PYGZgt{}\PYGZgt{} }\PYG{n}{mesh}\PYG{o}{.}\PYG{n}{stretch\PYGZus{}mesh}\PYG{p}{(}\PYG{p}{[}\PYG{l+m+mi}{3}\PYG{p}{,}\PYG{l+m+mi}{33}\PYG{p}{]}\PYG{p}{,} \PYG{p}{[}\PYG{l+m+mi}{1}\PYG{p}{,}\PYG{l+m+mi}{31}\PYG{p}{]}\PYG{p}{,} \PYG{n}{axis}\PYG{o}{=}\PYG{l+s+s1}{\PYGZsq{}}\PYG{l+s+s1}{z}\PYG{l+s+s1}{\PYGZsq{}}\PYG{p}{)}
\PYG{g+gp}{\PYGZgt{}\PYGZgt{}\PYGZgt{} }\PYG{n+nb}{print} \PYG{n}{mesh}\PYG{o}{.}\PYG{n}{format\PYGZus{}mesh}\PYG{p}{(}\PYG{p}{)}
\end{Verbatim}
\index{format\_mesh() (wfds.Mesh method)}

\begin{fulllineitems}
\phantomsection\label{wfds_Mesh:wfds.Mesh.format_mesh}\pysiglinewithargsret{\sphinxbfcode{format\_mesh}}{}{}
Formats mesh for WFDS input file
\begin{quote}\begin{description}
\item[{Returns}] \leavevmode
formated WFDS mesh

\item[{Return type}] \leavevmode
string

\end{description}\end{quote}

\end{fulllineitems}

\index{stretch\_mesh() (wfds.Mesh method)}

\begin{fulllineitems}
\phantomsection\label{wfds_Mesh:wfds.Mesh.stretch_mesh}\pysiglinewithargsret{\sphinxbfcode{stretch\_mesh}}{\emph{CC}, \emph{PC}, \emph{axis='z'}}{}
Apply stretching to the mesh along the specified axis
\begin{quote}\begin{description}
\item[{Parameters}] \leavevmode\begin{itemize}
\item {} 
\textbf{\texttt{CC}} (\emph{\texttt{python list}}) -- computation coordinates

\item {} 
\textbf{\texttt{PC}} (\emph{\texttt{python list}}) -- physical coordinates

\end{itemize}

\end{description}\end{quote}

\begin{notice}{note}{Note:}
CC and PC must have equal number of elements
\end{notice}
\begin{quote}\begin{description}
\item[{Example}] \leavevmode
\end{description}\end{quote}

\begin{Verbatim}[commandchars=\\\{\}]
\PYG{g+gp}{\PYGZgt{}\PYGZgt{}\PYGZgt{} }\PYG{n}{mesh} \PYG{o}{=} \PYG{n}{Mesh}\PYG{p}{(}\PYG{l+m+mi}{200}\PYG{p}{,} \PYG{l+m+mi}{150}\PYG{p}{,} \PYG{l+m+mi}{100}\PYG{p}{,} \PYG{l+m+mi}{1}\PYG{p}{,} \PYG{l+m+mi}{1}\PYG{p}{)}
\PYG{g+gp}{\PYGZgt{}\PYGZgt{}\PYGZgt{} }\PYG{n}{mesh}\PYG{o}{.}\PYG{n}{stretch\PYGZus{}mesh}\PYG{p}{(}\PYG{p}{[}\PYG{l+m+mi}{3}\PYG{p}{,}\PYG{l+m+mi}{33}\PYG{p}{]}\PYG{p}{,} \PYG{p}{[}\PYG{l+m+mi}{1}\PYG{p}{,}\PYG{l+m+mi}{31}\PYG{p}{]}\PYG{p}{,} \PYG{n}{axis}\PYG{o}{=}\PYG{l+s+s1}{\PYGZsq{}}\PYG{l+s+s1}{z}\PYG{l+s+s1}{\PYGZsq{}}\PYG{p}{)}
\PYG{g+gp}{\PYGZgt{}\PYGZgt{}\PYGZgt{} }\PYG{n+nb}{print} \PYG{n}{mesh}\PYG{o}{.}\PYG{n}{format\PYGZus{}mesh}\PYG{p}{(}\PYG{p}{)}
\end{Verbatim}

\end{fulllineitems}


\end{fulllineitems}



\subsection{WFDS}
\label{wfds_WFDS:wfds}\label{wfds_WFDS::doc}\index{WFDS (class in wfds)}

\begin{fulllineitems}
\phantomsection\label{wfds_WFDS:wfds.WFDS}\pysiglinewithargsret{\sphinxstrong{class }\sphinxcode{wfds.}\sphinxbfcode{WFDS}}{\emph{x}, \emph{y}, \emph{z}, \emph{res}, \emph{n}, \emph{fuels}}{}
Bases: {\hyperref[wfds_Mesh:wfds.Mesh]{\sphinxcrossref{\sphinxcode{wfds.Mesh}}}}

This class configures a WFDS simulation. This is a subclass of Mesh and
meshes are dealt with implicitly

\emph{Example:}

\begin{Verbatim}[commandchars=\\\{\}]
\PYG{g+gp}{\PYGZgt{}\PYGZgt{}\PYGZgt{} }\PYG{k+kn}{import} \PYG{n+nn}{wfds}
\PYG{g+gp}{\PYGZgt{}\PYGZgt{}\PYGZgt{} }\PYG{n}{fds} \PYG{o}{=} \PYG{n}{wfds}\PYG{o}{.}\PYG{n}{WFDS}\PYG{p}{(}\PYG{l+m+mi}{160}\PYG{p}{,} \PYG{l+m+mi}{90}\PYG{p}{,} \PYG{l+m+mi}{50}\PYG{p}{,} \PYG{l+m+mi}{1}\PYG{p}{,} \PYG{l+m+mi}{9}\PYG{p}{,} \PYG{n}{fuels}\PYG{p}{)}
\end{Verbatim}
\index{create\_ignition() (wfds.WFDS method)}

\begin{fulllineitems}
\phantomsection\label{wfds_WFDS:wfds.WFDS.create_ignition}\pysiglinewithargsret{\sphinxbfcode{create\_ignition}}{\emph{start\_time}, \emph{end\_time}, \emph{x0}, \emph{x1}, \emph{y0}, \emph{y1}}{}
Places an ignition strip at the specified location and generates fire
at the specified HRR for the specified duration
\begin{quote}\begin{description}
\item[{Parameters}] \leavevmode\begin{itemize}
\item {} 
\textbf{\texttt{start\_time}} (\emph{\texttt{integer}}) -- start time of the igniter fire

\item {} 
\textbf{\texttt{end\_time}} (\emph{\texttt{integer}}) -- finish time of teh igniter fire

\item {} 
\textbf{\texttt{x0}} (\emph{\texttt{float}}) -- starting x position of the ignition strip

\item {} 
\textbf{\texttt{x1}} (\emph{\texttt{float}}) -- ending x position of the ignition strip

\item {} 
\textbf{\texttt{y0}} (\emph{\texttt{float}}) -- starting y position of the ignition strip

\item {} 
\textbf{\texttt{y1}} (\emph{\texttt{float}}) -- ending y position of the ignition strip

\end{itemize}

\end{description}\end{quote}

\begin{notice}{note}{Note:}
Ignition ramping is dealt with implicitly to avoid `explosions'
\end{notice}

\end{fulllineitems}

\index{create\_mesh() (wfds.WFDS method)}

\begin{fulllineitems}
\phantomsection\label{wfds_WFDS:wfds.WFDS.create_mesh}\pysiglinewithargsret{\sphinxbfcode{create\_mesh}}{\emph{stretch=False}}{}
Call the \sphinxtitleref{stretch\_mesh()} method of the Mesh class
\begin{quote}\begin{description}
\item[{Parameters}] \leavevmode
\textbf{\texttt{stretch}} -- to stretch or not to strecth

\item[{Stretch type}] \leavevmode
False or stretch arguments

\end{description}\end{quote}

\begin{notice}{note}{Note:}
See \sphinxtitleref{Mesh.stretch\_mesh()} for setting the mesh arguments
\end{notice}

\end{fulllineitems}

\index{save\_input() (wfds.WFDS method)}

\begin{fulllineitems}
\phantomsection\label{wfds_WFDS:wfds.WFDS.save_input}\pysiglinewithargsret{\sphinxbfcode{save\_input}}{\emph{file\_name}}{}
Save the input file. Include the desired directory in the string
\begin{quote}\begin{description}
\item[{Parameters}] \leavevmode
\textbf{\texttt{file\_name}} (\emph{\texttt{string}}) -- path to and name of input file (.txt not .fds please)

\end{description}\end{quote}

\end{fulllineitems}

\index{set\_hrrpua() (wfds.WFDS method)}

\begin{fulllineitems}
\phantomsection\label{wfds_WFDS:wfds.WFDS.set_hrrpua}\pysiglinewithargsret{\sphinxbfcode{set\_hrrpua}}{\emph{hrr}}{}
Set the heat release rate per unit area for the ignition strip
\begin{quote}\begin{description}
\item[{Parameters}] \leavevmode
\textbf{\texttt{hrr}} (\emph{\texttt{integer}}) -- heat release rate per unit area (kW/m\textasciicircum{}2)

\end{description}\end{quote}

\end{fulllineitems}

\index{set\_init\_temp() (wfds.WFDS method)}

\begin{fulllineitems}
\phantomsection\label{wfds_WFDS:wfds.WFDS.set_init_temp}\pysiglinewithargsret{\sphinxbfcode{set\_init\_temp}}{\emph{temp}}{}
Set the initial temperature of the simulation
\begin{quote}\begin{description}
\item[{Parameters}] \leavevmode
\textbf{\texttt{temp}} (\emph{\texttt{float}}) -- temperature (celcius)

\end{description}\end{quote}

\end{fulllineitems}

\index{set\_run\_name() (wfds.WFDS method)}

\begin{fulllineitems}
\phantomsection\label{wfds_WFDS:wfds.WFDS.set_run_name}\pysiglinewithargsret{\sphinxbfcode{set\_run\_name}}{\emph{name}}{}
\end{fulllineitems}

\index{set\_simulation\_time() (wfds.WFDS method)}

\begin{fulllineitems}
\phantomsection\label{wfds_WFDS:wfds.WFDS.set_simulation_time}\pysiglinewithargsret{\sphinxbfcode{set\_simulation\_time}}{\emph{sim\_time}}{}
Set the duration of the simulation
\begin{quote}\begin{description}
\item[{Parameters}] \leavevmode
\textbf{\texttt{sim\_time}} (\emph{\texttt{float}}) -- duration of the simulation (s)

\end{description}\end{quote}

\end{fulllineitems}

\index{set\_wind\_speed() (wfds.WFDS method)}

\begin{fulllineitems}
\phantomsection\label{wfds_WFDS:wfds.WFDS.set_wind_speed}\pysiglinewithargsret{\sphinxbfcode{set\_wind\_speed}}{\emph{U0}}{}
Set the inflow wind speed
\begin{quote}\begin{description}
\item[{Parameters}] \leavevmode
\textbf{\texttt{U0}} (\emph{\texttt{float}}) -- inflow wind speed (m/s)

\end{description}\end{quote}

\end{fulllineitems}


\end{fulllineitems}



\subsection{Execute}
\label{wfds_Execute:execute}\label{wfds_Execute::doc}\index{Execute (class in wfds)}

\begin{fulllineitems}
\phantomsection\label{wfds_Execute:wfds.Execute}\pysiglinewithargsret{\sphinxstrong{class }\sphinxcode{wfds.}\sphinxbfcode{Execute}}{\emph{input\_file}, \emph{n\_proc}}{}
Bases: \sphinxcode{object}

The Execute class runs the WFDS input file (platform agnostic)
\begin{quote}\begin{description}
\item[{Parameters}] \leavevmode\begin{itemize}
\item {} 
\textbf{\texttt{input\_file}} (\emph{\texttt{string}}) -- path to and name of input file

\item {} 
\textbf{\texttt{n\_proc}} (\emph{\texttt{integer}}) -- number of processors

\end{itemize}

\end{description}\end{quote}

\end{fulllineitems}



\subsection{Generate Binary Grid}
\label{wfds_GenerateBinaryGrid::doc}\label{wfds_GenerateBinaryGrid:generate-binary-grid}\index{GenerateBinaryGrid (class in wfds)}

\begin{fulllineitems}
\phantomsection\label{wfds_GenerateBinaryGrid:wfds.GenerateBinaryGrid}\pysiglinewithargsret{\sphinxstrong{class }\sphinxcode{wfds.}\sphinxbfcode{GenerateBinaryGrid}}{\emph{x}, \emph{y}, \emph{z}, \emph{res}, \emph{n}, \emph{f\_name}, \emph{stretch=False}}{}
Bases: {\hyperref[wfds_Mesh:wfds.Mesh]{\sphinxcrossref{\sphinxcode{wfds.Mesh}}}}

This class is for generating binary FDS grid. Usefull for Capsis. Inherits
the mesh class.

\begin{notice}{note}{Note:}
See the wfds.Mesh() class for details
\end{notice}

\end{fulllineitems}



\section{Metrics module}
\label{metrics::doc}\label{metrics:metrics-module}\label{metrics:module-metrics}\index{metrics (module)}
This module contains class for calculating various metrics. These metrics are
specific to the WFDS configurations used in Standfire, i.e. They probably won't
work on all WFDS output direcotries.

Lots of work to do here. Work in progress.

Contents:


\subsection{Rate of Spread}
\label{metrics_ROS::doc}\label{metrics_ROS:rate-of-spread}\index{ROS (class in metrics)}

\begin{fulllineitems}
\phantomsection\label{metrics_ROS:metrics.ROS}\pysiglinewithargsret{\sphinxstrong{class }\sphinxcode{metrics.}\sphinxbfcode{ROS}}{\emph{wdir}, \emph{fuel\_1}, \emph{fuel\_2}, \emph{x\_diff}}{}
Bases: \sphinxcode{object}

Calculates rate of spread (m/s)

\# TODO: make subclass that reads the vegout files. Current each class reads them.
\index{get\_first\_burn\_time() (metrics.ROS method)}

\begin{fulllineitems}
\phantomsection\label{metrics_ROS:metrics.ROS.get_first_burn_time}\pysiglinewithargsret{\sphinxbfcode{get\_first\_burn\_time}}{\emph{fuel}}{}
Returns the time when the fuel begins to burn
\begin{quote}\begin{description}
\item[{Parameters}] \leavevmode
\textbf{\texttt{fuel}} -- The fuel that burns first

\end{description}\end{quote}

\end{fulllineitems}

\index{get\_ros() (metrics.ROS method)}

\begin{fulllineitems}
\phantomsection\label{metrics_ROS:metrics.ROS.get_ros}\pysiglinewithargsret{\sphinxbfcode{get\_ros}}{}{}
Returns the rate of spread in meters per second

\end{fulllineitems}


\end{fulllineitems}



\subsection{Mass Loss}
\label{metrics_MassLoss:mass-loss}\label{metrics_MassLoss::doc}\index{MassLoss (class in metrics)}

\begin{fulllineitems}
\phantomsection\label{metrics_MassLoss:metrics.MassLoss}\pysiglinewithargsret{\sphinxstrong{class }\sphinxcode{metrics.}\sphinxbfcode{MassLoss}}{\emph{wdir}}{}
Bases: \sphinxcode{object}

Calculates dry mass consumption
\index{get\_total\_mass\_loss() (metrics.MassLoss method)}

\begin{fulllineitems}
\phantomsection\label{metrics_MassLoss:metrics.MassLoss.get_total_mass_loss}\pysiglinewithargsret{\sphinxbfcode{get\_total\_mass\_loss}}{}{}
\end{fulllineitems}

\index{get\_tree\_files() (metrics.MassLoss method)}

\begin{fulllineitems}
\phantomsection\label{metrics_MassLoss:metrics.MassLoss.get_tree_files}\pysiglinewithargsret{\sphinxbfcode{get\_tree\_files}}{}{}
\end{fulllineitems}

\index{read\_tree\_mass() (metrics.MassLoss method)}

\begin{fulllineitems}
\phantomsection\label{metrics_MassLoss:metrics.MassLoss.read_tree_mass}\pysiglinewithargsret{\sphinxbfcode{read\_tree\_mass}}{}{}
\end{fulllineitems}


\end{fulllineitems}



\subsection{Wind Profile}
\label{metrics_WindProfile:wind-profile}\label{metrics_WindProfile::doc}\index{WindProfile (class in metrics)}

\begin{fulllineitems}
\phantomsection\label{metrics_WindProfile:metrics.WindProfile}\pysiglinewithargsret{\sphinxstrong{class }\sphinxcode{metrics.}\sphinxbfcode{WindProfile}}{\emph{wdir}, \emph{slice\_file}, \emph{t\_start}, \emph{t\_end}, \emph{t\_step}}{}
Bases: \sphinxcode{object}

Calculates wind profile
\index{get\_wind\_profile() (metrics.WindProfile method)}

\begin{fulllineitems}
\phantomsection\label{metrics_WindProfile:metrics.WindProfile.get_wind_profile}\pysiglinewithargsret{\sphinxbfcode{get\_wind\_profile}}{}{}
\end{fulllineitems}


\end{fulllineitems}



\subsection{Heat Transfer}
\label{metrics_HeatTransfer::doc}\label{metrics_HeatTransfer:heat-transfer}\index{HeatTransfer (class in metrics)}

\begin{fulllineitems}
\phantomsection\label{metrics_HeatTransfer:metrics.HeatTransfer}\pysiglinewithargsret{\sphinxstrong{class }\sphinxcode{metrics.}\sphinxbfcode{HeatTransfer}}{\emph{wdir}}{}
Bases: \sphinxcode{object}

Calculates crown heat transfer
\index{get\_tree\_files() (metrics.HeatTransfer method)}

\begin{fulllineitems}
\phantomsection\label{metrics_HeatTransfer:metrics.HeatTransfer.get_tree_files}\pysiglinewithargsret{\sphinxbfcode{get\_tree\_files}}{}{}
\end{fulllineitems}

\index{read\_tree\_conv() (metrics.HeatTransfer method)}

\begin{fulllineitems}
\phantomsection\label{metrics_HeatTransfer:metrics.HeatTransfer.read_tree_conv}\pysiglinewithargsret{\sphinxbfcode{read\_tree\_conv}}{}{}
\end{fulllineitems}

\index{read\_tree\_rad() (metrics.HeatTransfer method)}

\begin{fulllineitems}
\phantomsection\label{metrics_HeatTransfer:metrics.HeatTransfer.read_tree_rad}\pysiglinewithargsret{\sphinxbfcode{read\_tree\_rad}}{}{}
\end{fulllineitems}


\end{fulllineitems}



\chapter{Indices and tables}
\label{index:indices-and-tables}\begin{itemize}
\item {} 
\DUrole{xref,std,std-ref}{genindex}

\item {} 
\DUrole{xref,std,std-ref}{modindex}

\item {} 
\DUrole{xref,std,std-ref}{search}

\end{itemize}


\renewcommand{\indexname}{Python Module Index}
\begin{theindex}
\def\bigletter#1{{\Large\sffamily#1}\nopagebreak\vspace{1mm}}
\bigletter{c}
\item {\texttt{capsis}}, \pageref{capsis:module-capsis}
\indexspace
\bigletter{f}
\item {\texttt{fuels}}, \pageref{fuels:module-fuels}
\indexspace
\bigletter{m}
\item {\texttt{metrics}}, \pageref{metrics:module-metrics}
\indexspace
\bigletter{w}
\item {\texttt{wfds}}, \pageref{wfds:module-wfds}
\end{theindex}

\renewcommand{\indexname}{Index}
\printindex
\end{document}
