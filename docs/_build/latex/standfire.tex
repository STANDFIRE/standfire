% Generated by Sphinx.
\def\sphinxdocclass{report}
\documentclass[letterpaper,10pt,english]{sphinxmanual}
\usepackage[utf8]{inputenc}
\DeclareUnicodeCharacter{00A0}{\nobreakspace}
\usepackage{cmap}
\usepackage[T1]{fontenc}
\usepackage{babel}
\usepackage{times}
\usepackage[Sonny]{fncychap}
\usepackage{longtable}
\usepackage{sphinx}
\usepackage{multirow}


\addto\captionsenglish{\renewcommand{\figurename}{Fig. }}
\addto\captionsenglish{\renewcommand{\tablename}{Table }}
\floatname{literal-block}{Listing }



\title{standfire Documentation}
\date{December 07, 2015}
\release{}
\author{Author}
\newcommand{\sphinxlogo}{}
\renewcommand{\releasename}{Release}
\makeindex

\makeatletter
\def\PYG@reset{\let\PYG@it=\relax \let\PYG@bf=\relax%
    \let\PYG@ul=\relax \let\PYG@tc=\relax%
    \let\PYG@bc=\relax \let\PYG@ff=\relax}
\def\PYG@tok#1{\csname PYG@tok@#1\endcsname}
\def\PYG@toks#1+{\ifx\relax#1\empty\else%
    \PYG@tok{#1}\expandafter\PYG@toks\fi}
\def\PYG@do#1{\PYG@bc{\PYG@tc{\PYG@ul{%
    \PYG@it{\PYG@bf{\PYG@ff{#1}}}}}}}
\def\PYG#1#2{\PYG@reset\PYG@toks#1+\relax+\PYG@do{#2}}

\expandafter\def\csname PYG@tok@gd\endcsname{\def\PYG@tc##1{\textcolor[rgb]{0.63,0.00,0.00}{##1}}}
\expandafter\def\csname PYG@tok@gu\endcsname{\let\PYG@bf=\textbf\def\PYG@tc##1{\textcolor[rgb]{0.50,0.00,0.50}{##1}}}
\expandafter\def\csname PYG@tok@gt\endcsname{\def\PYG@tc##1{\textcolor[rgb]{0.00,0.27,0.87}{##1}}}
\expandafter\def\csname PYG@tok@gs\endcsname{\let\PYG@bf=\textbf}
\expandafter\def\csname PYG@tok@gr\endcsname{\def\PYG@tc##1{\textcolor[rgb]{1.00,0.00,0.00}{##1}}}
\expandafter\def\csname PYG@tok@cm\endcsname{\let\PYG@it=\textit\def\PYG@tc##1{\textcolor[rgb]{0.25,0.50,0.56}{##1}}}
\expandafter\def\csname PYG@tok@vg\endcsname{\def\PYG@tc##1{\textcolor[rgb]{0.73,0.38,0.84}{##1}}}
\expandafter\def\csname PYG@tok@m\endcsname{\def\PYG@tc##1{\textcolor[rgb]{0.13,0.50,0.31}{##1}}}
\expandafter\def\csname PYG@tok@mh\endcsname{\def\PYG@tc##1{\textcolor[rgb]{0.13,0.50,0.31}{##1}}}
\expandafter\def\csname PYG@tok@cs\endcsname{\def\PYG@tc##1{\textcolor[rgb]{0.25,0.50,0.56}{##1}}\def\PYG@bc##1{\setlength{\fboxsep}{0pt}\colorbox[rgb]{1.00,0.94,0.94}{\strut ##1}}}
\expandafter\def\csname PYG@tok@ge\endcsname{\let\PYG@it=\textit}
\expandafter\def\csname PYG@tok@vc\endcsname{\def\PYG@tc##1{\textcolor[rgb]{0.73,0.38,0.84}{##1}}}
\expandafter\def\csname PYG@tok@il\endcsname{\def\PYG@tc##1{\textcolor[rgb]{0.13,0.50,0.31}{##1}}}
\expandafter\def\csname PYG@tok@go\endcsname{\def\PYG@tc##1{\textcolor[rgb]{0.20,0.20,0.20}{##1}}}
\expandafter\def\csname PYG@tok@cp\endcsname{\def\PYG@tc##1{\textcolor[rgb]{0.00,0.44,0.13}{##1}}}
\expandafter\def\csname PYG@tok@gi\endcsname{\def\PYG@tc##1{\textcolor[rgb]{0.00,0.63,0.00}{##1}}}
\expandafter\def\csname PYG@tok@gh\endcsname{\let\PYG@bf=\textbf\def\PYG@tc##1{\textcolor[rgb]{0.00,0.00,0.50}{##1}}}
\expandafter\def\csname PYG@tok@ni\endcsname{\let\PYG@bf=\textbf\def\PYG@tc##1{\textcolor[rgb]{0.84,0.33,0.22}{##1}}}
\expandafter\def\csname PYG@tok@nl\endcsname{\let\PYG@bf=\textbf\def\PYG@tc##1{\textcolor[rgb]{0.00,0.13,0.44}{##1}}}
\expandafter\def\csname PYG@tok@nn\endcsname{\let\PYG@bf=\textbf\def\PYG@tc##1{\textcolor[rgb]{0.05,0.52,0.71}{##1}}}
\expandafter\def\csname PYG@tok@no\endcsname{\def\PYG@tc##1{\textcolor[rgb]{0.38,0.68,0.84}{##1}}}
\expandafter\def\csname PYG@tok@na\endcsname{\def\PYG@tc##1{\textcolor[rgb]{0.25,0.44,0.63}{##1}}}
\expandafter\def\csname PYG@tok@nb\endcsname{\def\PYG@tc##1{\textcolor[rgb]{0.00,0.44,0.13}{##1}}}
\expandafter\def\csname PYG@tok@nc\endcsname{\let\PYG@bf=\textbf\def\PYG@tc##1{\textcolor[rgb]{0.05,0.52,0.71}{##1}}}
\expandafter\def\csname PYG@tok@nd\endcsname{\let\PYG@bf=\textbf\def\PYG@tc##1{\textcolor[rgb]{0.33,0.33,0.33}{##1}}}
\expandafter\def\csname PYG@tok@ne\endcsname{\def\PYG@tc##1{\textcolor[rgb]{0.00,0.44,0.13}{##1}}}
\expandafter\def\csname PYG@tok@nf\endcsname{\def\PYG@tc##1{\textcolor[rgb]{0.02,0.16,0.49}{##1}}}
\expandafter\def\csname PYG@tok@si\endcsname{\let\PYG@it=\textit\def\PYG@tc##1{\textcolor[rgb]{0.44,0.63,0.82}{##1}}}
\expandafter\def\csname PYG@tok@s2\endcsname{\def\PYG@tc##1{\textcolor[rgb]{0.25,0.44,0.63}{##1}}}
\expandafter\def\csname PYG@tok@vi\endcsname{\def\PYG@tc##1{\textcolor[rgb]{0.73,0.38,0.84}{##1}}}
\expandafter\def\csname PYG@tok@nt\endcsname{\let\PYG@bf=\textbf\def\PYG@tc##1{\textcolor[rgb]{0.02,0.16,0.45}{##1}}}
\expandafter\def\csname PYG@tok@nv\endcsname{\def\PYG@tc##1{\textcolor[rgb]{0.73,0.38,0.84}{##1}}}
\expandafter\def\csname PYG@tok@s1\endcsname{\def\PYG@tc##1{\textcolor[rgb]{0.25,0.44,0.63}{##1}}}
\expandafter\def\csname PYG@tok@gp\endcsname{\let\PYG@bf=\textbf\def\PYG@tc##1{\textcolor[rgb]{0.78,0.36,0.04}{##1}}}
\expandafter\def\csname PYG@tok@sh\endcsname{\def\PYG@tc##1{\textcolor[rgb]{0.25,0.44,0.63}{##1}}}
\expandafter\def\csname PYG@tok@ow\endcsname{\let\PYG@bf=\textbf\def\PYG@tc##1{\textcolor[rgb]{0.00,0.44,0.13}{##1}}}
\expandafter\def\csname PYG@tok@sx\endcsname{\def\PYG@tc##1{\textcolor[rgb]{0.78,0.36,0.04}{##1}}}
\expandafter\def\csname PYG@tok@bp\endcsname{\def\PYG@tc##1{\textcolor[rgb]{0.00,0.44,0.13}{##1}}}
\expandafter\def\csname PYG@tok@c1\endcsname{\let\PYG@it=\textit\def\PYG@tc##1{\textcolor[rgb]{0.25,0.50,0.56}{##1}}}
\expandafter\def\csname PYG@tok@kc\endcsname{\let\PYG@bf=\textbf\def\PYG@tc##1{\textcolor[rgb]{0.00,0.44,0.13}{##1}}}
\expandafter\def\csname PYG@tok@c\endcsname{\let\PYG@it=\textit\def\PYG@tc##1{\textcolor[rgb]{0.25,0.50,0.56}{##1}}}
\expandafter\def\csname PYG@tok@mf\endcsname{\def\PYG@tc##1{\textcolor[rgb]{0.13,0.50,0.31}{##1}}}
\expandafter\def\csname PYG@tok@err\endcsname{\def\PYG@bc##1{\setlength{\fboxsep}{0pt}\fcolorbox[rgb]{1.00,0.00,0.00}{1,1,1}{\strut ##1}}}
\expandafter\def\csname PYG@tok@mb\endcsname{\def\PYG@tc##1{\textcolor[rgb]{0.13,0.50,0.31}{##1}}}
\expandafter\def\csname PYG@tok@ss\endcsname{\def\PYG@tc##1{\textcolor[rgb]{0.32,0.47,0.09}{##1}}}
\expandafter\def\csname PYG@tok@sr\endcsname{\def\PYG@tc##1{\textcolor[rgb]{0.14,0.33,0.53}{##1}}}
\expandafter\def\csname PYG@tok@mo\endcsname{\def\PYG@tc##1{\textcolor[rgb]{0.13,0.50,0.31}{##1}}}
\expandafter\def\csname PYG@tok@kd\endcsname{\let\PYG@bf=\textbf\def\PYG@tc##1{\textcolor[rgb]{0.00,0.44,0.13}{##1}}}
\expandafter\def\csname PYG@tok@mi\endcsname{\def\PYG@tc##1{\textcolor[rgb]{0.13,0.50,0.31}{##1}}}
\expandafter\def\csname PYG@tok@kn\endcsname{\let\PYG@bf=\textbf\def\PYG@tc##1{\textcolor[rgb]{0.00,0.44,0.13}{##1}}}
\expandafter\def\csname PYG@tok@o\endcsname{\def\PYG@tc##1{\textcolor[rgb]{0.40,0.40,0.40}{##1}}}
\expandafter\def\csname PYG@tok@kr\endcsname{\let\PYG@bf=\textbf\def\PYG@tc##1{\textcolor[rgb]{0.00,0.44,0.13}{##1}}}
\expandafter\def\csname PYG@tok@s\endcsname{\def\PYG@tc##1{\textcolor[rgb]{0.25,0.44,0.63}{##1}}}
\expandafter\def\csname PYG@tok@kp\endcsname{\def\PYG@tc##1{\textcolor[rgb]{0.00,0.44,0.13}{##1}}}
\expandafter\def\csname PYG@tok@w\endcsname{\def\PYG@tc##1{\textcolor[rgb]{0.73,0.73,0.73}{##1}}}
\expandafter\def\csname PYG@tok@kt\endcsname{\def\PYG@tc##1{\textcolor[rgb]{0.56,0.13,0.00}{##1}}}
\expandafter\def\csname PYG@tok@sc\endcsname{\def\PYG@tc##1{\textcolor[rgb]{0.25,0.44,0.63}{##1}}}
\expandafter\def\csname PYG@tok@sb\endcsname{\def\PYG@tc##1{\textcolor[rgb]{0.25,0.44,0.63}{##1}}}
\expandafter\def\csname PYG@tok@k\endcsname{\let\PYG@bf=\textbf\def\PYG@tc##1{\textcolor[rgb]{0.00,0.44,0.13}{##1}}}
\expandafter\def\csname PYG@tok@se\endcsname{\let\PYG@bf=\textbf\def\PYG@tc##1{\textcolor[rgb]{0.25,0.44,0.63}{##1}}}
\expandafter\def\csname PYG@tok@sd\endcsname{\let\PYG@it=\textit\def\PYG@tc##1{\textcolor[rgb]{0.25,0.44,0.63}{##1}}}

\def\PYGZbs{\char`\\}
\def\PYGZus{\char`\_}
\def\PYGZob{\char`\{}
\def\PYGZcb{\char`\}}
\def\PYGZca{\char`\^}
\def\PYGZam{\char`\&}
\def\PYGZlt{\char`\<}
\def\PYGZgt{\char`\>}
\def\PYGZsh{\char`\#}
\def\PYGZpc{\char`\%}
\def\PYGZdl{\char`\$}
\def\PYGZhy{\char`\-}
\def\PYGZsq{\char`\'}
\def\PYGZdq{\char`\"}
\def\PYGZti{\char`\~}
% for compatibility with earlier versions
\def\PYGZat{@}
\def\PYGZlb{[}
\def\PYGZrb{]}
\makeatother

\renewcommand\PYGZsq{\textquotesingle}

\begin{document}

\maketitle
\tableofcontents
\phantomsection\label{index::doc}


Contents:


\chapter{fuels module}
\label{fuels:welcome-to-standfire-s-documentation}\label{fuels:module-fuels}\label{fuels::doc}\label{fuels:fuels-module}\index{fuels (module)}
Created on Mon Dec  7 17:49:52 2015

@author: lucas wells
\index{Fvsfuels (class in fuels)}

\begin{fulllineitems}
\phantomsection\label{fuels:fuels.Fvsfuels}\pysiglinewithargsret{\strong{class }\code{fuels.}\bfcode{Fvsfuels}}{\emph{variant}}{}
Bases: \code{object}
\index{get\_obj\_data() (fuels.Fvsfuels method)}

\begin{fulllineitems}
\phantomsection\label{fuels:fuels.Fvsfuels.get_obj_data}\pysiglinewithargsret{\bfcode{get\_obj\_data}}{}{}
\end{fulllineitems}

\index{get\_simulation\_years() (fuels.Fvsfuels method)}

\begin{fulllineitems}
\phantomsection\label{fuels:fuels.Fvsfuels.get_simulation_years}\pysiglinewithargsret{\bfcode{get\_simulation\_years}}{}{}
\end{fulllineitems}

\index{get\_snags() (fuels.Fvsfuels method)}

\begin{fulllineitems}
\phantomsection\label{fuels:fuels.Fvsfuels.get_snags}\pysiglinewithargsret{\bfcode{get\_snags}}{\emph{year}}{}
\end{fulllineitems}

\index{get\_spcodes() (fuels.Fvsfuels method)}

\begin{fulllineitems}
\phantomsection\label{fuels:fuels.Fvsfuels.get_spcodes}\pysiglinewithargsret{\bfcode{get\_spcodes}}{}{}
\end{fulllineitems}

\index{get\_standid() (fuels.Fvsfuels method)}

\begin{fulllineitems}
\phantomsection\label{fuels:fuels.Fvsfuels.get_standid}\pysiglinewithargsret{\bfcode{get\_standid}}{}{}
\end{fulllineitems}

\index{get\_trees() (fuels.Fvsfuels method)}

\begin{fulllineitems}
\phantomsection\label{fuels:fuels.Fvsfuels.get_trees}\pysiglinewithargsret{\bfcode{get\_trees}}{\emph{year}}{}
\end{fulllineitems}

\index{run\_fvs() (fuels.Fvsfuels method)}

\begin{fulllineitems}
\phantomsection\label{fuels:fuels.Fvsfuels.run_fvs}\pysiglinewithargsret{\bfcode{run\_fvs}}{}{}
\end{fulllineitems}

\index{save\_all() (fuels.Fvsfuels method)}

\begin{fulllineitems}
\phantomsection\label{fuels:fuels.Fvsfuels.save_all}\pysiglinewithargsret{\bfcode{save\_all}}{}{}
\end{fulllineitems}

\index{save\_snags\_by\_year() (fuels.Fvsfuels method)}

\begin{fulllineitems}
\phantomsection\label{fuels:fuels.Fvsfuels.save_snags_by_year}\pysiglinewithargsret{\bfcode{save\_snags\_by\_year}}{\emph{year}}{}
\end{fulllineitems}

\index{save\_trees\_by\_year() (fuels.Fvsfuels method)}

\begin{fulllineitems}
\phantomsection\label{fuels:fuels.Fvsfuels.save_trees_by_year}\pysiglinewithargsret{\bfcode{save\_trees\_by\_year}}{\emph{year}}{}
\end{fulllineitems}

\index{set\_dir() (fuels.Fvsfuels method)}

\begin{fulllineitems}
\phantomsection\label{fuels:fuels.Fvsfuels.set_dir}\pysiglinewithargsret{\bfcode{set\_dir}}{\emph{wdir}}{}
Sets the working directory of a Fvsfuels object

This methods is called by Fvsfuels.set\_keyword(). Thus, the default
working directory is the folder containing the specified keyword file. 
If you wish to store simulation outputs in a different directory then
use this methods to do so.
\begin{quote}\begin{description}
\item[{Parameters}] \leavevmode
\textbf{\texttt{wdir}} (\emph{\texttt{string}}) -- path/to/desired\_directory

\item[{Example}] \leavevmode
\end{description}\end{quote}

\begin{Verbatim}[commandchars=\\\{\}]
\PYG{g+gp}{\PYGZgt{}\PYGZgt{}\PYGZgt{} }\PYG{k+kn}{from} \PYG{n+nn}{standfire.fuel} \PYG{k+kn}{import} \PYG{n}{Fvsfuels}
\PYG{g+gp}{\PYGZgt{}\PYGZgt{}\PYGZgt{} }\PYG{n}{test} \PYG{o}{=} \PYG{n}{Fvsfuels}\PYG{p}{(}\PYG{l+s}{\PYGZdq{}}\PYG{l+s}{emc}\PYG{l+s}{\PYGZdq{}}\PYG{p}{)}
\PYG{g+gp}{\PYGZgt{}\PYGZgt{}\PYGZgt{} }\PYG{n}{test}\PYG{o}{.}\PYG{n}{set\PYGZus{}keyword}\PYG{p}{(}\PYG{l+s}{\PYGZdq{}}\PYG{l+s}{/Users/standfire/test/example.key}\PYG{l+s}{\PYGZdq{}}\PYG{p}{)}
\end{Verbatim}

Whoops, I would like to store simulation outputs elsewhere...

\begin{Verbatim}[commandchars=\\\{\}]
\PYG{g+gp}{\PYGZgt{}\PYGZgt{}\PYGZgt{} }\PYG{n}{test}\PYG{o}{.}\PYG{n}{set\PYGZus{}dir}\PYG{p}{(}\PYG{l+s}{\PYGZdq{}}\PYG{l+s}{/Users/standfire/outputs/}\PYG{l+s}{\PYGZdq{}}\PYG{p}{)}
\end{Verbatim}

\end{fulllineitems}

\index{set\_inv\_year() (fuels.Fvsfuels method)}

\begin{fulllineitems}
\phantomsection\label{fuels:fuels.Fvsfuels.set_inv_year}\pysiglinewithargsret{\bfcode{set\_inv\_year}}{\emph{inv\_year}}{}
\end{fulllineitems}

\index{set\_keyword() (fuels.Fvsfuels method)}

\begin{fulllineitems}
\phantomsection\label{fuels:fuels.Fvsfuels.set_keyword}\pysiglinewithargsret{\bfcode{set\_keyword}}{\emph{keyfile}}{}
Sets the keyword file to be used in the FVS simulation

This method will initalize a FVS simulation by registering the
specified keyword file (.key) with FVS. The working directory of a
Fvsfuels object will be set to the folder containing the keyword file. 
You can manually change the working directory with Fvsfuels.set\_dir(). 
This function will also call private methods in this class to extract
information from the keyword file and set class fields accordingly for
use in other methods.
\begin{quote}\begin{description}
\item[{Parameters}] \leavevmode
\textbf{\texttt{keyfile}} (\emph{\texttt{string}}) -- path/to/keyword\_file. This must have a .key extension

\item[{Example}] \leavevmode
\end{description}\end{quote}

\begin{Verbatim}[commandchars=\\\{\}]
\PYG{g+gp}{\PYGZgt{}\PYGZgt{}\PYGZgt{} }\PYG{k+kn}{from} \PYG{n+nn}{standfire.fuels} \PYG{k+kn}{import} \PYG{n}{Fvsfuels}
\PYG{g+gp}{\PYGZgt{}\PYGZgt{}\PYGZgt{} }\PYG{n}{test} \PYG{o}{=} \PYG{n}{Fvsfuels}\PYG{p}{(}\PYG{l+s}{\PYGZdq{}}\PYG{l+s}{iec}\PYG{l+s}{\PYGZdq{}}\PYG{p}{)}
\PYG{g+gp}{\PYGZgt{}\PYGZgt{}\PYGZgt{} }\PYG{n}{test}\PYG{o}{.}\PYG{n}{set\PYGZus{}keyword}\PYG{p}{(}\PYG{l+s}{\PYGZdq{}}\PYG{l+s}{/Users/standfire/test/example.key}\PYG{l+s}{\PYGZdq{}}\PYG{p}{)}
\end{Verbatim}

\end{fulllineitems}

\index{set\_num\_cycles() (fuels.Fvsfuels method)}

\begin{fulllineitems}
\phantomsection\label{fuels:fuels.Fvsfuels.set_num_cycles}\pysiglinewithargsret{\bfcode{set\_num\_cycles}}{\emph{num\_cyc}}{}
\end{fulllineitems}

\index{set\_time\_int() (fuels.Fvsfuels method)}

\begin{fulllineitems}
\phantomsection\label{fuels:fuels.Fvsfuels.set_time_int}\pysiglinewithargsret{\bfcode{set\_time\_int}}{\emph{time\_int}}{}
\end{fulllineitems}


\end{fulllineitems}



\chapter{Indices and tables}
\label{index:indices-and-tables}\begin{itemize}
\item {} 
\DUspan{xref,std,std-ref}{genindex}

\item {} 
\DUspan{xref,std,std-ref}{modindex}

\item {} 
\DUspan{xref,std,std-ref}{search}

\end{itemize}


\renewcommand{\indexname}{Python Module Index}
\begin{theindex}
\def\bigletter#1{{\Large\sffamily#1}\nopagebreak\vspace{1mm}}
\bigletter{f}
\item {\texttt{fuels}}, \pageref{fuels:module-fuels}
\end{theindex}

\renewcommand{\indexname}{Index}
\printindex
\end{document}
