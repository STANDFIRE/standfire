% Generated by Sphinx.
\def\sphinxdocclass{report}
\documentclass[letterpaper,10pt,english]{sphinxmanual}
\usepackage[utf8]{inputenc}
\DeclareUnicodeCharacter{00A0}{\nobreakspace}
\usepackage{cmap}
\usepackage[T1]{fontenc}
\usepackage{babel}
\usepackage{times}
\usepackage[Sonny]{fncychap}
\usepackage{longtable}
\usepackage{sphinx}
\usepackage{multirow}


\addto\captionsenglish{\renewcommand{\figurename}{Fig. }}
\addto\captionsenglish{\renewcommand{\tablename}{Table }}
\floatname{literal-block}{Listing }



\title{STANDFIRE}
\date{June 30, 2016}
\release{}
\author{}
\newcommand{\sphinxlogo}{}
\renewcommand{\releasename}{Release}
\makeindex

\makeatletter
\def\PYG@reset{\let\PYG@it=\relax \let\PYG@bf=\relax%
    \let\PYG@ul=\relax \let\PYG@tc=\relax%
    \let\PYG@bc=\relax \let\PYG@ff=\relax}
\def\PYG@tok#1{\csname PYG@tok@#1\endcsname}
\def\PYG@toks#1+{\ifx\relax#1\empty\else%
    \PYG@tok{#1}\expandafter\PYG@toks\fi}
\def\PYG@do#1{\PYG@bc{\PYG@tc{\PYG@ul{%
    \PYG@it{\PYG@bf{\PYG@ff{#1}}}}}}}
\def\PYG#1#2{\PYG@reset\PYG@toks#1+\relax+\PYG@do{#2}}

\expandafter\def\csname PYG@tok@gd\endcsname{\def\PYG@tc##1{\textcolor[rgb]{0.63,0.00,0.00}{##1}}}
\expandafter\def\csname PYG@tok@gu\endcsname{\let\PYG@bf=\textbf\def\PYG@tc##1{\textcolor[rgb]{0.50,0.00,0.50}{##1}}}
\expandafter\def\csname PYG@tok@gt\endcsname{\def\PYG@tc##1{\textcolor[rgb]{0.00,0.27,0.87}{##1}}}
\expandafter\def\csname PYG@tok@gs\endcsname{\let\PYG@bf=\textbf}
\expandafter\def\csname PYG@tok@gr\endcsname{\def\PYG@tc##1{\textcolor[rgb]{1.00,0.00,0.00}{##1}}}
\expandafter\def\csname PYG@tok@cm\endcsname{\let\PYG@it=\textit\def\PYG@tc##1{\textcolor[rgb]{0.25,0.50,0.56}{##1}}}
\expandafter\def\csname PYG@tok@vg\endcsname{\def\PYG@tc##1{\textcolor[rgb]{0.73,0.38,0.84}{##1}}}
\expandafter\def\csname PYG@tok@m\endcsname{\def\PYG@tc##1{\textcolor[rgb]{0.13,0.50,0.31}{##1}}}
\expandafter\def\csname PYG@tok@mh\endcsname{\def\PYG@tc##1{\textcolor[rgb]{0.13,0.50,0.31}{##1}}}
\expandafter\def\csname PYG@tok@cs\endcsname{\def\PYG@tc##1{\textcolor[rgb]{0.25,0.50,0.56}{##1}}\def\PYG@bc##1{\setlength{\fboxsep}{0pt}\colorbox[rgb]{1.00,0.94,0.94}{\strut ##1}}}
\expandafter\def\csname PYG@tok@ge\endcsname{\let\PYG@it=\textit}
\expandafter\def\csname PYG@tok@vc\endcsname{\def\PYG@tc##1{\textcolor[rgb]{0.73,0.38,0.84}{##1}}}
\expandafter\def\csname PYG@tok@il\endcsname{\def\PYG@tc##1{\textcolor[rgb]{0.13,0.50,0.31}{##1}}}
\expandafter\def\csname PYG@tok@go\endcsname{\def\PYG@tc##1{\textcolor[rgb]{0.20,0.20,0.20}{##1}}}
\expandafter\def\csname PYG@tok@cp\endcsname{\def\PYG@tc##1{\textcolor[rgb]{0.00,0.44,0.13}{##1}}}
\expandafter\def\csname PYG@tok@gi\endcsname{\def\PYG@tc##1{\textcolor[rgb]{0.00,0.63,0.00}{##1}}}
\expandafter\def\csname PYG@tok@gh\endcsname{\let\PYG@bf=\textbf\def\PYG@tc##1{\textcolor[rgb]{0.00,0.00,0.50}{##1}}}
\expandafter\def\csname PYG@tok@ni\endcsname{\let\PYG@bf=\textbf\def\PYG@tc##1{\textcolor[rgb]{0.84,0.33,0.22}{##1}}}
\expandafter\def\csname PYG@tok@nl\endcsname{\let\PYG@bf=\textbf\def\PYG@tc##1{\textcolor[rgb]{0.00,0.13,0.44}{##1}}}
\expandafter\def\csname PYG@tok@nn\endcsname{\let\PYG@bf=\textbf\def\PYG@tc##1{\textcolor[rgb]{0.05,0.52,0.71}{##1}}}
\expandafter\def\csname PYG@tok@no\endcsname{\def\PYG@tc##1{\textcolor[rgb]{0.38,0.68,0.84}{##1}}}
\expandafter\def\csname PYG@tok@na\endcsname{\def\PYG@tc##1{\textcolor[rgb]{0.25,0.44,0.63}{##1}}}
\expandafter\def\csname PYG@tok@nb\endcsname{\def\PYG@tc##1{\textcolor[rgb]{0.00,0.44,0.13}{##1}}}
\expandafter\def\csname PYG@tok@nc\endcsname{\let\PYG@bf=\textbf\def\PYG@tc##1{\textcolor[rgb]{0.05,0.52,0.71}{##1}}}
\expandafter\def\csname PYG@tok@nd\endcsname{\let\PYG@bf=\textbf\def\PYG@tc##1{\textcolor[rgb]{0.33,0.33,0.33}{##1}}}
\expandafter\def\csname PYG@tok@ne\endcsname{\def\PYG@tc##1{\textcolor[rgb]{0.00,0.44,0.13}{##1}}}
\expandafter\def\csname PYG@tok@nf\endcsname{\def\PYG@tc##1{\textcolor[rgb]{0.02,0.16,0.49}{##1}}}
\expandafter\def\csname PYG@tok@si\endcsname{\let\PYG@it=\textit\def\PYG@tc##1{\textcolor[rgb]{0.44,0.63,0.82}{##1}}}
\expandafter\def\csname PYG@tok@s2\endcsname{\def\PYG@tc##1{\textcolor[rgb]{0.25,0.44,0.63}{##1}}}
\expandafter\def\csname PYG@tok@vi\endcsname{\def\PYG@tc##1{\textcolor[rgb]{0.73,0.38,0.84}{##1}}}
\expandafter\def\csname PYG@tok@nt\endcsname{\let\PYG@bf=\textbf\def\PYG@tc##1{\textcolor[rgb]{0.02,0.16,0.45}{##1}}}
\expandafter\def\csname PYG@tok@nv\endcsname{\def\PYG@tc##1{\textcolor[rgb]{0.73,0.38,0.84}{##1}}}
\expandafter\def\csname PYG@tok@s1\endcsname{\def\PYG@tc##1{\textcolor[rgb]{0.25,0.44,0.63}{##1}}}
\expandafter\def\csname PYG@tok@gp\endcsname{\let\PYG@bf=\textbf\def\PYG@tc##1{\textcolor[rgb]{0.78,0.36,0.04}{##1}}}
\expandafter\def\csname PYG@tok@sh\endcsname{\def\PYG@tc##1{\textcolor[rgb]{0.25,0.44,0.63}{##1}}}
\expandafter\def\csname PYG@tok@ow\endcsname{\let\PYG@bf=\textbf\def\PYG@tc##1{\textcolor[rgb]{0.00,0.44,0.13}{##1}}}
\expandafter\def\csname PYG@tok@sx\endcsname{\def\PYG@tc##1{\textcolor[rgb]{0.78,0.36,0.04}{##1}}}
\expandafter\def\csname PYG@tok@bp\endcsname{\def\PYG@tc##1{\textcolor[rgb]{0.00,0.44,0.13}{##1}}}
\expandafter\def\csname PYG@tok@c1\endcsname{\let\PYG@it=\textit\def\PYG@tc##1{\textcolor[rgb]{0.25,0.50,0.56}{##1}}}
\expandafter\def\csname PYG@tok@kc\endcsname{\let\PYG@bf=\textbf\def\PYG@tc##1{\textcolor[rgb]{0.00,0.44,0.13}{##1}}}
\expandafter\def\csname PYG@tok@c\endcsname{\let\PYG@it=\textit\def\PYG@tc##1{\textcolor[rgb]{0.25,0.50,0.56}{##1}}}
\expandafter\def\csname PYG@tok@mf\endcsname{\def\PYG@tc##1{\textcolor[rgb]{0.13,0.50,0.31}{##1}}}
\expandafter\def\csname PYG@tok@err\endcsname{\def\PYG@bc##1{\setlength{\fboxsep}{0pt}\fcolorbox[rgb]{1.00,0.00,0.00}{1,1,1}{\strut ##1}}}
\expandafter\def\csname PYG@tok@mb\endcsname{\def\PYG@tc##1{\textcolor[rgb]{0.13,0.50,0.31}{##1}}}
\expandafter\def\csname PYG@tok@ss\endcsname{\def\PYG@tc##1{\textcolor[rgb]{0.32,0.47,0.09}{##1}}}
\expandafter\def\csname PYG@tok@sr\endcsname{\def\PYG@tc##1{\textcolor[rgb]{0.14,0.33,0.53}{##1}}}
\expandafter\def\csname PYG@tok@mo\endcsname{\def\PYG@tc##1{\textcolor[rgb]{0.13,0.50,0.31}{##1}}}
\expandafter\def\csname PYG@tok@kd\endcsname{\let\PYG@bf=\textbf\def\PYG@tc##1{\textcolor[rgb]{0.00,0.44,0.13}{##1}}}
\expandafter\def\csname PYG@tok@mi\endcsname{\def\PYG@tc##1{\textcolor[rgb]{0.13,0.50,0.31}{##1}}}
\expandafter\def\csname PYG@tok@kn\endcsname{\let\PYG@bf=\textbf\def\PYG@tc##1{\textcolor[rgb]{0.00,0.44,0.13}{##1}}}
\expandafter\def\csname PYG@tok@o\endcsname{\def\PYG@tc##1{\textcolor[rgb]{0.40,0.40,0.40}{##1}}}
\expandafter\def\csname PYG@tok@kr\endcsname{\let\PYG@bf=\textbf\def\PYG@tc##1{\textcolor[rgb]{0.00,0.44,0.13}{##1}}}
\expandafter\def\csname PYG@tok@s\endcsname{\def\PYG@tc##1{\textcolor[rgb]{0.25,0.44,0.63}{##1}}}
\expandafter\def\csname PYG@tok@kp\endcsname{\def\PYG@tc##1{\textcolor[rgb]{0.00,0.44,0.13}{##1}}}
\expandafter\def\csname PYG@tok@w\endcsname{\def\PYG@tc##1{\textcolor[rgb]{0.73,0.73,0.73}{##1}}}
\expandafter\def\csname PYG@tok@kt\endcsname{\def\PYG@tc##1{\textcolor[rgb]{0.56,0.13,0.00}{##1}}}
\expandafter\def\csname PYG@tok@sc\endcsname{\def\PYG@tc##1{\textcolor[rgb]{0.25,0.44,0.63}{##1}}}
\expandafter\def\csname PYG@tok@sb\endcsname{\def\PYG@tc##1{\textcolor[rgb]{0.25,0.44,0.63}{##1}}}
\expandafter\def\csname PYG@tok@k\endcsname{\let\PYG@bf=\textbf\def\PYG@tc##1{\textcolor[rgb]{0.00,0.44,0.13}{##1}}}
\expandafter\def\csname PYG@tok@se\endcsname{\let\PYG@bf=\textbf\def\PYG@tc##1{\textcolor[rgb]{0.25,0.44,0.63}{##1}}}
\expandafter\def\csname PYG@tok@sd\endcsname{\let\PYG@it=\textit\def\PYG@tc##1{\textcolor[rgb]{0.25,0.44,0.63}{##1}}}

\def\PYGZbs{\char`\\}
\def\PYGZus{\char`\_}
\def\PYGZob{\char`\{}
\def\PYGZcb{\char`\}}
\def\PYGZca{\char`\^}
\def\PYGZam{\char`\&}
\def\PYGZlt{\char`\<}
\def\PYGZgt{\char`\>}
\def\PYGZsh{\char`\#}
\def\PYGZpc{\char`\%}
\def\PYGZdl{\char`\$}
\def\PYGZhy{\char`\-}
\def\PYGZsq{\char`\'}
\def\PYGZdq{\char`\"}
\def\PYGZti{\char`\~}
% for compatibility with earlier versions
\def\PYGZat{@}
\def\PYGZlb{[}
\def\PYGZrb{]}
\makeatother

\renewcommand\PYGZsq{\textquotesingle}

\begin{document}

\maketitle
\tableofcontents
\phantomsection\label{index::doc}


Contents:


\chapter{STANDFIRE User Guide}
\label{user_guide:standfire-user-guide}\label{user_guide::doc}\label{user_guide:welcome-to-standfire}
Contents:


\section{Overview of STANDFIRE}
\label{overview:overview-of-standfire}\label{overview::doc}

\section{Installation}
\label{installation:installation}\label{installation::doc}

\section{Basic Usage}
\label{basic_usage::doc}\label{basic_usage:basic-usage}

\section{Advance Topics}
\label{advance_topics:advance-topics}\label{advance_topics::doc}

\section{Tutorial 1: Interfacing with FVS}
\label{tutorial_1:tutorial-1-interfacing-with-fvs}\label{tutorial_1::doc}
Use Suppose to generate a keyword file. Or use the following example .key

\begin{Verbatim}[commandchars=\\\{\}]
NOSCREEN
RANNSEED           0
!STATS
STDIDENT
STANDFIRE\PYGZus{}example
DESIGN           \PYGZhy{}10       500         5         9
STDINFO          103       140      60.0       0.0       0.0      36.0
INVYEAR         2010
NUMCYCLE          10
TREEDATA
FMIN
END
STATS
SVS                0                   0         0        15
FMIn
Potfire
FuelOut
BurnRept
MortRept
FuelRept
SnagSum
End
PROCESS
STOP
\end{Verbatim}

If don't have a FVS tree list file, then copy and paste the following text and save  it to the same directory where the keyword file lives, give it the same prefix as the \code{.key} but with a \code{.tre} extension.

\begin{Verbatim}[commandchars=\\\{\}]
1   95       9PP 105    35                          0 0
1   96       0PP 43     17        1                 0 0
1   97       0PP 148    43        2                 0 0
1   98       0PP 49     30        1                 0 0
1   99       9PP 54     30                          0 0
1   100      0PP 100    40        3                 0 0
1   101      0PP 42     30        2                 0 0
1   102      0PP 53     34        1                 0 0
1   103      0PP 97     42        3                 0 0
1   104      0PP 61     35        1                 0 0
1   105      0PP 81     40        1                 0 0
1   106      9PP 80     33                          0 0
1   107      0PP 41     32        2                 0 0
1   108      9PP 71     40                          0 0
1   109      9PP 73     41                          0 0
1   110      9PP 94     35                          0 0
1   111      9PP 103    32                          0 0
\end{Verbatim}

Once you have a keyword file and a tree list file in the same directory we can start to build a script to do some work.

First we import the Fvsfuels class from the fuels module.

\begin{Verbatim}[commandchars=\\\{\}]
\PYG{g+gp}{\PYGZgt{}\PYGZgt{}\PYGZgt{} }\PYG{k+kn}{from} \PYG{n+nn}{standfire.fuels} \PYG{k+kn}{import} \PYG{n}{Fvsfuels}
\end{Verbatim}

Next create an instance of the class passes the desired variant as an argument and register the keyword file.

\begin{Verbatim}[commandchars=\\\{\}]
\PYG{g+gp}{\PYGZgt{}\PYGZgt{}\PYGZgt{} }\PYG{n}{stand\PYGZus{}1} \PYG{o}{=} \PYG{n}{Fvsfuel}\PYG{p}{(}\PYG{l+s}{\PYGZdq{}}\PYG{l+s}{iec}\PYG{l+s}{\PYGZdq{}}\PYG{p}{)}
\PYG{g+gp}{\PYGZgt{}\PYGZgt{}\PYGZgt{} }\PYG{n}{stand\PYGZus{}1}\PYG{o}{.}\PYG{n}{set\PYGZus{}keyword}\PYG{p}{(}\PYG{l+s}{\PYGZdq{}}\PYG{l+s}{/Users/standfire/fvs\PYGZus{}exp/example.key}\PYG{l+s}{\PYGZdq{}}\PYG{p}{)}
\PYG{g+go}{TIMEINT not found in keyword file, default is 10 years}
\end{Verbatim}

We get a message telling us that the TIMEINT keyword was not found in the keyword file. No problem, STANDFIRE automatically sets this value to 10 years.

\begin{Verbatim}[commandchars=\\\{\}]
\PYG{g+gp}{\PYGZgt{}\PYGZgt{}\PYGZgt{} }\PYG{n}{stand\PYGZus{}1}\PYG{o}{.}\PYG{n}{keywords}
\PYG{g+go}{\PYGZob{}\PYGZsq{}TIMEINT\PYGZsq{}: 10, \PYGZsq{}NUMCYCLE\PYGZsq{}: 10, \PYGZsq{}INVYEAR\PYGZsq{}: 2010, \PYGZsq{}SVS\PYGZsq{}: 15, \PYGZsq{}FUELOUT\PYGZsq{}: 1\PYGZcb{}}
\end{Verbatim}

Notice the keys in the keywords dictionary.  \code{TIMEINT} is the time interval of the FVS simulation in year, \code{NUMCYCLE} is the number of cycles, \code{INVYEAR} is the year of the inventory, and \code{SVS} and \code{FUELOUT} are there to check if these keywords are in the keyword file. If the \code{SVS} and \code{FUELOUT} keywords are not defined the keyword file then FVS will not calculate tree positions or fuel attributes. So be sure you add these to your keyword file before registering the .key with FVS. You can use \emph{post processors*} in Suppose to do so.  \code{TIMEINT}, \code{NUMCYCLE}, and \code{INVYEAR} can be manually changed by calling setters for each. For instance, if you only want to calculate fuel attributes for trees during the year of the inventory then simply change the \code{NUMCYCLE} value in the keyword dictionary.

\begin{Verbatim}[commandchars=\\\{\}]
\PYG{g+gp}{\PYGZgt{}\PYGZgt{}\PYGZgt{} }\PYG{n}{stand\PYGZus{}1}\PYG{o}{.}\PYG{n}{set\PYGZus{}num\PYGZus{}cycle}\PYG{p}{(}\PYG{l+m+mi}{0}\PYG{p}{)}
\PYG{g+gp}{\PYGZgt{}\PYGZgt{}\PYGZgt{} }\PYG{n}{stand\PYGZus{}1}\PYG{o}{.}\PYG{n}{keywords}
\PYG{g+go}{\PYGZob{}\PYGZsq{}TIMEINT\PYGZsq{}: 10, \PYGZsq{}NUMCYCLE\PYGZsq{}: 0, \PYGZsq{}INVYEAR\PYGZsq{}: 2010, \PYGZsq{}SVS\PYGZsq{}: 15, \PYGZsq{}FUELOUT\PYGZsq{}: 1\PYGZcb{}}
\end{Verbatim}

Now that we have our simulation parameters established, we startup FVS.

\begin{Verbatim}[commandchars=\\\{\}]
\PYG{g+gp}{\PYGZgt{}\PYGZgt{}\PYGZgt{} }\PYG{n}{stand\PYGZus{}1}\PYG{o}{.}\PYG{n}{run\PYGZus{}fvs}\PYG{p}{(}\PYG{p}{)}
\end{Verbatim}


\chapter{STANDFIRE API Reference}
\label{api_ref:standfire-api-reference}\label{api_ref::doc}
Contents:


\section{fuels module}
\label{fuels:module-fuels}\label{fuels::doc}\label{fuels:fuels-module}\index{fuels (module)}
Contents:


\subsection{Fvsfuels}
\label{Fvsfuels::doc}\label{Fvsfuels:fvsfuels}\index{Fvsfuels (class in fuels)}

\begin{fulllineitems}
\phantomsection\label{Fvsfuels:fuels.Fvsfuels}\pysiglinewithargsret{\strong{class }\code{fuels.}\bfcode{Fvsfuels}}{\emph{variant}}{}
Bases: \code{object}

A Fvsfuels object is used to calculate component fuels at the individual
tree level using the Forest Vegetation Simulator. To create an instance
of this class you need two items: a keyword file (.key) and tree list file
(.tre) with the same prefix as the keyword file. If you don't already have
a tree list file then you can use \code{{}`fuels.Inventory{}`} class to generate
one.
\begin{quote}\begin{description}
\item[{Parameters}] \leavevmode
\textbf{\texttt{variant}} (\emph{\texttt{string}}) -- FVS variant to be imported

\end{description}\end{quote}

\textbf{Example:}

A basic example to extract live canopy biomass for individual trees during
year of inventory

\begin{Verbatim}[commandchars=\\\{\}]
\PYG{g+gp}{\PYGZgt{}\PYGZgt{}\PYGZgt{} }\PYG{k+kn}{from} \PYG{n+nn}{standfire.fuels} \PYG{k+kn}{import} \PYG{n}{Fvsfuels}
\PYG{g+gp}{\PYGZgt{}\PYGZgt{}\PYGZgt{} }\PYG{n}{stand001} \PYG{o}{=} \PYG{n}{Fvsfuels}\PYG{p}{(}\PYG{l+s}{\PYGZdq{}}\PYG{l+s}{iec}\PYG{l+s}{\PYGZdq{}}\PYG{p}{)}
\PYG{g+gp}{\PYGZgt{}\PYGZgt{}\PYGZgt{} }\PYG{n}{stand001}\PYG{o}{.}\PYG{n}{set\PYGZus{}keyword}\PYG{p}{(}\PYG{l+s}{\PYGZdq{}}\PYG{l+s}{/Users/standfire/test/example.key}\PYG{l+s}{\PYGZdq{}}\PYG{p}{)}
\PYG{g+go}{TIMEINT not found in keyword file, default is 10 years}
\PYG{g+gp}{\PYGZgt{}\PYGZgt{}\PYGZgt{} }\PYG{n}{stand001}\PYG{o}{.}\PYG{n}{keywords}
\PYG{g+go}{\PYGZob{}\PYGZsq{}TIMEINT\PYGZsq{}: 10, \PYGZsq{}NUMCYCLE\PYGZsq{}: 10, \PYGZsq{}INVYEAR\PYGZsq{}: 2010, \PYGZsq{}SVS\PYGZsq{}: 15, \PYGZsq{}FUELOUT\PYGZsq{}: 1\PYGZcb{}}
\end{Verbatim}

The keyword file is setup to simulate 100 years at a time interval of 10
years. Lets change this to only simulate the inventory year.

\begin{Verbatim}[commandchars=\\\{\}]
\PYG{g+gp}{\PYGZgt{}\PYGZgt{}\PYGZgt{} }\PYG{n}{stand001}\PYG{o}{.}\PYG{n}{set\PYGZus{}num\PYGZus{}cycles}\PYG{p}{(}\PYG{l+m+mi}{0}\PYG{p}{)}
\PYG{g+gp}{\PYGZgt{}\PYGZgt{}\PYGZgt{} }\PYG{n}{stand001}\PYG{o}{.}\PYG{n}{keywords}
\PYG{g+go}{\PYGZob{}\PYGZsq{}TIMEINT\PYGZsq{}: 10, \PYGZsq{}NUMCYCLE\PYGZsq{}: 0, \PYGZsq{}INVYEAR\PYGZsq{}: 2010, \PYGZsq{}SVS\PYGZsq{}: 15, \PYGZsq{}FUELOUT\PYGZsq{}: 1\PYGZcb{}}
\PYG{g+gp}{\PYGZgt{}\PYGZgt{}\PYGZgt{} }\PYG{n}{stand001}\PYG{o}{.}\PYG{n}{run\PYGZus{}fvs}\PYG{p}{(}\PYG{p}{)}
\end{Verbatim}

Now we can write the trees data frame to disk

\begin{Verbatim}[commandchars=\\\{\}]
\PYG{g+gp}{\PYGZgt{}\PYGZgt{}\PYGZgt{} }\PYG{n}{stand001}\PYG{o}{.}\PYG{n}{save\PYGZus{}trees\PYGZus{}by\PYGZus{}year}\PYG{p}{(}\PYG{l+m+mi}{2010}\PYG{p}{)}
\end{Verbatim}

\begin{notice}{note}{Note:}
The argument must match one of the available variant in the
PyFVS module. Search through standfire/pyfvs/ to see all
variants
\end{notice}
\index{get\_simulation\_years() (fuels.Fvsfuels method)}

\begin{fulllineitems}
\phantomsection\label{Fvsfuels:fuels.Fvsfuels.get_simulation_years}\pysiglinewithargsret{\bfcode{get\_simulation\_years}}{}{}
Returns a list of the simulated years
\begin{quote}\begin{description}
\item[{Returns}] \leavevmode
simulated year

\item[{Return type}] \leavevmode
list of integers

\end{description}\end{quote}

\end{fulllineitems}

\index{get\_snags() (fuels.Fvsfuels method)}

\begin{fulllineitems}
\phantomsection\label{Fvsfuels:fuels.Fvsfuels.get_snags}\pysiglinewithargsret{\bfcode{get\_snags}}{\emph{year}}{}
Returns pandas data fram of the snags by indexed year
\begin{quote}\begin{description}
\item[{Parameters}] \leavevmode
\textbf{\texttt{year}} (\emph{\texttt{int}}) -- simulation year of the data frame to return

\item[{Returns}] \leavevmode
data frame of snags at indexed year

\item[{Return type}] \leavevmode
pandas dataframe

\end{description}\end{quote}

\begin{notice}{note}{Note:}
If a data frame for the specified year does not exist then
a message will be printed to the console.
\end{notice}

\end{fulllineitems}

\index{get\_standid() (fuels.Fvsfuels method)}

\begin{fulllineitems}
\phantomsection\label{Fvsfuels:fuels.Fvsfuels.get_standid}\pysiglinewithargsret{\bfcode{get\_standid}}{}{}
Returns stand ID as defined in the keyword file of the class instance
\begin{quote}\begin{description}
\item[{Returns}] \leavevmode
stand ID value

\item[{Return type}] \leavevmode
string

\end{description}\end{quote}

\end{fulllineitems}

\index{get\_trees() (fuels.Fvsfuels method)}

\begin{fulllineitems}
\phantomsection\label{Fvsfuels:fuels.Fvsfuels.get_trees}\pysiglinewithargsret{\bfcode{get\_trees}}{\emph{year}}{}
Returns pandas data fram of the trees by indexed year
\begin{quote}\begin{description}
\item[{Parameters}] \leavevmode
\textbf{\texttt{year}} (\emph{\texttt{int}}) -- simulation year of the data frame to return

\item[{Returns}] \leavevmode
data frame of trees at indexed year

\item[{Return type}] \leavevmode
pandas dataframe

\end{description}\end{quote}

\begin{notice}{note}{Note:}
If a data frame for the specified year does not exist then
a message will be printed to the console.
\end{notice}

\end{fulllineitems}

\index{run\_fvs() (fuels.Fvsfuels method)}

\begin{fulllineitems}
\phantomsection\label{Fvsfuels:fuels.Fvsfuels.run_fvs}\pysiglinewithargsret{\bfcode{run\_fvs}}{}{}
Runs the FVS simulation

This method run a FVS simulation using the previously specified keyword
file. The simulation will be paused at each time interval and the trees
and snag data collected and appended to the fuels attribute of the
Fvsfuels object.

\textbf{Example:}

\begin{Verbatim}[commandchars=\\\{\}]
\PYG{g+gp}{\PYGZgt{}\PYGZgt{}\PYGZgt{} }\PYG{k+kn}{from} \PYG{n+nn}{standfire.fuels} \PYG{k+kn}{import} \PYG{n}{Fvsfuels}
\PYG{g+gp}{\PYGZgt{}\PYGZgt{}\PYGZgt{} }\PYG{n}{stand010} \PYG{o}{=} \PYG{n}{Fvsfuels}\PYG{p}{(}\PYG{l+s}{\PYGZdq{}}\PYG{l+s}{iec}\PYG{l+s}{\PYGZdq{}}\PYG{p}{)}
\PYG{g+gp}{\PYGZgt{}\PYGZgt{}\PYGZgt{} }\PYG{n}{stand010}\PYG{o}{.}\PYG{n}{set\PYGZus{}keyword}\PYG{p}{(}\PYG{l+s}{\PYGZdq{}}\PYG{l+s}{/Users/standfire/example/test.key}\PYG{l+s}{\PYGZdq{}}\PYG{p}{)}
\PYG{g+gp}{\PYGZgt{}\PYGZgt{}\PYGZgt{} }\PYG{n}{stand010}\PYG{o}{.}\PYG{n}{run\PYGZus{}fvs}\PYG{p}{(}\PYG{p}{)}
\PYG{g+gp}{\PYGZgt{}\PYGZgt{}\PYGZgt{} }\PYG{n}{stand010}\PYG{o}{.}\PYG{n}{fuels}\PYG{p}{[}\PYG{l+s}{\PYGZdq{}}\PYG{l+s}{trees}\PYG{l+s}{\PYGZdq{}}\PYG{p}{]}\PYG{p}{[}\PYG{l+m+mi}{2010}\PYG{p}{]}
\PYG{g+go}{xloc    yloc    species   dbh     ht    crd    cratio  crownwt0  crownwt1 ...}
\PYG{g+go}{33.49  108.58   PIPO     19.43   68.31  8.77     25    33.46      4.3}
\PYG{g+go}{24.3    90.4    PIPO     11.46   56.6   5.63     15     6.55     2.33}
\PYG{g+go}{88.84  162.98   PIPO     18.63   67.76  9.48     45    75.88     6.89}
\PYG{g+gp}{...}
\end{Verbatim}

\end{fulllineitems}

\index{save\_all() (fuels.Fvsfuels method)}

\begin{fulllineitems}
\phantomsection\label{Fvsfuels:fuels.Fvsfuels.save_all}\pysiglinewithargsret{\bfcode{save\_all}}{}{}
Writes all data frame in the \code{fuels} attribute of the class to the
specified working directory. Output file are .csv.

\end{fulllineitems}

\index{save\_snags\_by\_year() (fuels.Fvsfuels method)}

\begin{fulllineitems}
\phantomsection\label{Fvsfuels:fuels.Fvsfuels.save_snags_by_year}\pysiglinewithargsret{\bfcode{save\_snags\_by\_year}}{\emph{year}}{}
Writes snag data frame at indexed year to .csv in working directory

\end{fulllineitems}

\index{save\_trees\_by\_year() (fuels.Fvsfuels method)}

\begin{fulllineitems}
\phantomsection\label{Fvsfuels:fuels.Fvsfuels.save_trees_by_year}\pysiglinewithargsret{\bfcode{save\_trees\_by\_year}}{\emph{year}}{}
Writes tree data frame at indexed year to .csv in working directory

\end{fulllineitems}

\index{set\_dir() (fuels.Fvsfuels method)}

\begin{fulllineitems}
\phantomsection\label{Fvsfuels:fuels.Fvsfuels.set_dir}\pysiglinewithargsret{\bfcode{set\_dir}}{\emph{wdir}}{}
Sets the working directory of a Fvsfuels object

This method is called by \code{Fvsfuels.set\_keyword()}. Thus, the default
working directory is the folder containing the specified keyword file.
If you wish to store simulation outputs in a different directory then
use this methods to do so.
\begin{quote}\begin{description}
\item[{Parameters}] \leavevmode
\textbf{\texttt{wdir}} (\emph{\texttt{string}}) -- path/to/desired\_directory

\item[{Example}] \leavevmode
\end{description}\end{quote}

\begin{Verbatim}[commandchars=\\\{\}]
\PYG{g+gp}{\PYGZgt{}\PYGZgt{}\PYGZgt{} }\PYG{k+kn}{from} \PYG{n+nn}{standfire.fuel} \PYG{k+kn}{import} \PYG{n}{Fvsfuels}
\PYG{g+gp}{\PYGZgt{}\PYGZgt{}\PYGZgt{} }\PYG{n}{test} \PYG{o}{=} \PYG{n}{Fvsfuels}\PYG{p}{(}\PYG{l+s}{\PYGZdq{}}\PYG{l+s}{emc}\PYG{l+s}{\PYGZdq{}}\PYG{p}{)}
\PYG{g+gp}{\PYGZgt{}\PYGZgt{}\PYGZgt{} }\PYG{n}{test}\PYG{o}{.}\PYG{n}{set\PYGZus{}keyword}\PYG{p}{(}\PYG{l+s}{\PYGZdq{}}\PYG{l+s}{/Users/standfire/test/example.key}\PYG{l+s}{\PYGZdq{}}\PYG{p}{)}
\end{Verbatim}

Whoops, I would like to store simulation outputs elsewhere...

\begin{Verbatim}[commandchars=\\\{\}]
\PYG{g+gp}{\PYGZgt{}\PYGZgt{}\PYGZgt{} }\PYG{n}{test}\PYG{o}{.}\PYG{n}{set\PYGZus{}dir}\PYG{p}{(}\PYG{l+s}{\PYGZdq{}}\PYG{l+s}{/Users/standfire/outputs/}\PYG{l+s}{\PYGZdq{}}\PYG{p}{)}
\end{Verbatim}

\end{fulllineitems}

\index{set\_inv\_year() (fuels.Fvsfuels method)}

\begin{fulllineitems}
\phantomsection\label{Fvsfuels:fuels.Fvsfuels.set_inv_year}\pysiglinewithargsret{\bfcode{set\_inv\_year}}{\emph{inv\_year}}{}
Sets inventory year for FVS simulation
\begin{quote}\begin{description}
\item[{Parameters}] \leavevmode
\textbf{\texttt{inv\_year}} (\emph{\texttt{int}}) -- year of the inventory

\end{description}\end{quote}

\end{fulllineitems}

\index{set\_keyword() (fuels.Fvsfuels method)}

\begin{fulllineitems}
\phantomsection\label{Fvsfuels:fuels.Fvsfuels.set_keyword}\pysiglinewithargsret{\bfcode{set\_keyword}}{\emph{keyfile}}{}
Sets the keyword file to be used in the FVS simulation
\begin{quote}\begin{description}
\item[{Date}] \leavevmode
2015-8-12

\item[{Authors}] \leavevmode
Lucas Wells

\end{description}\end{quote}

This method will initalize a FVS simulation by registering the
specified keyword file (.key) with FVS. The working directory of a
Fvsfuels object will be set to the folder containing the keyword file.
You can manually change the working directory with Fvsfuels.set\_dir().
This function will also call private methods in this class to extract
information from the keyword file and set class fields accordingly for
use in other methods.
\begin{quote}\begin{description}
\item[{Parameters}] \leavevmode
\textbf{\texttt{keyfile}} (\emph{\texttt{string}}) -- path/to/keyword\_file. This must have a .key extension

\end{description}\end{quote}

\textbf{Example:}

\begin{Verbatim}[commandchars=\\\{\}]
\PYG{g+gp}{\PYGZgt{}\PYGZgt{}\PYGZgt{} }\PYG{k+kn}{from} \PYG{n+nn}{standfire.fuels} \PYG{k+kn}{import} \PYG{n}{Fvsfuels}
\PYG{g+gp}{\PYGZgt{}\PYGZgt{}\PYGZgt{} }\PYG{n}{test} \PYG{o}{=} \PYG{n}{Fvsfuels}\PYG{p}{(}\PYG{l+s}{\PYGZdq{}}\PYG{l+s}{iec}\PYG{l+s}{\PYGZdq{}}\PYG{p}{)}
\PYG{g+gp}{\PYGZgt{}\PYGZgt{}\PYGZgt{} }\PYG{n}{test}\PYG{o}{.}\PYG{n}{set\PYGZus{}keyword}\PYG{p}{(}\PYG{l+s}{\PYGZdq{}}\PYG{l+s}{/Users/standfire/test/example.key}\PYG{l+s}{\PYGZdq{}}\PYG{p}{)}
\end{Verbatim}

\end{fulllineitems}

\index{set\_num\_cycles() (fuels.Fvsfuels method)}

\begin{fulllineitems}
\phantomsection\label{Fvsfuels:fuels.Fvsfuels.set_num_cycles}\pysiglinewithargsret{\bfcode{set\_num\_cycles}}{\emph{num\_cyc}}{}
Sets number of cycles for FVS simulation
\begin{quote}\begin{description}
\item[{Parameters}] \leavevmode
\textbf{\texttt{num\_cyc}} (\emph{\texttt{int}}) -- number of simulation cycles

\end{description}\end{quote}

\end{fulllineitems}

\index{set\_time\_int() (fuels.Fvsfuels method)}

\begin{fulllineitems}
\phantomsection\label{Fvsfuels:fuels.Fvsfuels.set_time_int}\pysiglinewithargsret{\bfcode{set\_time\_int}}{\emph{time\_int}}{}
Sets time interval for FVS simulation
\begin{quote}\begin{description}
\item[{Parameters}] \leavevmode
\textbf{\texttt{time\_int}} (\emph{\texttt{int}}) -- length of simulation time step

\end{description}\end{quote}

\end{fulllineitems}


\end{fulllineitems}



\subsection{Inventory}
\label{Inventory::doc}\label{Inventory:inventory}\index{Inventory (class in fuels)}

\begin{fulllineitems}
\phantomsection\label{Inventory:fuels.Inventory}\pysigline{\strong{class }\code{fuels.}\bfcode{Inventory}}
Bases: \code{object}

This class contains methods for converting inventory data to FVS .tre
format

This class currently does not read inventory data from an FVS access
database.  The FVS\_TreeInit database first needs to be exported as comma
delimited values. Multiple stands can be exported in the same file, the
\code{formatFvsTreeFile()} function will format a .tre string for each stand.
All column headings must be default headings and unaltered during export.
You can view the default format by importing this class and typing \code{FMT}.
See the FVS guide \footnote[1]{
Gary E. Dixon, Essential FVS: A User's Guide to the Forest
Vegetation Simulator Tech. Rep., U.S. Department of Agriculture , Forest
Service, Forest Management Service Center, Fort Collins, Colo, USA, 2003.
} for more information regarding the format of .tre
files.

\textbf{Example:}

\begin{Verbatim}[commandchars=\\\{\}]
\PYG{g+gp}{\PYGZgt{}\PYGZgt{}\PYGZgt{} }\PYG{k+kn}{from} \PYG{n+nn}{standfire} \PYG{k+kn}{import} \PYG{n}{fuels}
\PYG{g+gp}{\PYGZgt{}\PYGZgt{}\PYGZgt{} }\PYG{n}{toDotTree} \PYG{o}{=} \PYG{n}{fuels}\PYG{o}{.}\PYG{n}{Inventory}\PYG{p}{(}\PYG{p}{)}
\PYG{g+gp}{\PYGZgt{}\PYGZgt{}\PYGZgt{} }\PYG{n}{toDotTree}\PYG{o}{.}\PYG{n}{read\PYGZus{}inventory}\PYG{p}{(}\PYG{l+s}{\PYGZdq{}}\PYG{l+s}{path/to/FVS\PYGZus{}TreeInit.csv}\PYG{l+s}{\PYGZdq{}}\PYG{p}{)}
\PYG{g+gp}{\PYGZgt{}\PYGZgt{}\PYGZgt{} }\PYG{n}{toDotTree}\PYG{o}{.}\PYG{n}{format\PYGZus{}fvs\PYGZus{}tree\PYGZus{}file}\PYG{p}{(}\PYG{p}{)}
\PYG{g+gp}{\PYGZgt{}\PYGZgt{}\PYGZgt{} }\PYG{n}{toDotTree}\PYG{o}{.}\PYG{n}{save}\PYG{p}{(}\PYG{p}{)}
\end{Verbatim}
\begin{quote}\begin{description}
\item[{References}] \leavevmode
\end{description}\end{quote}
\index{convert\_sp\_codes() (fuels.Inventory method)}

\begin{fulllineitems}
\phantomsection\label{Inventory:fuels.Inventory.convert_sp_codes}\pysiglinewithargsret{\bfcode{convert\_sp\_codes}}{\emph{method=`2to4'}}{}
Converts species codes from 4 letter codes to 2 letter codes
or vise versa
\begin{quote}\begin{description}
\item[{Parameters}] \leavevmode
\textbf{\texttt{method}} (\emph{\texttt{string}}) -- must be either ``2to4'' or ``4to2''

\end{description}\end{quote}

\end{fulllineitems}

\index{crwratio\_percent\_to\_code() (fuels.Inventory method)}

\begin{fulllineitems}
\phantomsection\label{Inventory:fuels.Inventory.crwratio_percent_to_code}\pysiglinewithargsret{\bfcode{crwratio\_percent\_to\_code}}{}{}
Converts crown ratio from percent to ICR code

ICR code is described in the Essential FVS Guide on pages 58 and 59.
This method should only be used if crown ratios values are percentages
in the FVS\_TreeInit.csv.  If you use this method before calling
\emph{formatFvsTreeFile()} then you must set the optional argument
\code{cratioToCode} to \code{False}.

\end{fulllineitems}

\index{filter\_by\_stand() (fuels.Inventory method)}

\begin{fulllineitems}
\phantomsection\label{Inventory:fuels.Inventory.filter_by_stand}\pysiglinewithargsret{\bfcode{filter\_by\_stand}}{\emph{stand\_list}}{}
Filters data by a list of stand IDs
\begin{quote}\begin{description}
\item[{Parameters}] \leavevmode
\textbf{\texttt{stand\_list}} (\emph{\texttt{python list}}) -- List of stand ID to retain in the data. All other
stands will be removed

\end{description}\end{quote}

\end{fulllineitems}

\index{format\_fvs\_tree\_file() (fuels.Inventory method)}

\begin{fulllineitems}
\phantomsection\label{Inventory:fuels.Inventory.format_fvs_tree_file}\pysiglinewithargsret{\bfcode{format\_fvs\_tree\_file}}{\emph{cratio\_to\_code=True}}{}
Converts data in FVS\_TreeInit.csv to FVS .tre format

This methods reads entries in the pandas data frame (\code{self.data}) and
writes them to a formated text string following FVS .tre data formating
standards shown in \code{FMT}.  If multiple stands exist in \code{self.data}
then each stand will written as a (key,value) pair in
\code{self.fvsTreeFile} where the key is the stand ID and the value is the
formated text string.
\begin{quote}\begin{description}
\item[{Parameters}] \leavevmode
\textbf{\texttt{cratio\_to\_code}} (\emph{\texttt{boolean}}) -- default = True

\end{description}\end{quote}

\begin{notice}{note}{Note:}
If the \code{crwratio\_percent\_to\_code()} methods has
been called prior to call this methods, then the \code{cratio\_to\_code}
optional argument must be set to \code{False} to prevent errors in crown
ratio values.
\end{notice}

\textbf{Example:}

\begin{Verbatim}[commandchars=\\\{\}]
\PYG{g+gp}{\PYGZgt{}\PYGZgt{}\PYGZgt{} }\PYG{n}{toDotTree}\PYG{o}{.}\PYG{n}{format\PYGZus{}fvs\PYGZus{}tree\PYGZus{}file}\PYG{p}{(}\PYG{p}{)}
\PYG{g+gp}{\PYGZgt{}\PYGZgt{}\PYGZgt{} }\PYG{n}{toDotTree}\PYG{o}{.}\PYG{n}{fvsTreeFile}\PYG{p}{[}\PYG{l+s}{\PYGZsq{}}\PYG{l+s}{Stand\PYGZus{}ID\PYGZus{}1}\PYG{l+s}{\PYGZsq{}}\PYG{p}{]}
\PYG{g+go}{5   1  5     0PP 189    65        3                 0 0}
\PYG{g+go}{5   2  15    0PP 110    52        2                 0 0}
\PYG{g+go}{5   3  5     0PP 180    64        5                 0 0}
\PYG{g+go}{5   4  14    0PP 112    56        3                 0 0}
\PYG{g+go}{5   5  6     0PP 167    60        4                 0 0}
\PYG{g+go}{5   6  5     0PP 190    60        5                 0 0}
\PYG{g+go}{5   7  7     0PP 161    62        3                 0 0}
\PYG{g+go}{5   8  86    0PP 46     37        1                 0 0}
\PYG{g+go}{5   9  10    0PP 130    50        2                 0 0}
\PYG{g+go}{5   10 5     0PP 182    60        3                 0 0}
\PYG{g+go}{5   11 8     9PP 144    50                          0 0}
\PYG{g+go}{6   1  16    0PP 107    42        4                 0 0}
\PYG{g+go}{6   2  109   0PP 41     27        2                 0 0}
\PYG{g+gp}{...}
\end{Verbatim}

\end{fulllineitems}

\index{get\_fvs\_cols() (fuels.Inventory method)}

\begin{fulllineitems}
\phantomsection\label{Inventory:fuels.Inventory.get_fvs_cols}\pysiglinewithargsret{\bfcode{get\_fvs\_cols}}{}{}
Get list of FVS standard columns
\begin{quote}\begin{description}
\item[{Returns}] \leavevmode
FVS standard columns

\item[{Return type}] \leavevmode
list of strings

\end{description}\end{quote}

\end{fulllineitems}

\index{get\_stands() (fuels.Inventory method)}

\begin{fulllineitems}
\phantomsection\label{Inventory:fuels.Inventory.get_stands}\pysiglinewithargsret{\bfcode{get\_stands}}{}{}
Returns unique stand IDs
\begin{quote}\begin{description}
\item[{Returns}] \leavevmode
stand IDs

\item[{Return type}] \leavevmode
list of strings

\end{description}\end{quote}

\textbf{Example:}

\begin{Verbatim}[commandchars=\\\{\}]
\PYG{g+gp}{\PYGZgt{}\PYGZgt{}\PYGZgt{} }\PYG{n}{toDotTree}\PYG{o}{.}\PYG{n}{get\PYGZus{}stands}\PYG{p}{(}\PYG{p}{)}
\PYG{g+go}{[\PYGZsq{}BR\PYGZsq{}, \PYGZsq{}TM\PYGZsq{}, \PYGZsq{}SW\PYGZsq{}, HB\PYGZsq{}]}
\end{Verbatim}

\end{fulllineitems}

\index{print\_format\_standards() (fuels.Inventory method)}

\begin{fulllineitems}
\phantomsection\label{Inventory:fuels.Inventory.print_format_standards}\pysiglinewithargsret{\bfcode{print\_format\_standards}}{}{}
Print FVS formating standards

The FVS formating standard for .tre files as described in the Essenital
FVS Guide is stored in \code{FMT} as a class attribute.  This method
is for viewing this format.  The keys of the dictionary are the column
headings and values are as follows: 0 = variable name, 1 = variable
type, 2 = column location, 3 = units, and 4 = implied decimal place.

\textbf{Example:}

\begin{Verbatim}[commandchars=\\\{\}]
\PYG{g+gp}{\PYGZgt{}\PYGZgt{}\PYGZgt{} }\PYG{n}{toDotTree}\PYG{o}{.}\PYG{n}{print\PYGZus{}format\PYGZus{}standards}\PYG{p}{(}\PYG{p}{)}
\PYG{g+go}{\PYGZob{}\PYGZsq{}Plot\PYGZus{}ID\PYGZsq{}       : [\PYGZsq{}ITRE\PYGZsq{},      \PYGZsq{}integer\PYGZsq{},  [0,3],   None,      None],}
\PYG{g+go}{ \PYGZsq{}Tree\PYGZus{}ID\PYGZsq{}       : [\PYGZsq{}IDTREE2\PYGZsq{},   \PYGZsq{}integer\PYGZsq{},  [4,6],   None,      None],}
\PYG{g+go}{ \PYGZsq{}Tree\PYGZus{}Count\PYGZsq{}    : [\PYGZsq{}PROB\PYGZsq{},      \PYGZsq{}integer\PYGZsq{},  [7,12],  None,      None],}
\PYG{g+go}{ \PYGZsq{}History\PYGZsq{}       : [\PYGZsq{}ITH\PYGZsq{},       \PYGZsq{}integer\PYGZsq{},  [13,13], \PYGZsq{}trees\PYGZsq{},   0   ],}
\PYG{g+go}{ \PYGZsq{}Species\PYGZsq{}       : [\PYGZsq{}ISP\PYGZsq{},       \PYGZsq{}alphanum\PYGZsq{}, [14,16], None,      None],}
\PYG{g+go}{ \PYGZsq{}DBH\PYGZsq{}           : [\PYGZsq{}DBH\PYGZsq{},       \PYGZsq{}real\PYGZsq{},     [17,20], \PYGZsq{}inches\PYGZsq{},  1   ],}
\PYG{g+go}{ \PYGZsq{}DG\PYGZsq{}            : [\PYGZsq{}DG\PYGZsq{},        \PYGZsq{}real\PYGZsq{},     [21,23], \PYGZsq{}inches\PYGZsq{},  1   ],}
\PYG{g+go}{ \PYGZsq{}Ht\PYGZsq{}            : [\PYGZsq{}HT\PYGZsq{},        \PYGZsq{}real\PYGZsq{},     [24,26], \PYGZsq{}feet\PYGZsq{},    0   ],}
\PYG{g+go}{ \PYGZsq{}HtTopK\PYGZsq{}        : [\PYGZsq{}THT\PYGZsq{},       \PYGZsq{}real\PYGZsq{},     [27,29], \PYGZsq{}feet\PYGZsq{},    0   ],}
\PYG{g+go}{ \PYGZsq{}HTG\PYGZsq{}           : [\PYGZsq{}HTG\PYGZsq{},       \PYGZsq{}real\PYGZsq{},     [30,33], \PYGZsq{}feet\PYGZsq{},    1   ],}
\PYG{g+go}{ \PYGZsq{}CrRatio\PYGZsq{}       : [\PYGZsq{}ICR\PYGZsq{},       \PYGZsq{}integer\PYGZsq{},  [34,34], None,      None],}
\PYG{g+go}{ \PYGZsq{}Damage1\PYGZsq{}       : [\PYGZsq{}IDCD(1)\PYGZsq{},   \PYGZsq{}integer\PYGZsq{},  [35,36], None,      None],}
\PYG{g+go}{ \PYGZsq{}Severity1\PYGZsq{}     : [\PYGZsq{}IDCD(2)\PYGZsq{},   \PYGZsq{}integer\PYGZsq{},  [37,38], None,      None],}
\PYG{g+go}{ \PYGZsq{}Damage2\PYGZsq{}       : [\PYGZsq{}IDCD(3)\PYGZsq{},   \PYGZsq{}integer\PYGZsq{},  [39,40], None,      None],}
\PYG{g+go}{ \PYGZsq{}Severity2\PYGZsq{}     : [\PYGZsq{}IDCD(4)\PYGZsq{},   \PYGZsq{}integer\PYGZsq{},  [41,42], None,      None],}
\PYG{g+go}{ \PYGZsq{}Damage3\PYGZsq{}       : [\PYGZsq{}IDCD(5)\PYGZsq{},   \PYGZsq{}integer\PYGZsq{},  [43,44], None,      None],}
\PYG{g+go}{ \PYGZsq{}Severity3\PYGZsq{}     : [\PYGZsq{}IDCD(6)\PYGZsq{},   \PYGZsq{}integer\PYGZsq{},  [45,46], None,      None],}
\PYG{g+go}{ \PYGZsq{}TreeValue\PYGZsq{}     : [\PYGZsq{}IMC\PYGZsq{},       \PYGZsq{}integer\PYGZsq{},  [47,47], None,      None],}
\PYG{g+go}{ \PYGZsq{}Prescription\PYGZsq{}  : [\PYGZsq{}IPRSC\PYGZsq{},     \PYGZsq{}integer\PYGZsq{},  [48,48], None,      None],}
\PYG{g+go}{ \PYGZsq{}Slope\PYGZsq{}         : [\PYGZsq{}IPVARS(1)\PYGZsq{}, \PYGZsq{}integer\PYGZsq{},  [49,50], \PYGZsq{}percent\PYGZsq{}, None],}
\PYG{g+go}{ \PYGZsq{}Aspect\PYGZsq{}        : [\PYGZsq{}IPVARS(2)\PYGZsq{}, \PYGZsq{}integer\PYGZsq{},  [51,53], \PYGZsq{}code\PYGZsq{},    None],}
\PYG{g+go}{ \PYGZsq{}PV\PYGZus{}Code\PYGZsq{}       : [\PYGZsq{}IPVARS(3)\PYGZsq{}, \PYGZsq{}integer\PYGZsq{},  [54,56], \PYGZsq{}code\PYGZsq{},    None],}
\PYG{g+go}{ \PYGZsq{}TopoCode\PYGZsq{}      : [\PYGZsq{}IPVARS(4)\PYGZsq{}, \PYGZsq{}integer\PYGZsq{},  [57,59], \PYGZsq{}code\PYGZsq{},    None],}
\PYG{g+go}{ \PYGZsq{}SitePrep\PYGZsq{}      : [\PYGZsq{}IPVARS(5)\PYGZsq{}, \PYGZsq{}integer\PYGZsq{},  [58,58], \PYGZsq{}code\PYGZsq{},    None],}
\PYG{g+go}{ \PYGZsq{}Age\PYGZsq{}           : [\PYGZsq{}ABIRTH\PYGZsq{},    \PYGZsq{}real\PYGZsq{},     [59,61], \PYGZsq{}years\PYGZsq{},   0   ]\PYGZcb{}}
\end{Verbatim}

See page 61 and 62 in the Essential FVS Guide.

\end{fulllineitems}

\index{read\_inventory() (fuels.Inventory method)}

\begin{fulllineitems}
\phantomsection\label{Inventory:fuels.Inventory.read_inventory}\pysiglinewithargsret{\bfcode{read\_inventory}}{\emph{fname}}{}
Reads a .csv file containing tree records.

The csv must be in the correct format as described in \code{FMT}.  This
method check the format of the file by calling a private method
\code{\_is\_correct\_format()} that raises a value error.
\begin{quote}\begin{description}
\item[{Parameters}] \leavevmode
\textbf{\texttt{fname}} (\emph{\texttt{string}}) -- path to and file name of the Fvs\_TreeInit.csv file

\end{description}\end{quote}

\textbf{Example:}

\begin{Verbatim}[commandchars=\\\{\}]
\PYG{g+gp}{\PYGZgt{}\PYGZgt{}\PYGZgt{} }\PYG{k+kn}{from} \PYG{n+nn}{standfire} \PYG{k+kn}{import} \PYG{n}{fuels}
\PYG{g+gp}{\PYGZgt{}\PYGZgt{}\PYGZgt{} }\PYG{n}{toDotTree} \PYG{o}{=} \PYG{n}{fuels}\PYG{o}{.}\PYG{n}{Inventory}\PYG{p}{(}\PYG{p}{)}
\PYG{g+gp}{\PYGZgt{}\PYGZgt{}\PYGZgt{} }\PYG{n}{toDotTree}\PYG{o}{.}\PYG{n}{readInventory}\PYG{p}{(}\PYG{l+s}{\PYGZdq{}}\PYG{l+s}{path/to/FVS\PYGZus{}TreeInit.csv}\PYG{l+s}{\PYGZdq{}}\PYG{p}{)}
\PYG{g+gp}{\PYGZgt{}\PYGZgt{}\PYGZgt{} }\PYG{n}{np}\PYG{o}{.}\PYG{n}{mean}\PYG{p}{(}\PYG{n}{toDotTree}\PYG{o}{.}\PYG{n}{data}\PYG{p}{[}\PYG{l+s}{\PYGZsq{}}\PYG{l+s}{DBH}\PYG{l+s}{\PYGZsq{}}\PYG{p}{]}\PYG{p}{)}
\PYG{g+go}{9.0028318584070828}
\end{Verbatim}

The \code{read\_inventory()} method stores the data in a pandas data frame.
There are countless operations that can be performed on these objects.
For example, we can explore the relationship between diameter and
height by fitting a linear model

\begin{Verbatim}[commandchars=\\\{\}]
\PYG{g+gp}{\PYGZgt{}\PYGZgt{}\PYGZgt{} }\PYG{k+kn}{import} \PYG{n+nn}{statsmodels.formula.api} \PYG{k+kn}{as} \PYG{n+nn}{sm}
\PYG{g+gp}{\PYGZgt{}\PYGZgt{}\PYGZgt{} }\PYG{n}{fit} \PYG{o}{=} \PYG{n}{sm}\PYG{o}{.}\PYG{n}{ols}\PYG{p}{(}\PYG{n}{formula}\PYG{o}{=}\PYG{l+s}{\PYGZdq{}}\PYG{l+s}{HT \PYGZti{} DBH}\PYG{l+s}{\PYGZdq{}}\PYG{p}{,} \PYG{n}{data}\PYG{o}{=}\PYG{n}{test}\PYG{o}{.}\PYG{n}{data}\PYG{p}{)}\PYG{o}{.}\PYG{n}{fit}\PYG{p}{(}\PYG{p}{)}
\PYG{g+gp}{\PYGZgt{}\PYGZgt{}\PYGZgt{} }\PYG{k}{print} \PYG{n}{fit}\PYG{o}{.}\PYG{n}{params}
\PYG{g+go}{Intercept    19.688167}
\PYG{g+go}{DBH           2.161420}
\PYG{g+go}{dtype: float64}
\PYG{g+gp}{\PYGZgt{}\PYGZgt{}\PYGZgt{} }\PYG{k}{print} \PYG{n}{fit}\PYG{o}{.}\PYG{n}{summary}\PYG{p}{(}\PYG{p}{)}
\PYG{g+go}{OLS Regression Results}
\PYG{g+go}{==============================================================================}
\PYG{g+go}{Dep. Variable:                     Ht   R\PYGZhy{}squared:                       0.738}
\PYG{g+go}{Model:                            OLS   Adj. R\PYGZhy{}squared:                  0.736}
\PYG{g+go}{Method:                 Least Squares   F\PYGZhy{}statistic:                     351.8}
\PYG{g+go}{Date:                Tue, 07 Jul 2015   Prob (F\PYGZhy{}statistic):           3.77e\PYGZhy{}38}
\PYG{g+go}{Time:                        08:32:02   Log\PYGZhy{}Likelihood:                \PYGZhy{}407.10}
\PYG{g+go}{No. Observations:                 127   AIC:                             818.2}
\PYG{g+go}{Df Residuals:                     125   BIC:                             823.9}
\PYG{g+go}{Df Model:                           1}
\PYG{g+go}{Covariance Type:            nonrobust}
\PYG{g+go}{==============================================================================}
\PYG{g+go}{coef    std err          t      P\PYGZgt{}\textbar{}t\textbar{}      [95.0\PYGZpc{} Conf. Int.]}
\PYG{g+go}{\PYGZhy{}\PYGZhy{}\PYGZhy{}\PYGZhy{}\PYGZhy{}\PYGZhy{}\PYGZhy{}\PYGZhy{}\PYGZhy{}\PYGZhy{}\PYGZhy{}\PYGZhy{}\PYGZhy{}\PYGZhy{}\PYGZhy{}\PYGZhy{}\PYGZhy{}\PYGZhy{}\PYGZhy{}\PYGZhy{}\PYGZhy{}\PYGZhy{}\PYGZhy{}\PYGZhy{}\PYGZhy{}\PYGZhy{}\PYGZhy{}\PYGZhy{}\PYGZhy{}\PYGZhy{}\PYGZhy{}\PYGZhy{}\PYGZhy{}\PYGZhy{}\PYGZhy{}\PYGZhy{}\PYGZhy{}\PYGZhy{}\PYGZhy{}\PYGZhy{}\PYGZhy{}\PYGZhy{}\PYGZhy{}\PYGZhy{}\PYGZhy{}\PYGZhy{}\PYGZhy{}\PYGZhy{}\PYGZhy{}\PYGZhy{}\PYGZhy{}\PYGZhy{}\PYGZhy{}\PYGZhy{}\PYGZhy{}\PYGZhy{}\PYGZhy{}\PYGZhy{}\PYGZhy{}\PYGZhy{}\PYGZhy{}\PYGZhy{}\PYGZhy{}\PYGZhy{}\PYGZhy{}\PYGZhy{}\PYGZhy{}\PYGZhy{}\PYGZhy{}\PYGZhy{}\PYGZhy{}\PYGZhy{}\PYGZhy{}\PYGZhy{}\PYGZhy{}\PYGZhy{}\PYGZhy{}\PYGZhy{}}
\PYG{g+go}{Intercept     19.6882      1.205     16.338      0.000        17.303    22.073}
\PYG{g+go}{DBH            2.1614      0.115     18.757      0.000         1.933     2.389}
\PYG{g+go}{==============================================================================}
\PYG{g+go}{Omnibus:                        2.658   Durbin\PYGZhy{}Watson:                   0.995}
\PYG{g+go}{Prob(Omnibus):                  0.265   Jarque\PYGZhy{}Bera (JB):                2.115}
\PYG{g+go}{Skew:                          \PYGZhy{}0.251   Prob(JB):                        0.347}
\PYG{g+go}{Kurtosis:                       3.385   Cond. No.                         23.8}
\PYG{g+go}{==============================================================================}
\end{Verbatim}

Read more about pandas at \href{http://pandas.pydata.org/}{http://pandas.pydata.org/}

\end{fulllineitems}

\index{save() (fuels.Inventory method)}

\begin{fulllineitems}
\phantomsection\label{Inventory:fuels.Inventory.save}\pysiglinewithargsret{\bfcode{save}}{\emph{outputPath}}{}
Writes formated fvs tree files to specified location

If multiple stands exist in the FVS\_TreeInit then the same number of
files will be created in the specified directory.  The file names
will be the same as the Stand\_ID with a \code{.tre} extension.
\begin{quote}\begin{description}
\item[{Parameters}] \leavevmode
\textbf{\texttt{outputPath}} (\emph{\texttt{string}}) -- directory to store output .tre files

\end{description}\end{quote}

\begin{notice}{note}{Note:}
This method will throw an error if it is called prior to the
\code{format\_fvs\_tree\_file()} method.
\end{notice}

\end{fulllineitems}

\index{set\_FVS\_variant() (fuels.Inventory method)}

\begin{fulllineitems}
\phantomsection\label{Inventory:fuels.Inventory.set_FVS_variant}\pysiglinewithargsret{\bfcode{set\_FVS\_variant}}{\emph{var}}{}
Sets FVS variant. This is need if converting from USDA plant symbols
(PSME) to alpha codes (DF). If so, then all stand you wish to process
must be of the same FVS variant
\begin{quote}\begin{description}
\item[{Parameters}] \leavevmode
\textbf{\texttt{var}} (\emph{\texttt{string}}) -- FVS variant (`iec', `emc' ...)

\end{description}\end{quote}

\end{fulllineitems}


\end{fulllineitems}



\section{intervene module}
\label{intervene:module-intervene}\label{intervene:intervene-module}\label{intervene::doc}\index{intervene (module)}
The intervene module is a collection of treatment algorithms

Contents:


\subsection{SpaceCrowns}
\label{SpaceCrowns:spacecrowns}\label{SpaceCrowns::doc}\index{SpaceCrowns (class in intervene)}

\begin{fulllineitems}
\phantomsection\label{SpaceCrowns:intervene.SpaceCrowns}\pysiglinewithargsret{\strong{class }\code{intervene.}\bfcode{SpaceCrowns}}{\emph{trees}}{}
Bases: \code{intervene.BaseSilv}
\begin{quote}\begin{description}
\item[{Variables}] \leavevmode
\textbf{\texttt{crown\_space}} -- instance variable for crown spacing; initial value = 0

\end{description}\end{quote}
\index{add\_to\_treatment\_collection() (intervene.SpaceCrowns method)}

\begin{fulllineitems}
\phantomsection\label{SpaceCrowns:intervene.SpaceCrowns.add_to_treatment_collection}\pysiglinewithargsret{\bfcode{add\_to\_treatment\_collection}}{\emph{treatment}, \emph{ID}}{}
Adds treatment to static class attribute in intervene.BaseSilv()

\end{fulllineitems}

\index{clear\_treatment\_collection() (intervene.SpaceCrowns method)}

\begin{fulllineitems}
\phantomsection\label{SpaceCrowns:intervene.SpaceCrowns.clear_treatment_collection}\pysiglinewithargsret{\bfcode{clear\_treatment\_collection}}{}{}
Deletes all treatment currently in the treatment collection class
attribute

\end{fulllineitems}

\index{get\_distance() (intervene.SpaceCrowns method)}

\begin{fulllineitems}
\phantomsection\label{SpaceCrowns:intervene.SpaceCrowns.get_distance}\pysiglinewithargsret{\bfcode{get\_distance}}{\emph{tree\_a}, \emph{tree\_b}}{}
Calculate the distance between two trees

Uses Pythagoras' theorem to calculate distance between two tree crowns
in units of input data frame
\begin{quote}\begin{description}
\item[{Parameters}] \leavevmode\begin{itemize}
\item {} 
\textbf{\texttt{tree\_a}} (\emph{\texttt{int}}) -- indexed row of tree a in Pandas data frame

\item {} 
\textbf{\texttt{tree\_b}} (\emph{\texttt{int}}) -- indexed row of tree b in Pandas data frame

\end{itemize}

\item[{Returns}] \leavevmode
distance between two crowns in units of input data frame

\item[{Return type}] \leavevmode
float

\end{description}\end{quote}

\end{fulllineitems}

\index{get\_extent() (intervene.SpaceCrowns method)}

\begin{fulllineitems}
\phantomsection\label{SpaceCrowns:intervene.SpaceCrowns.get_extent}\pysiglinewithargsret{\bfcode{get\_extent}}{}{}
Returns bounding box of tree coordinates
\begin{quote}\begin{description}
\item[{Returns}] \leavevmode
{[}min\_x, min\_y, max\_x, max\_y{]}

\item[{Return type}] \leavevmode
list of floats

\item[{Examples}] \leavevmode
\begin{Verbatim}[commandchars=\\\{\}]
\PYG{g+gp}{\PYGZgt{}\PYGZgt{}\PYGZgt{} }\PYG{k+kn}{from} \PYG{n+nn}{standfire.intervene} \PYG{k+kn}{import} \PYG{n}{SpaceCrowns}
\PYG{g+gp}{\PYGZgt{}\PYGZgt{}\PYGZgt{} }\PYG{n}{space} \PYG{o}{=} \PYG{n}{SpaceCrowns}\PYG{p}{(}\PYG{l+s}{\PYGZdq{}}\PYG{l+s}{/Users/standfire/test\PYGZus{}trees.csv}\PYG{l+s}{\PYGZdq{}}\PYG{p}{)}
\PYG{g+gp}{\PYGZgt{}\PYGZgt{}\PYGZgt{} }\PYG{n}{bbox} \PYG{o}{=} \PYG{n}{space}\PYG{o}{.}\PYG{n}{get\PYGZus{}extent}\PYG{p}{(}\PYG{p}{)}
\PYG{g+gp}{\PYGZgt{}\PYGZgt{}\PYGZgt{} }\PYG{n}{bbox}
\PYG{g+go}{[1.3, 3.5, 63.1, 61.4]}
\end{Verbatim}

\end{description}\end{quote}

\end{fulllineitems}

\index{get\_treatment\_options() (intervene.SpaceCrowns method)}

\begin{fulllineitems}
\phantomsection\label{SpaceCrowns:intervene.SpaceCrowns.get_treatment_options}\pysiglinewithargsret{\bfcode{get\_treatment\_options}}{}{}
Returns dictionary of treatment options
\begin{quote}\begin{description}
\item[{Returns}] \leavevmode
treatment option codes and description

\item[{Return type}] \leavevmode
dictionary

\end{description}\end{quote}

\end{fulllineitems}

\index{get\_trees() (intervene.SpaceCrowns method)}

\begin{fulllineitems}
\phantomsection\label{SpaceCrowns:intervene.SpaceCrowns.get_trees}\pysiglinewithargsret{\bfcode{get\_trees}}{}{}
Returns the trees data frame of the object
\begin{quote}\begin{description}
\item[{Returns}] \leavevmode
trees data frame

\item[{Return type}] \leavevmode
Pandas.DataFrame

\end{description}\end{quote}

\end{fulllineitems}

\index{set\_crown\_space() (intervene.SpaceCrowns method)}

\begin{fulllineitems}
\phantomsection\label{SpaceCrowns:intervene.SpaceCrowns.set_crown_space}\pysiglinewithargsret{\bfcode{set\_crown\_space}}{\emph{crown\_space}}{}
Sets spacing between crowns for the treatment
\begin{quote}\begin{description}
\item[{Parameters}] \leavevmode
\textbf{\texttt{crown\_space}} (\emph{\texttt{float}}) -- crown spacing in units of input data frame

\end{description}\end{quote}

\end{fulllineitems}

\index{set\_extent() (intervene.SpaceCrowns method)}

\begin{fulllineitems}
\phantomsection\label{SpaceCrowns:intervene.SpaceCrowns.set_extent}\pysiglinewithargsret{\bfcode{set\_extent}}{\emph{min\_x}, \emph{min\_y}, \emph{max\_x}, \emph{max\_y}}{}
Sets extent instance variable
\begin{quote}\begin{description}
\item[{Parameters}] \leavevmode\begin{itemize}
\item {} 
\textbf{\texttt{min\_x}} (\emph{\texttt{float}}) -- minimum x coordinate

\item {} 
\textbf{\texttt{min\_y}} (\emph{\texttt{float}}) -- minimum y coordinate

\item {} 
\textbf{\texttt{max\_x}} (\emph{\texttt{float}}) -- maximum x coordinate

\item {} 
\textbf{\texttt{max\_y}} (\emph{\texttt{float}}) -- maximum y coordinate

\end{itemize}

\end{description}\end{quote}

\begin{notice}{note}{Note:}
\code{set\_extent} is automatically called by \code{BaseSilv()}
constructor
\end{notice}

\end{fulllineitems}

\index{treat() (intervene.SpaceCrowns method)}

\begin{fulllineitems}
\phantomsection\label{SpaceCrowns:intervene.SpaceCrowns.treat}\pysiglinewithargsret{\bfcode{treat}}{}{}
Treatment algorithm for removing trees based on input crown spacing

\begin{notice}{note}{Todo}

Optimize algorithm by incorporating \code{search\_rad}.
\end{notice}

\begin{notice}{note}{Todo}

split this function into 3
\end{notice}

\end{fulllineitems}

\index{treatment\_collection (intervene.SpaceCrowns attribute)}

\begin{fulllineitems}
\phantomsection\label{SpaceCrowns:intervene.SpaceCrowns.treatment_collection}\pysigline{\bfcode{treatment\_collection}\strong{ = \{\}}}
\end{fulllineitems}


\end{fulllineitems}



\chapter{Indices and tables}
\label{index:indices-and-tables}\begin{itemize}
\item {} 
\DUspan{xref,std,std-ref}{genindex}

\item {} 
\DUspan{xref,std,std-ref}{modindex}

\item {} 
\DUspan{xref,std,std-ref}{search}

\end{itemize}


\renewcommand{\indexname}{Python Module Index}
\begin{theindex}
\def\bigletter#1{{\Large\sffamily#1}\nopagebreak\vspace{1mm}}
\bigletter{f}
\item {\texttt{fuels}}, \pageref{fuels:module-fuels}
\indexspace
\bigletter{i}
\item {\texttt{intervene}}, \pageref{intervene:module-intervene}
\end{theindex}

\renewcommand{\indexname}{Index}
\printindex
\end{document}
